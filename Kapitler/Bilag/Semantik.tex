% Nødvendig for pæn opsætning
\renewcommand{\syntleft}{\normalfont\itshape}
\renewcommand{\syntright}{$$}
% Slut

Alle transition rules er Bigstep semantics

\tocless \section{Syntaktiske kategorier}

Der adskilles ikke mellem arismetiske og boolske expressions i de syntaktiske kategorier, da disse kan defineres sammen på grund af typesystemet. Eventuelle tvetydigheder løses i den komplette kontekst-fri grammatik (bilag \ref{bil:cfg})

\begin{itemize}
    \item $ i \in \textbf{Int} $ - Heltal
    \item $ fp \in \textbf{FPoints} $ - Decimaltal
    \item $ e \in \textbf{Exp} $ - Expressions
    \item $ x \in \textbf{Var} $ - Variabler
    \item $ St \in \textbf{Stm} $ - Statements
    \item $ T \in \textbf{Types} $ - Typer
    \item $ X \in \textbf{Svar} $ Struct-variabler
    \item $ SF \in \textbf{SFnames} $ Struct-funktioner
    \item $ F \in \textbf{Fnames} $ - Funktionsnavne
    \item $ S \in \textbf{Snames} $ - Struct-navne
    \item $ A \in \textbf{Anames} $ - Array-navne
    \item $ D_V \in \textbf{DecV} $ - Deklaration af variabler
    \item $ D_F \in \textbf{DecF} $ - Deklaration af funktioner
    \item $ D_S \in \textbf{DecS} $ - Deklaration af structs
    \item $ D_A \in \textbf{DecA} $ - Deklaration af arrays
\end{itemize}
\pagebreak

\tocless \section{Formation rules}
\noindent Formation rules, også kendt som den abstrakte syntaks, findes i grammatik \ref{gra:formationrules}

\begin{Grammar}
 \begin{grammar}

 <X> ::= $x$ | $S.X$
 
 <SF> ::= $F$ | $S.SF$

 <e> ::= $n$ | $fp$ | $X$  | $e_1$ "+" $e_2$ | $e_1$ "-" $e_2$ | $e_1$ "*" $e_2$ | $e_1$ "/" $e_2$ | $e_1$ "\%" $e_2$ | "("$e_1$")" | $x$ "=" $e_1$ | $e$ "?" $e_1$ ":" $e_2$ | $e$"++" | $e$"--" | "-"$e$ | $A[e]$ | "true" | "false" | $e_1$ "==" $e_2$ | $e_1$ != $e_2$ | $e_1$ "<" $e_2$ | $e_1$ "<=" $e_2$ | $e_1$ ">" $e_2$ | $e_1$ ">=" $e_2$ | $e_1$ "\&\&" $e_2$ | $e_1$ "||" $e_2$ | "!"$e_1$ | $\epsilon$

 <T> ::= $i$ | $fp$ | "bool"
 
 <St> ::= "{" $Sts$ "}" | "if" "("$e_1$")" $St$ | "if" "("$e_1$")" $St_1$ "else" $St_2$ | "while" $e$ $St$ | $F$"("$P_F$")" | "for" "("$e$";" $e$";" $e$")" $St$ | $x$ "+=" $e$ | $x$ "-=" $e$ | $X$ "*=" $e$ | $x$ "/=" $e$ | $x$ "\%=" $e$ | $\epsilon$
 
 <Sts> ::= $St$";" $Sts$ | $\epsilon$
 
 <$D_V$> ::= $T$ $x$ "=" $e$";"
 
 <$D_F$> ::= $T$ $F$"("$P_A$")" "{" $St$ "}"
 
 <$D_S$> ::= "struct" $S$ "{" $Sts$ "}"
 
 <$D_A$> ::= $T$ A "[" $e$ "]"
 
 <$P_F$> ::= $T$ x, $P_A$ | $T$ x | $\epsilon$
 
 <$P_F$> ::= e, $P_F$ | e | $\epsilon$

 \end{grammar}
 \caption{Formation rules}\label{gra:formationrules}
\end{Grammar}

\tocless \section{Environments}

\noindent Der bruges environment-store modellen, hvilket betyder at alle variables skal gemmes i et environment, hvor variablen er associeret med værdien.

$$ \textbf{EnvV} = \textbf{Var} \cup \{next\} \rightharpoonup \textbf{Loc} $$

\noindent samt

$$ \textbf{Sto} = \textbf{Loc} \rightharpoonup \mathds{Z} $$

\noindent Derudover skal structs og funktioner også have et environment for sig selv. Da disse hænger sammen med de andre environments, får vi at

$$ \textbf{EnvF} = \textbf{Fnames} \rightharpoonup \textbf{EnvV} \times \textbf{EnvF} \times \textbf{EnvS} $$

\noindent og

$$ \textbf{EnvS} = \textbf{Snames} \rightharpoonup \textbf{EnvV} \times \textbf{EnvF} \times \textbf{EnvS} $$
\pagebreak

\tocless \section{Typechecker}
\noindent I forbindelse med udarbejdelsen af typechecker, er det nødvendigt at opstille transition rules for disse. Der tages udgangspunkt i simple formation rules

\begin{grammar}
    \centering
    <B> ::= "Bool" | "Int" | "Float"
    
    <T> ::= $B$ | $x$ ":" $B$ $\rightarrow$ "ok"
\end{grammar}

\noindent hvor \textit{Bool} angiver sættet $\{\top, \bot\}$ for henholdsvis sande og falske boolske værdier - \textit{i} og \textit{fp} er tidligere defineret. Det er derudover også nødvendigt ved et type environment, som indeholder typer af variabler og funktionsnavne defineret ved 

$$ E : \textbf{Var} \cup \textbf{Fnames} \cup \textbf{Snames} \rightharpoonup \textbf{Types} $$

\noindent Derudover kan der skrives $E[x \rightarrow T]$ for type environment $E^\prime$, som er defineret ved 

\[ E^\prime(y) =
  \begin{cases}
    E(y)       & \quad \text{if } y \ne x\\
    T          & \quad \text{if } y = x\\
  \end{cases}
\]

\noindent Det kan generelt siges at hvis disse regler er overholdt, er programmet "well-typed"\mbox{}.

\tocless \subsection{Expressions}

\begin{align*}
&[\text{INT}]           &  &E \vdash i : \text{Int} & 
&[\text{FLOAT}]           &  &E \vdash fp : \text{Float}\\\\
%
&[\text{PAR}]           &  &\frac{E \vdash e_1 : T}{E \vdash (e_1) : T} & 
&[\text{VAR}]           &  &\frac{E(x) = T}{E \vdash x : T}\\\\
%
&[\top]           &  &E \vdash \text{true} : \text{Bool} & 
&[\bot]           &  &E \vdash \text{false} : \text{Bool}\\\\
%
&[\text{INC}_I]           &  &\frac{E \vdash e_1 : \text{Int}}{E \vdash e_1\text{++} : \text{Int}} & 
&[\text{INC}_F]           &  &\frac{E \vdash e_1 : \text{Float}}{E \vdash e_1\text{++} : \text{Float}}\\\\
%
&[\text{DEC}_I]           &  &\frac{E \vdash e_1 : \text{Int}}{E \vdash e_1\text{-\medskip-} : \text{Int}} & 
&[\text{DEC}_F]           &  &\frac{E \vdash e_1 : \text{Float}}{E \vdash e_1\text{-\medskip-} : \text{Float}}\\\\
%
&[\text{NEG}_I]           &  &\frac{E \vdash e_1 : \text{Int}}{E \vdash -e_1 : \text{Int}} & 
&[\text{NEG}_F]           &  &\frac{E \vdash e_1 : \text{Float}}{E \vdash -e_1 : \text{Float}}\\\\
%
&[\text{NEG}_B]           &  &\frac{E \vdash e_1 : \text{Bool}}{E \vdash\; !\smallskip e_1 : \text{Bool}} &
&[\text{ASS}]           &  &\frac{E \vdash x : \text{T}\quad E \vdash e : \text{T}}{E \vdash x = e : \text{T}}\\\\
%
&[\text{ASMD}_{II}]           &  &\frac{E \vdash e_1 : \text{Int} \enskip E \vdash e_2 : \text{Int}}{E \vdash e_1\; op \;e_2 : \text{Int}} & &[\text{ASMD}_{FF}]           &  &\frac{E \vdash e_1 : \text{Float} \enskip E \vdash e_2 : \text{Float}}{E \vdash e_1\; op \;e_2 : \text{Int}}\\    
& & &\text{where}\; op \in \{+, -, *, /, \%\} & & & &\text{where}\; op \in \{+, -, *, /\}\\\\
%
%&[\text{MOD}_{II}]           &  &\frac{E \vdash e_1 : \text{Int} \enskip E \vdash e_2 : \text{Int}}{E \vdash e_1\; \% \;e_2 : \text{Int}} &
&[\text{COM}_{BB}]           &  &\frac{E \vdash e_1 : \text{Bool} \enskip E \vdash e_2 : \text{Bool}}{E \vdash e_1\; op \;e_2 : \text{Bool}}&&[\text{ARR}]      &    &\frac{E \vdash A : \text{T}\quad E \vdash e : \text{Int}}{E \vdash \text{A[}e\text{]} : \text{T}}\\
& & &\text{where}\; op \in \{\&\&, ||\}\\\\
%
&[\text{COM}_{II}]           &  &\frac{E \vdash e_1 : \text{Int} \enskip E \vdash e_2 : \text{Int}}{E \vdash e_1\; op \;e_2 : \text{Bool}} & &[\text{COM}_{FF}]           &  &\frac{E \vdash e_1 : \text{Float} \enskip E \vdash e_2 : \text{Float}}{E \vdash e_1\; op \;e_2 : \text{Bool}}\\    
& & &\text{where}\; op \in & & & &op \in\\
& & &\{==, !=, <, <=, >, >=\}  & & & &\{==, !=, <, <=, >, >=\}\\\\
%
%&[\text{ARR}]      &    &\frac{E \vdash A : \text{T}\quad E \vdash e : \text{Int}}{E \vdash \text{A[}e\text{]} : \text{T}}\\\\
%
&[\text{TERN}]      &    &\frac{E \vdash e_1 : \text{Bool} \enskip E \vdash e_2 : T \enskip E \vdash e_3 : T}{E \vdash (e_1 ?\; e_2 : e_3) : \text{T}} \span\span\span\span\span
\end{align*}

\tocless \subsection{Declarations}
\begin{align*}
&[EMPTY] & &E \vdash \epsilon : \text{ok}\\\\
%
&[VAR] & &\frac{E[x \mapsto T] \vdash D_V : \text{ok}\quad E \vdash e : \text{T}}{E \vdash T x = e; D_V : \text{ok}}\\\\
%
&[FUNC] & &\frac{E[F \mapsto (x : T_1 \rightarrow \text{ok}, F : T_2 \rightarrow \text{ok})] \vdash D_F : \text{ok}}{E \vdash T_1\; F (T_2\; x)\; \{\; St\; \}\; D_F : \text{ok}}\\\\
%
&[STRUCT] & &\frac{E[S \mapsto (x : T \rightarrow \text{ok})] \vdash D_S : \text{ok}}{E \vdash \text{struct}\; S\; \{\;T\; x\;\}\; D_S: \text{ok}}\\\\
%
&[ARRAY] & &\frac{E[A \mapsto T] \vdash D_A : \text{ok} \quad E \vdash e : \text{Int}}{E \vdash T A [e]; D_A : \text{ok}}
\end{align*}

\tocless \subsection{Statements}

\begin{align*}
&[BRACES] & & \frac{E\vdash Sts : \text{ok}}{E\vdash\{\;Sts\;\} : \text{ok}}\\\\
&[COMP] & &\frac{E \vdash St : \text{ok}\quad E \vdash Sts : \text{ok}}{E \vdash St;\;Sts : \text{ok}}\\\\
&[IF] & &\frac{E \vdash e : \text{Bool}\quad E \vdash St : \text{ok}}{E \vdash \text{if}\; (e)\;  St : \text{ok}}\\\\
&[IF-ELSE] & &\frac{E \vdash e : \text{Bool}\quad E \vdash St_1 : \text{ok}\quad E \vdash St_2 : \text{ok}}{E \vdash \text{if}\; (e)\; St_1\; \text{else} \;St_2 : \text{ok}}\\\\
&[WHILE] & &\frac{E \vdash e : \text{Bool}\quad E \vdash St : \text{ok}}{E \vdash while\; (e)\; St : \text{ok}}\\\\
&[FOR] & &\frac{E \vdash e_1 : \text{ok}\enskip E \vdash e_2 : \text{bool}\enskip E \vdash e_3 : \text{ok}\enskip E \vdash : \text{ok}}{E \vdash \text{for } (\; e_1;\; e_2;\; e_3\;) \; St : \text{ok}}\\\\
&[CALL] & &\frac{E \vdash F : (x : T \rightarrow \text{ok})\quad E \vdash e : \text{T}}{E \vdash F(e) : \text{ok}}\\\\
&[CPS] & &\frac{E \vdash x : T\quad E \vdash e : T}{E \vdash x\; op\; e : \text{ok}}\\
& & &\text{where}\; op \in \{+=, -=, *=, *=, \%=\}
\end{align*}


\tocless \section{Transition rules}
\noindent Da der arbejdes med en del environments samt storage, er det fordelagtigt et definere det samlet, så der derved spares plads

$$ Env = \{env_V, env_F, env_S\} $$
$$ SEnv = \{Env, sto\}$$

\noindent Transitionssystemet er defineret ved

$$ (\textbf{Expr} \cup \mathds{Z} \cup \{\top, \bot\}, \rightarrow_e, \mathds{Z} \cup \{\top, \bot\} $$

\tocless \subsection{Expressions}

$(\Gamma_\textbf{Expr},\to_e,T_\textbf{Expr})$ er defineret neden under.

Konfigurationer
\[
    \Gamma_\textbf{Expr} = \textbf{Expr} \cup \mathbb{R} \cup \top \cup \bot \cup \emptyset 
\]
Terminerende konfigurationer
\[
    (T_\textbf{Expr}=\mathbb{R}\cup\top\cup\bot\cup \emptyset)
\]
Transitioner i dette transitions system er relativt til et environment-store pair og er derfor i formen
\[
    SEnv \vdash e \to_e (v, sto')
\]
$\to_e$ er defineret som den mindste relation der tilfredstiller reglerne fra de næste to sektioner. For at gøre disse regler mere overskuelige, er expressions delt ind i arismetiske og boolske expressions, og ikke relavante værdier er ikke vist, disse skal bare antages som at være uændret.\\

\noindent \subsubsection{Arismetiske}
\label{ssec:aexpr}

\begin{align*}
&[VAR] & &SEnv \vdash x \rightarrow_e v & &\text{if } x = l \text{ and } sto\; l = v\\\\
&[SVAR] & &\frac{env_S^\prime, env_V^\prime \vdash X \rightarrow l}{env_S, env_V, sto \vdash s.X \rightarrow l} & &\text{where } env_S\; s = (env_V^\prime, env_F^\prime, env_S^\prime)\\\\
&[ADD] & &\frac{SEnv \vdash e_1 \rightarrow_e\; v_1\quad SEnv \vdash e_2 \rightarrow_e\; v_2}{SEnv \vdash e_1 + e_2 \rightarrow_e\; v} & &\text{where}\; v = v_1 + v_2\\\\
&[SUB] & &\frac{SEnv \vdash e_1 \rightarrow_e\; v_1\quad SEnv \vdash e_2 \rightarrow_e\; v_2}{SEnv \vdash e_1 - e_2 \rightarrow_e\; v} & &\text{where}\; v = v_1 - v_2\\\\
&[MULT] & &\frac{SEnv \vdash e_1 \rightarrow_e\; v_1\quad SEnv \vdash e_2 \rightarrow_e\; v_2}{SEnv \vdash e_1 * e_2 \rightarrow_e\; v} & &\text{where}\; v = v_1 \cdot v_2\\\\
&[DIV] & &\frac{SEnv \vdash e_1 \rightarrow_e\; v_1\quad SEnv \vdash e_2 \rightarrow_e\; v_2}{SEnv \vdash e_1\; /\; e_2 \rightarrow_e\; v} & &\text{where}\; v = \frac{v_1}{v_2}\\\\
&[MOD] & &\frac{SEnv \vdash e_1 \rightarrow_e\; v_1\quad SEnv \vdash e_2 \rightarrow_e\; v_2}{SEnv \vdash e_1\; \%\; e_2 \rightarrow_e\; v} & &\text{where}\; v = v_1 - (v_2 \cdot \floor*{\frac{v_1}{v_2}})\\\\
&[PAR] & &\frac{SEnv \vdash e_1 \rightarrow_e v_1}{SEnv \vdash (e_1) \rightarrow_e v_1}\\\\
&[ASS] & &Env \vdash \langle x = e, sto \rangle \rightarrow_e sto[l \mapsto v] & &\text{where } SEnv \vdash e \rightarrow_e v\\
& & & & &\text{ and } SEnv_V\; x = l\\\\
&[TERN_\top] & &\frac{SEnv \vdash e_2 \rightarrow_e v_1}{SEnv \vdash e_1 \text{ ? } e_2 \text{ : } e_3 \rightarrow_e v_1} & &\text{if } SEnv \vdash e_1 \rightarrow_e \top\\\\
&[TERN_\bot] & &\frac{SEnv \vdash e_3 \rightarrow_e v_2}{SEnv \vdash e_1 \text{ ? } e_2 \text{ : } e_3 \rightarrow_e v_2} & &\text{if } SEnv \vdash e_1 \rightarrow_e \bot\\\\
&[INC_{POST}] & &\frac{SEnv \vdash e \rightarrow v}{Env \vdash \langle e\text{++}, sto \rangle \rightarrow \langle v, sto[e \mapsto v^\prime] \rangle} & &\text{where }v^\prime = v + 1\\\\
&[DEC_{POST}] & &\frac{SEnv \vdash e \rightarrow v}{Env \vdash \langle e\text{- -}, sto \rangle \rightarrow \langle v, sto[e \mapsto v^\prime] \rangle} & &\text{where }v^\prime = v - 1\\\\
&[INC_{PRE}] & &\frac{SEnv \vdash e \rightarrow v}{Env \vdash \langle \text{++}e, sto \rangle \rightarrow \langle v^\prime, sto[e \mapsto v^\prime] \rangle} & &\text{where }v^\prime = v + 1\\\\
&[DEC_{PRE}] & &\frac{SEnv \vdash e \rightarrow v}{Env \vdash \langle \text{- -}e, sto \rangle \rightarrow \langle v^\prime, sto[e \mapsto v^\prime] \rangle} & &\text{where }v^\prime = v - 1\\\\
&[NEG_{FI}] & &\frac{SEnv \vdash e_1 \rightarrow_e v_1}{SEnv \vdash -e_1 \rightarrow_e -v_1}  & &\\\\
\end{align*}

\subsubsection{Boolske}

\begin{align*}
&[EQ_\top] & &\frac{SEnv \vdash e_1 \rightarrow_e v_1 \quad SEnv \vdash e_2 \rightarrow_e v_2}{SEnv \vdash e_1 == e_2 \rightarrow_e \top} & &\text{if } v_1 = v_2\\\\
&[EQ_\bot] & &\frac{SEnv \vdash e_1 \rightarrow_e v_1 \quad SEnv \vdash e_2 \rightarrow_e v_2}{SEnv \vdash e_1 == e_2 \rightarrow_e \bot} & &\text{if } v_1 \ne v_2\\\\
&[NEQ_\top] & &\frac{SEnv \vdash e_1 \rightarrow_e v_1 \quad SEnv \vdash e_2 \rightarrow_e v_2}{SEnv \vdash e_1\; != e_2 \rightarrow_e \top} & &\text{if } v_1 \ne v_2\\\\
&[NEQ_\bot] & &\frac{SEnv \vdash e_1 \rightarrow_e v_1 \quad SEnv \vdash e_2 \rightarrow_e v_2}{SEnv \vdash e_1\; != e_2 \rightarrow_e \bot} & &\text{if } v_1 = v_2\\\\
&[LS_\top] & &\frac{SEnv \vdash e_1 \rightarrow_e v_1 \quad SEnv \vdash e_2 \rightarrow_e v_2}{SEnv \vdash e_1 < e_2 \rightarrow_e \top} & &\text{if } v_1 < v_2\\\\
&[LS_\bot] & &\frac{SEnv \vdash e_1 \rightarrow_e v_1 \quad SEnv \vdash e_2 \rightarrow_e v_2}{SEnv \vdash e_1 < e_2 \rightarrow_e \bot} & &\text{if } v_1 \nless v_2\\\\
&[LSE_\top] & &\frac{SEnv \vdash e_1 \rightarrow_e v_1 \quad SEnv \vdash e_2 \rightarrow_e v_2}{SEnv \vdash e_1 <= e_2 \rightarrow_e \top} & &\text{if } v_1 \leq v_2\\\\
&[LSE_\bot] & &\frac{SEnv \vdash e_1 \rightarrow_e v_1 \quad SEnv \vdash e_2 \rightarrow_e v_2}{SEnv \vdash e_1 <= e_2 \rightarrow_e \bot} & &\text{if } v_1 \nleq v_2\\\\
&[GR_\top] & &\frac{SEnv \vdash e_1 \rightarrow_e v_1 \quad SEnv \vdash e_2 \rightarrow_e v_2}{SEnv \vdash e_1 > e_2 \rightarrow_e \top} & &\text{if } v_1 > v_2\\\\
&[GR_\bot] & &\frac{SEnv \vdash e_1 \rightarrow_e v_1 \quad SEnv \vdash e_2 \rightarrow_e v_2}{SEnv \vdash e_1 > e_2 \rightarrow_e \bot} & &\text{if } v_1 \ngtr v_2\\\\
&[GRE_\top] & &\frac{SEnv \vdash e_1 \rightarrow_e v_1 \quad SEnv \vdash e_2 \rightarrow_e v_2}{SEnv \vdash e_1 >= e_2 \rightarrow_e \top} & &\text{if } v_1 \geq v_2\\\\
&[GRE_\bot] & &\frac{SEnv \vdash e_1 \rightarrow_e v_1 \quad SEnv \vdash e_2 \rightarrow_e v_2}{SEnv \vdash e_1 >= e_2 \rightarrow_e \bot} & &\text{if } v_1 \ngeq v_2\\\\
&[AND_\top] & &\frac{SEnv \vdash e_1 \rightarrow_e \top \quad SEnv \vdash e_2 \rightarrow_e \top}{SEnv \vdash e_1\; \&\&\; e_2 \rightarrow_e \top}\\\\
&[AND_{\top\bot}] & &\frac{SEnv \vdash e_1 \rightarrow_e \top \quad SEnv \vdash e_2 \rightarrow_e \bot}{SEnv \vdash e_1\; \&\&\; e_2 \rightarrow_e \bot}\\\\
&[AND_\bot] & &\frac{SEnv \vdash e_1 \rightarrow_e \bot \quad SEnv \vdash e_2 \rightarrow_e \bot}{SEnv \vdash e_1\; \&\&\; e_2 \rightarrow_e \bot}\\\\
&[OR_{\top}] & &\frac{SEnv \vdash e_1 \rightarrow_e \top \quad SEnv \vdash e_2 \rightarrow_e \top}{SEnv \vdash e_1\; ||\; e_2 \rightarrow_e \top}\\\\
&[OR_{\bot\top}] & &\frac{SEnv \vdash e_1 \rightarrow_e \bot \quad SEnv \vdash e_2 \rightarrow_e \top}{SEnv \vdash e_1\; ||\; e_2 \rightarrow_e \top}\\\\
&[OR_{\bot}] & &\frac{SEnv \vdash e_1 \rightarrow_e \bot \quad SEnv \vdash e_2 \rightarrow_e \bot}{SEnv \vdash e_1\; ||\; e_2 \rightarrow_e \bot}\\\\
&[NEG_\top] & &\frac{SEnv \vdash e \rightarrow_e \top}{SEnv \vdash\; !e \rightarrow_e \bot}\\\\
&[NEG_\bot] & &\frac{SEnv \vdash e \rightarrow_e \bot}{SEnv \vdash\; !e \rightarrow_e \top}\\\\
\end{align*}

\tocless \subsection{Statements}

$(\Gamma_\textbf{Stmt},\to_e,T_\textbf{Stmt})$ er defineret neden under.

\noindent \textit{Konfigurationer}
\[
    \Gamma_\textbf{Stmt} = (\textbf{Stmt}\times\textbf{SEnv})\cup(\textbf{SEnv}) 
\]
\textit{Terminerende konfigurationer}
\[
    (T_\textbf{Stmt}=\textbf{SEnv})
\]
Transitioner i dette transitions system er relativt til et environment-store pair og er derfor i formen
\[
    SEnv \vdash e \to_e SEnv'
\]
$\to$ er defineret som den mindste relation der tilfredstiller reglerne fra næste sektion. For at gøre disse regler mere overskuelige, er ikke relavante værdier er ikke vist, disse skal antages for at være uændret.\\

\begin{align*}
&[VDECL]&&\frac{\langle Sts,env''_V, sto[l\mapsto v]\rangle\to(env'_V,sto')}{\langle T x = e_1, env_V, sto\rangle\to(env'_V,sto')}&&\begin{array}{l}
    \text{where } env_V, sto\vdash a\to_e v\\
    \text{and } l = env_V \text{ next}\\
    \text{and } env_V'' = env_V[x\mapsto l][\text{next}\mapsto \text{new } l]
\end{array}\\\\
%
%&[EMPTY-VAR]&&\langle\epsilon,env_V,sto\rangle\to_{DV}(env_V, sto)\\\\
%
&[FDECL]&&env_V\vdash\langle P_F, env_V\rangle\to env'_V\\
&&&\frac{env'_V\vdash env_F[F\mapsto(T,env_V, St)]\to env'_F}{SEnv\vdash T F(P_F) St\to env'_F}\\\\
%
&[SDECL]&&env_V, env_F\vdash St\to (env'_V, env_F')\\
&&&\frac{env_S[S\mapsto (env_V', env_F')]\to env_S'}{SEnv\vdash \text{struct } S \; \{\;Sts\;\}\to env'_S}\\\\
%
&[COM] & &\frac{Env \vdash \langle St, \; sto \rangle \rightarrow sto^{\prime\prime} \enskip Env \vdash \langle Sts, \; sto^{\prime\prime} \rangle \rightarrow sto^\prime}{Env \vdash \langle St; Sts, \; sto \rangle \rightarrow sto^\prime} \span \span \\\\
%
&[BRACES]&&\frac{Env^\prime\vdash\langle Sts, sto\rangle \to sto^\prime}{Env\vdash\langle\{\;Sts\;\}, sto\rangle\to sto^\prime }& & \text{where } Env^\prime = \text{copy of } Env\\\\
%
&[IF_\top] & &\frac{Env \vdash \langle St, sto \rangle \rightarrow sto^\prime}{Env \vdash \langle \text{if } (e)\; St,\; sto \rangle \rightarrow sto^\prime} & &\text{if } Env \vdash e \rightarrow_e \top\\\\
%
&[IF_\bot] & &Env \vdash \langle \text{if } (e)\; St,\; sto \rangle \rightarrow sto & &\text{if } Env \vdash e \rightarrow_e \bot\\\\
%
&[IF-E_\top] & &\frac{Env \vdash \langle St_1, sto \rangle \rightarrow sto^\prime}{Env \vdash \langle \text{if } (e)\;St_1 \text{ else } St_2,\; sto \rangle \rightarrow sto^\prime} & &\text{if } Env \vdash e \rightarrow_e \top\\\\
%
&[IF-E_\bot] & &\frac{Env \vdash \langle St_2, sto \rangle \rightarrow sto^\prime}{Env \vdash \langle \text{if } (e)\; St_1 \text{ else } St_2,\; sto \rangle \rightarrow sto^\prime} & &\text{if } Env \vdash e \rightarrow_e \bot\\\\
%
&[WHILE_\top] & &Env \vdash \langle St, sto \rangle \rightarrow sto^{\prime\prime}\\
& & &\frac{Env \vdash \langle \text{while } (e)\; St,\; sto^{\prime\prime} \rangle \rightarrow sto^\prime}{Env \vdash \langle \text{while } (e)\; St,\; sto \rangle \rightarrow sto^\prime} & &\text{if } Env \vdash e \rightarrow_e \top\\\\
%
&[WHILE_\bot] & &Env \vdash \langle \text{while } (e)\; St,\; sto \rangle \rightarrow sto & &\text{if } Env \vdash e \rightarrow_e \bot\\\\
&[CPS_+] & &\frac{Env \vdash x \rightarrow_e v_1 \quad Env \vdash e \rightarrow_e v_2}{Env \vdash \langle x \text{ += } e, sto \rangle \rightarrow sto[x \mapsto v]} & &\text{where } v = v_1 + v_2\\\\
%
&[CPS_-] & &\frac{Env \vdash x \rightarrow_e v_1 \quad Env \vdash e \rightarrow_e v_2}{Env \vdash \langle x \text{ -= } e, sto \rangle \rightarrow sto[x \mapsto v]} & &\text{where } v = v_1 - v_2\\\\
%
&[CPS_*] & &\frac{Env \vdash x \rightarrow_e v_1 \quad Env \vdash e \rightarrow_e v_2}{Env \vdash \langle x \text{ *= } e, sto \rangle \rightarrow sto[x \mapsto v]} & &\text{where } v = v_1 \cdot v_2\\\\
%
&[CPS_/] & &\frac{Env \vdash x \rightarrow_e v_1 \quad Env \vdash e \rightarrow_e v_2}{Env \vdash \langle x \text{ /= } e, sto \rangle \rightarrow sto[x \mapsto v]} & &\text{where } v = \frac{v_1}{v_2}\\\\
%
&[CPS_\%] & &\frac{Env \vdash x \rightarrow_e v_1 \quad Env \vdash e \rightarrow_e v_2}{Env \vdash \langle x \text{ \%= } e, sto \rangle \rightarrow sto[x \mapsto v]} & &\text{where } v = v_1 - (v_2 \cdot \floor*{\frac{v_1}{v_2}})\\\\
%
&[CALL] & & \frac{Env^\prime_V[x\mapsto l][next\mapsto l^\prime], Env_F\vdash\langle S, sto\rangle\to sto^\prime}{Env_V, Env_F\vdash\langle F(y), sto\rangle\to sto^\prime} & 
&\begin{array}{l}
    \text{where } Env_F F = (S, x, Env^\prime_V)\\
    \text{and } Env_V y = l\\
    \text{and } l^\prime=Env_V \text{ next}
\end{array} \\\\
&[SFUNC] & &\frac{env_S^\prime, env_F^\prime \vdash SF \rightarrow (St, env_V^{\prime\prime}, env_F^{\prime\prime}, env_S^{\prime\prime})}{env_S, env_F \vdash s.SF \rightarrow (St, env_V^{\prime\prime}, env_F^{\prime\prime}, env_S^{\prime\prime})} & &\text{where } env_S\; s = (env_V^\prime, env_F^\prime, env_S^\prime)
\end{align*}