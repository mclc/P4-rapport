\textit{I indledningen blev \gls{plc}'en introduceret med en kort historisk gennemgang, og en forklaring om hvor \gls{plc}'en oftest anvendes. Problemanalysen vil grave ned i \gls{plc}'ens hardware og software, hvor formålet er at belyse, hvilke potentielle problemstillinger der foreligger. Afslutningsvist vil problemanalysen afdække \gls{plc}'ens begrænsninger. Problemanalysen vil omend muligt, fokusere på redegørelsen af det initierende problem, som ligeledes blev fremført i indledningen.}

%I tidligere kapitel blev den historiske kontekst af programmerings metoder for PLC'er beskrevet, og som man kan se er der sket meget lidt med programmeringsmetoder til PLC'er i forhold til mange andre.\cfix{Dette er en antagelse, feel free til at rette så det passer.} Derfor vil dette kapitel redegøre for hvordan programmering forgår for en PLC med eksempler på programmer i disse sprog.

%Til sidst vil alternativer til disse metoder, der af forskellige grunde ikke er blevet populære, blive gennemgået med fokus på hvad deres styrker og svagheder er, så dette kan blive indraget i overvejelserne i designet af vores sprog.