\chapter{Diskussion}\label{sec:Diskussion}
\textit{Diskussionen gennemgår og belyser projektets udfordrende emner, så som konverteringen fra PLC++ til \gls{il} samt hvordan værdier gemmes i hukommelsen på \gls{plc}en.
Specielt udnyttelse af hukommelsen vil blive diskuteret, herunder brugen af stack. 
Dertil vil beslutningerne som er foretaget igennem projektet blive diskuteret, samt erkendes som værende acceptable eller utilfredsstillende.}

\subsection*{Hukommelse}
En \gls{plc} har en begrænset mængde hukommelse, som er fordelt på flere forskellige hukommelsesområder.
Dette gør at man skal være opmærksom på hvordan hukommelsen benyttes, og derved at man ikke kan allokere uendelig mængder hukommelse.
Der bliver i programmet brugt flere forskellige hukommelsesområder, til hvert deres formål. Dette er med til at sikre at man ikke bruger den samme adresse og derved overskriver en anden værdi, medmindre det er hensigten.\\

\noindent Til variabler benyttes hukommelsesområdet D, sammen med en stack, som er inspireret fra C, herved får man en let og overskueligt måde at adressere en ny hukommelsesadresse.
Her gemmes værdien af variablen på stacken mens, selve variablen bliver gemt på stacken.
Værdien ligges derefter over i hukommelsesområdet D.
Dette giver et par fordele og ulemper, heriblandt at man ikke kan genbruge de adresser som allerede er benyttet. Dette er grundet at man ikke ved hvornår programmet er færdig med at bruge en variable, og man derved ikke kan fjerne den fra dets hukommelsesområde.
Fordelen ved stacken er at man generelt, bortset fra variabler, har en god skik for genbrug af hukommelsesadresser.

Dette er implementeret ved at man i push funktionen til stacken tæller adressen op, og ved pop funktionen tæller den ned igen. Derved undgår man et stort spild af adresser, hvilket er praktisk i forhold til hukommelsesområdets størrelse.
Derudover giver det et godt overblik i Cx One, da programmøren har mulighed for at se værdien af den konkrete variable, og derved giver mulighed for at fejlrette kode direkte i CX-programmer.\\

\noindent Det er blevet vurderet som et incitament for codegeneration, at genbruge hukommelsesadresser så meget som muligt. Derfor er der lagt vægt på at benytte stacken til så mange operationer og værdier som muligt. Da variabler og funktionskald ikke benytter stacken er det ikke muligt at kalde funktioner rekursivt, da variablerne derfor vil overskrive hinanden.
Et alternativ som der blev undersøgt var en optimizer, som kunne hjælpe med at fjerne ubrugte variabler fra hukommelsen, dette blev dog nedprioriteret, da der skal være omtrent 500 variabler før det bliver et problem, og det dermed vurderes til at være irrelevant for en \gls{plc}. Dog vil en optimizer ikke kunne løse alle problemer uden at hukommelseshåndteringen skal benytte en stack.

\subsection*{Paradigme}
Valget af paradigme blev hurtigt truffet i projektet, og på baggrund af egne erfaringer med \gls{plc}'er, \gls{plc}'ernes hardware begrænsninger, samt undersøgelsen af hvad virksomheder benytter \gls{plc}'er til, viste det sig at være mest logisk at vælge det imperativ paradigme.
Det har vist sig at være et fornuftigt valg, da de fleste programmører kommer med en industriuddannelse, og derfor ikke kender til det objektorienterede paradigme.
En anden grund til at det imperative paradigme blev valgt, er fordi det er relativt simpelt, og en almindelig person kan forholde sig til det, med meget lidt erfaringen indenfor sproget. 
Derudover er projektets formål at simplificere programmeringssproget, således at det er let og overskueligt at skrive programmer i det.\\

\noindent Et eksemplet på dette kan ses i kodeeksempel \ref{code:fixedfee-il} som viser hvordan et simpelt program skrives i \gls{il}, og henholdsvis i \gls{lad} vist på figur \ref{fig:fixedfee-graphic}

\noindent Dette giver mulighed for at skrive i et sprog (\gls{il}), som er mere kompakt, og derved tager kortere tid at skrive. Udfordringen ved det, er at det ikke er særlig let at lære, da man skal huske opbygningen af funktioner, samt referencer til de enkelte kald, som man skal foretage.
Ligeledes er der ikke behov for den kompleksitet som det objektorienteret paradigme giver, da \gls{plc}-programmer for det mest er opbygget i flere sektioner, som hver især håndterer hver deres opgave.\\

\noindent \Gls{plc}'er har også nogle meget strenge hardware begrænsninger. For eksempel kan man kun have 256 subroutiner og 256 hop. Det betyder at de mange hop og funktioner i et ikke-optimeret objektorienteret system ville hurtigt få systemet til at overskride sine grænser. Derfor er store projekter, hvor det objektorienterede paradigme virkelig viser sine fordele, ikke mulige. Derved vurderes det at det imperative paradigme er det mest passende, i forhold til de definerede krav, omkring simplicitet samt at det skal være overskueligt for en programmør med industriuddannelsen.

\subsection*{Kravspecifikation}
Kravspecifikation ligger til baggrund for projektets valg af funktioner som skal implementeres.
Her vurderede gruppen at kunne nå at implementere flere forskellige og krævende funktioner som beskrevet i \gls{moscow}-analyse ved tabel \ref{tab:moscow}.
Dette var dog ikke en realitet, og tidspres begyndte at være en faktorer, som gruppen måtte tage stilling til.
Derfor blev det besluttet at afgrænse i det perspektiv at studieordningen skulle overholdes, samt at der skulle kunne genereres noget brugbart kode.
Derfor blev det valgt at prioritere de funktioner som anses for at være essentielle for at programmeringssproget stadig kan bruges. Hertil blev der vurderet at structs ville tage for lang tid at implementere i forhold til dens kompleksitet, samt hvad det ville bidrage til i sproget.
Ulempen ved at afgrænse structs, er selvfølgelig at man ikke får mulighed for at benytte structs, men også at funktioner som afhænger af structs ikke bliver implementeret.
En af disse er Timer som blev specificeret til at skulle bruge structs i grammar, dette er dog ikke særlig smart, da timer ofte bliver brugt på en \gls{plc}, men på daværende tidspunkt, var ideen at man kunne genbruge noget kode, i forhold til at omskrive Timer til en Struct.
Det vurderes til at være et nødvendigt valg at afgrænse denne funktion, for at opretholde studieordningen samt deadlinen og afleveringen af projektet, dog ikke hensigtsmæssigt i forhold til programmeringssprogets funktionalitet.\\

\noindent \gls{moscow}-analysen har ligeledes port som en type som skal implementeres, da det er her \gls{plc}en modtager og udsender data fra.
Det er derfor vigtigt at der er en let måde at håndtere port på, hvilket er håndteret ved at benytte en promilleskala.
Dette giver et højt abstraktionsniveau for programmøren, da han ikke skal benytte kommatal, eller enorme værdier hvilket hurtigt kan gøre det uoverskueligt.
Selvom dette aldrig blev implementeret på grund af tidsmangel, er der nogle ulemper som blev forudset.
Der kan forekomme moduler, som afgiver et analogt inputsignal som er udenfor promilleskalaen, hvilket derfor ikke kan aflæses til fuld præcision. 
Hertil kan der være moduler som har en præcision på mere end promille skalaen kan håndtere, og derved ville man tabe præcision.
Dette kunne have været implementeret ved en hvilken som helst skala der kunne give mere præcision fx 0-1024, hvilket giver mere mening i forhold til den binære repræsentation. Men da det ikke er fast hvor stor en spænding der kan være over output portene, og det virker mere logisk at arbejde i procenter eller promiller, blev det promiller. 

\subsection*{Globale variabler}
Globale variabler er et attraktivt tiltag til programmering, hvorledes man kan have globale variabler tilgængelig for alle funktioner.
Dette giver mulighed for at have en global counter variable, som tælles op når en handling er udført.
Implementationen af globale variabler har ledt an til en problemstilling, som forekommer når PLC++ omskrives til instruction list. Problemet er at \gls{plc}en kun kan håndtere variabler som var de globale.
Derfor kan man ikke redefinere en variable, heller inde i et scope, hvilket giver anledning til problemer.
Dette er tilgængeligt i de fleste programmeringssprog at man kan have en variable som en lokalvariable, og ligeledes have en globalvariable.

\subsection*{Dekoreret \gls{ast}}
En standard del af kompilerings processen er at man dekorere sit \gls{ast} med værdier under kontekst analysen. Det blev først sent i udviklingsforøbet opdaget hvordan SabelCC understøtter dekorering af \gls{ast}. Derfor blev det besluttet at fortsætte med den løsning der var blevet udviklet i mellemtiden. Dette har haft den konsekvens at symboltabellen fik meget af den information man ellers ville have dekoreret med, og da dekoration ikke kunne fortælle noget om fx hvor stor en given variable var, blev stacken implementeret så den største mulige variable er altid det der blev returneret, og derfor bliver alle variabler konverteret til den størrelse værdier.

\subsection*{Stack}
Stacken i sig selv har sine problemer, ud over den har en masse spild plads, så bliver den defineret i \gls{il}-koden uanset om den bliver brugt eller ej. Dette blev implementeret sådan for at sikre at stacken altid var initialisere når den skal bruges og da den bliver brugt til næsten alle udtryk burde det ikke være et problem. Der for kunne man alligevel godt samle information omkring programmet i en optimerings runde hvor man fjernede dette stykke kode hvis det ikke var relevant.

\subsection*{Strings}
Datatypen strings er et array af karaktere som bruges til at kommunikere med andre systemer eller en bruger. Dette kan være meget brugbar til at debugge systemer eller seriel kommunikation. Men det blev ikke implementeret i første version af PLC++. Eftersom en string kan ses som et array af karaktere og der findes allerede gode debugging muligheder for \gls{plc}'er blev dette ikke implementeret.

\subsection*{Port}
Idéen med port var at man havde en simpel måde til at tilgå porte der kan ændres til at referere andre udgange senere. Den første del af dette er implementeret og fungere fint, men man kan ikke udskifte portene, og det betyder at idéen bag port ikke fungere da man kan direkte referere til portene. Det betyder også at der ikke er noget funktionalitet der mangler fra sproget som konsekvens at dette ikke er implementeret. Derfor blev det vurderet til ikke at være høj prioritet, selvom det reducere \textit{writeability} af sproget som blev vurderet til at være meget vigtig i problem beskrivelsen. 