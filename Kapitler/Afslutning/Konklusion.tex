\chapter{Konklusion}
\label{sec:konklusion}

Efter at have fuldendt problemløsningen kan der konkluderes i hvilken grad problemformuleringen er blevet løst.
Problemformuleringen har lagt til grunde for kravspecifikationen, som indeholder udspecificerede funktionaliteter.
Disse funktionaliteter er ved hjælp af en \gls{moscow}-analyse blevet tildelt en prioritet, således at funktionen kunne klassificeres ud fra dens vigtighed af den samlede løsning.

Den samlede løsning vurderes udfra i hvilken grad kriterierne er opfyldt, dette belyses i afsnit \ref{subsec:il-pa}, hvor der påvises hvordan et eksempel i \gls{il} hurtig giver overblik, hvorimod \gls{lad} giver en mere indviklet oversigt. Derimod hvis man opretholder eksemplet fra \gls{il} med PLC++ som kan ses i bilag \label{bil:semantik}, kan man se at koden er væsentlig mindre for PLC++.
Herudover kan programmøren let få et overblik med sine egne funktionsnavne, hvorimod man i \gls{il} har en SBN og SBS block, som giver en reference i form af et id til den \textit{block} den kører.
Derfor menes det at PLC++ giver et bedre overblik, samt en højere \textit{readability} i form af at man ikke behøver at forholde sig til adressering af hukommelsesadresser.
Dertil kunne det tænkes at programmøren har lettere ved at lære PLC++ fremfor \gls{il}, da det har en højere  \textit{readability} samt at programmøren ikke skal lære instructioner, hvilket derfor leder til en højere \textit{learnability}. Dette er en bred påstand som der ikke er evidens for, foruden kodeeksempler som konkret viser kompleksiteten af \gls{il} og PLC++.
Det er dog langt fra alle instrustri tekniker som programmere i \gls{il}, mest af alt på grund af dets kompleksitet som vist i kodeeksemple \ref{code:fixedfee-il}.
Derimod kodes der i \gls{lad}, som førhen er blevet konkluderet som et sprog med dårlig overblik for programmøren.
Når man konkludere på \gls{lad}-programmering mod PLC++, er det vigtigt at man ser på et hvilken målgruppe som der arbejdes med.
Hvilket tidligere er defineret som værende til industri tekniker med nogen programmeringserfaring, samt at programmet har en mellemstor størrelse.
Hvis man derfor kigger på mellemstore programmer gør det sig gældende at \gls{lad}-programmering bliver meget uoverskueligt, og generelt er svært at rette i, hvorimod PLC++ giver et godt overblik.
Kigger man derimod på små programmer har \gls{lad}-programmering en klar fordel, da programmøren ikke behøver at 






Kriterier 4.3 er de opfyldt, kik på kodeeksemplerne og se hvor meget hvert fylder kodeeksemple 2.2

Der erklæres måske for mange ting i programmet, i forhold til hvad brugeren skal bruge. => Hvis man ikke bruge udtryk som 1+1 så skal stacken ikke være der, den bliver dog altid erklæret.

Konklusion, hvornår er vores sprog mere brugbart end andre? - Små programmere store? mellem?

