\chapter{Perspektivering}
\label{sec:Perspektivering}
\textit{Perspektiveringen gennemgår og belyser projektets muligheder for udvikling, samt en fremtidig prognose.
Der vil i perspektiveringen være eksempler på hvordan projektet ville kunne udvides, samt hvordan det kan videreføres til andre platforme.
}

\subsection*{Objektorienteret}
Det objektorienterede paradigme har været afgrænset fra projektet, da det ikke levede op til de opstillede krav, samt at målgruppen ikke var korrekt. 
Der er dog mange fornuftige ting i det objektorienteret paradigme, så som nedarvning hvilket kunne være smart i den forstand, at programmører til \gls{plc}er normalt kode mange af de samme features flere gange, men med små justeringer.
Derfor kunne nedarvning give mening, i forhold til at en programmør derved kunne have en standard klasse, som personen kan kode videre på, til de enkelte moduler.
Dette ville øge abstraktionsniveauet og derved gøre det lettere at dele store programmer op i små blokke af kode.
Det objektorienteret paradigme vil dog kræve en omstrukturering af industriuddannelsen.


\subsection*{IDE}

\subsection*{Instructionlist til PLC++}
