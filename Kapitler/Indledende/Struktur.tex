\label{sec:rapportstruktur}

\mfix{korrektur}

\subsubsection*{Problemanalyse}
Problemanalysen vil gennemgå \gls{plc}'ens anvendelsesområde og hvad den består af. De forskellige programmeringssprog til \gls{plc}'en og ligeledes \gls{plc}'ens mekaniske konstruktion. 

Problemanalysen vil imidlertid have fokus på selve begrebsliggørelsen af \gls{plc}'en - de største producenter og de mest brugte programmeringssprog, hvoraf de problemstillinger, som er blevet udledet fra diverse fora og egne betragtninger, vil blive belyst. Problemanalysen er inddelt i følgende afsnit:

\begin{itemize_small}
    \item \textbf{Hardware} \\
    Kort gennemgang af \gls{plc}'en hardware. Der vil primært være fokus på Omron's CP1H-model, da denne \gls{plc} er i gruppens besiddelse.
    \item \textbf{Software} \\
    Gennemgang af de mest populære programmeringssprog til \gls{plc}. I dette afsnit vil der ligeledes være fokus på Omron.
    \item \textbf{PLC begrænsninger} \\
    Belysning af PLC'ernes mest relevante hardware begrænsninger og hvilke indflydelse disse har på udviklingen af PLC programmer.
\end{itemize_small}

\subsubsection*{Problembeskrivelse}
Problembeskrivelsen vil samle problemanalysens fremførte problemstillinger, hvilket vil lede op til den opstillede problemformulering.

\subsubsection*{Problemløsning}
Problemløsningen vil beskæftige sig med at løse de problemer der er opstillet i problemformuleringen og munde ud i et sprog defineret med en syntaks og tilhørende semantik, efterfulgt af en compiler der kan compilere sproget til en \gls{plc}.

Problemløsningen vil indeholde følgende punkter
\begin{itemize_small}
    \item \textbf{Programmeringsparadigmer}\\
    Afsnittet vil komme ind på 3 af de 4 hovedparadigmer og gennemgå det specielle ved disse.
    \item \textbf{Kravspecifikation}\\
    Kravsspecikationen vil belyse krav til det sprog der skal løse de opstillede problemstillinger i problembeskrivelsen. Afsnittet indlededer med at vælge paradigme til udviklingen af et nyt sprog ved navn PLC++. Herefter vil det lede over i en MoSCoW analyse der har til formål prioritere kravene til den formodede løsning. Kravsspecikationen vil afsluttes med kriterier til sproget og en prioritering af disse.
    \item \textbf{Gramatik}\\ Afsnittet specificerer hvordan man opstiller et sprog vha. en kontekst fri grammatik.
    \item \textbf{Syntaks og semantik}\\
    Afsnittet indeholder en komplet syntax og semantik specifikation over sproget PLC++. Ligeledes vil de vigtigste komponenter i sproget få en kortfattet gennemgang.
    \item \textbf{Compilere}\\
    Compilerafsnittet indeholder teori og implementationen af compileren til PLC++. Den overordnede struktur for afsnittet er bygget op omkring de tre faser i en compiler - Syntaks ananyse, context analyse og kode generering.
\end{itemize_small}

\subsubsection*{Diskussion}
Diskutionen diskuterer eventuelle fordele samt ulemper ved de metoder der er blevet brugt, samt hvad der kunne have været gjort anderledes, til udvikling af PLC++ og den tilhørende compiler. 
\subsubsection*{Konklusion}
Konklusionen konkluderer på om den opstillede problemformulering er løst.

\subsubsection*{Perspektivering}
Perspektiveringen tager udgangspunkt i projektet og belyser eventuelle features, samt potentialer i videre udviklingsforløb.

\subsubsection*{Vigtige bilag}
\begin{itemize_small}
    \item \textbf{Context-free Grammar}\\
    Indeholder gramatikken udtrykt i EBNF
    \item \textbf{Semantik}\\
    Bilaget indeholder alle de semantiske regler for sproget
    \item \textbf{CD}\\
    Indhold af vedlagt CD
\end{itemize_small}

%Det fremførte data, som vil blive gennemgået i problemanalysen, er blevet indsamlet via troværdige websider og faglitteratur. Der er således ikke blevet foretaget nogen dybdegående dataindsamling, da selve problemanalyseringen ikke har været primære fokusområde på dette semester (jf. semesterbeskrivelse). 