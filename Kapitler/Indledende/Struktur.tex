\label{sec:rapportstruktur}

\mfix{Rapportstruktur skal laves færdig}

\subsubsection*{Problemanalyse}
Problemanalysen vil gennemgå de forskellige programmeringssprog til \gls{plc}'en. Ydermere vil \gls{plc}'ens mekaniske konstruktion kortfattet blive gennemgået. 

Problemanalysen vil imidlertid have fokus på selve begrebsliggørelsen af \gls{plc}'en - de største producenter og de mest brugte programmeringssprog vil blive gennemgået, hvoraf de problemstillinger, som er blevet udledet fra diverse fora og egne betragtninger, vil blive belyst. Problemanalysen er inddelt i følgende afsnit:

\begin{itemize_small}
    \item \textbf{Hardware} \\
    Kort gennemgang af \gls{plc}'en hardware. Der vil primært være fokus på Omron's CP1H-model, da denne \gls{plc} er i gruppens besiddelse.
    \item \textbf{Software} \\
    Gennemgang af de mest populære programmeringssprog til \gls{plc}. I dette afsnit vil der ligeledes være fokus på Omron.
    \item \textbf{PLC begrænsninger} \\
    Belysning af PLC'ernes mest relevante hardware begrænsninger og hvilke indflydelse disse har på udviklingen af programmer.
\end{itemize_small}

\subsubsection*{Problembeskrivelse}
Problembeskrivelsen vil samle problemanalysens fremførte problemstillinger, hvilket vil lede op til den opstillede problemformulering.

\subsubsection*{Problemløsning}

\begin{itemize_small}
    \item Programmeringsparadigmer\\
    Afsnittet vil komme ind på 3 af de 4 hovedparadigmer og gennemgå det specielle ved dette som leder videre til valg af paradigme i kravsspecikationen.
    \item Kravspecifikation\\
    Kravsspecikationen vil belyse konkrete krav til implementationen af programmeringssproget PLC++. Herved vil der ledes over på en MoSCoW analyse der vil prioritere kravene til PLC++. Kravsspecikationen vil afsluttes med kriterier til sproget og en prioritering af disse.
    \item Gramatik\\ Afsnittet indeholder forklaring af hvordan sproget beskrives vha. et kontekst frit gramatik.
    \item Syntaks og semantik\\
    Afsnittet indeholder den opstillede syntaks og semantik for PLC++.
    \item Compilere\\
    Afsnittet indeholder konkret teori og implementation af compileren i praksis.
\end{itemize_small}

\subsubsection*{Diskussion}

\subsubsection*{Konklusion}

\subsubsection*{Perspektivering}

\subsubsection*{Vigtige bilag}
\begin{itemize_small}
    \item Context-free Grammar
    \item Semantik
    \item CD
\end{itemize_small}

%Det fremførte data, som vil blive gennemgået i problemanalysen, er blevet indsamlet via troværdige websider og faglitteratur. Der er således ikke blevet foretaget nogen dybdegående dataindsamling, da selve problemanalyseringen ikke har været primære fokusområde på dette semester (jf. semesterbeskrivelse). 