\gls{plc} blev introduceret i slutningen af 1960'erne af firmaet Modicon - et datterselskab til Bedford Associates. Det var grundlæggende bilproducenten, General Motors, der forårsagede den hastige eksponering af PLC’en i industrien. General Motors efterspurgte en mere fleksibel og effektiv løsning til deres automatiserede styringsanlæg – som primært var opbygget af relæer, timere og controllere. \gls{plc}'en skulle dermed erstatte den eksisterende løsning, i håbet om at kunne opnå en forøget effektivisering. \\

\noindent Sidenhen er der kommet flere fabrikanter og modeller på markedet. Omron, Siemens, Beckhoff og Allan Bradly er nogle af de store spillere på markedet. \gls{plc}'ens anvendelsesområde er ligeledes blevet udvidet voldsomt – gående fra automatisk styring af bilproduktionen til fødevareindustrien, slagterier, landbruget og mange flere. 

\noindent \gls{plc}'en egner sig især til disse produktionsområder, fordi dens konstruktion er tolerant overfor store temperatursvingninger, vibrationer, luftfugtighed og har ligeledes en høj levetid. Det adskiller sig i høj grad fra den almindelige computer, som traditionelt ikke er designet til de hårde arbejdsmiljøer. \gls{plc}'en konstrueres med et udvidelses \gls{io} modul, som giver mulighed for at tilslutte analoge og digitale instrumenter, såsom sensorer, temperaturmålere og aktuatorer. \\

\noindent \gls{plc}'en skal programmeres til dens konkrete anvendelsesområde, hvilket ofte programmeres af industriteknikere, som ikke nødvendigvis besidder den store programmeringserfaring. De fleste \gls{plc}'er programmeres i et ladder-logisk paradigme, hvor mange af \gls{plc}-producenterne har udviklet et \gls{ide}, som underbygger drag-and-drop. Paradigmet og \gls{ide}'et gør det relativt nemt at kode \gls{plc}'en til dens arbejdsformål. Det kan imidlertid være en tidskrævende og uoverskueligt proces, hvis det automatiserede anlæg skal håndtere flere digitale indgange og udgange. Programmeringssproget har simpelthen ikke udviklet sig i takt med de mange nye tilkomne programmeringsparadigmer og sprog, såsom det objekt-orienterede paradigme, som giver mulighed for polymorfi, nedarvning og indkapsling. \\

\noindent På baggrund af denne indledning, opstilles den initierende problemstilling: \\

\noindent \textit{Det er for ineffektivitet at konstruere et stort \gls{plc}-program i det logiske programmeringsparadigme, da programmøren hurtigt kan miste overblikket, grundet manglende abstraktionsniveau.}




%Computerens indtog startede for alvor, da elektronikgiganten IBM i året 1953, lancerede den første kommercielle computer, IBM 701. Computerens størrelse, de begrænsede anvendelsesmuligheder og en relativ høj pris, var alle faktorer som medførte et snævert salgstal. Det var dog første spadestik, til dén konsumer-computer, som i dag er allemandseje – og for mange anses som værende fuldstændig uundværlig. \\

%\noindent Kompileres introduktion i 1960’eren, gjorde det væsentlige nemmere at læse og skrive kode, hvilket var en central faktor for den videre udvikling af computere. En ny og interessant forgrening af computeren – de såkaldte mikrocontrollers – blev skabt et par årtier senere. Disse enheder var fuldt funktionelle computere, hvor alle nødvendige komponenter var samlet på ét board. \\

%\noindent Firmaet Parallax udgav i året 1992, en single-board computer, som de navngav ”BASIC Stamp”. Mikrocontrolleren, som det grundlæggende var, blev især populær blandt forskere, hobbister og studerende. Populariteten skyldes at BASIC-sproget var let at komme i gang med, samt havde en mangfoldig og letlæselig dokumentation. I løbet af 10 år var mikrocontrolleren, udviklet af Parallax, blevet distribueret i over 3 millioner eksemplarer. BASIC Stamp havde dog en række ulemper: BASIC kostede omkring 700,- kr. hvilket for mange studerende, syntes at være for dyrt. Ydermere var mange frustreret over den dårlige ydeevne, som BASIC Stamp viste, når brugeren tilkoblede mange tilbehørsmoduler på enheden. Det valgte en gruppe studerende på Interaction Design Institute, Ivrea i Italien, at gøre noget ved; hvilket senere medførte  Arduinos officielle fødsel (året 2005). \\

%\noindent Arduino har sidenhen solgt mere end 250.000 enheder, hvor den almene bruger – tilsvarende BASIC Stamp – forefindes iblandt studerende og hobbister. På trods af de mange brugere, kan det dog formodes, at mange folk uden yderligere programmeringserfaringer, har svært ved at komme i gang med Arduinoen. Dette skyldes primært den mangelfulde dokumentation, dårlige abstraktioner samt misvisende navngivning [www.hackvandedam.nl]\mfix{Kilde}. Dette leder derfor over til den opstillede initierende problemstilling: \\

% \noindent \textit{Det er for besværligt for hobbyister og studerende at komme i gang med at udvikle software til mikrocontrollere,% herunder Arduino.}



%7. april 1953 udkom den første kommercielle computer - dette var IBMs 701-maskine. Dengang havde meget få mennesker hverken plads eller råd til at anskaffe sig en computer. Udover det var det svært at udvikle software til disse tidligere computere, da dette foregik i et meget lavt og maskinnært sprog - eksempelvis maskinkode eller højest assembly. Som årene gik, blev computere både billigere og mindre, men der skete også noget andet, som så småt åbnede dørene for at flere kunne udvikle software til maskinerne: Nemlig at compilere begyndte at udkomme i 1960'erne. Dette gør det muligt at skrive kode, som er væsentligt nemmere for mennesker at læse. Derefter sørger et andet program, compileren, for at lave det om til maskinkode, som computere kan arbejde med.

%Springer vi 20-30 år frem i tiden begynder lancering af de såkaldt single-board computers. Disse er fuldt funktionelle computere, hvor alle nødvendige komponenter ligger på ét board - eksempelvis en blade server. Spoler vi tiden lidt længere frem begynder disse single-board computers at dukke op i meget små størrelser. Her kan der eksempelvis tales om Raspberry Pi eller Arduino. 

%\figur{Figurer/google_trends.png}{Google Trends: \textcolor{red}{Raspberry Pi} vs \textcolor{blue}{Arduino}}{google_trends}{1}
%\mfix{Indsæt evt. ny graf}

%\noindent Som det ses på figur \ref{fig:google_trends} viser Google Trends at interessen for Raspberry Pi og Arduino er steget markant de seneste år. En af grundene til den store popularitet kan skyldes at det er et meget billigt stykke hardware, som kan købes for mellem 200 og 300 kroner og de fleste derved har råd til at købe det som "legetøj"\mbox{}. 

%Selvom de er nemme at komme i besiddelse af, viser erfaring dog at det, for hobbyister, kan være meget svært at komme i gang med at programmere disse. Dette leder derved hen til den initierende problemstilling for projekter, som er:\\

%\noindent \textit{Det er for svært for hobbyister at komme i gang med at udvikle software til single-board computere - herunder Arduino.}