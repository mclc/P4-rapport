\section*{Læsevejledning}
\label{sec:laesevejledning}

\subsubsection*{Kildehenvisning}
Der anvendes Vancouver-metode til kildeanvisninger. Tallet i de hårde parenteser i slutningen af en given påstand henviser til et punkt i litteraturlisten sidst i rapporten. Følgende er et eksempel på en kildehenvisning ud fra en simpel påstand.\\

\noindent \textit{Alle svaner er hvide\emph{[1]}}.

\subsubsection*{Litteraturlisten}
Vancouver-metoden omfatter også en fast opstilling til litteraturlistens poster, hvor informationer er listet: Forfatter(e), Titel på artikel/afsnit, sider med relevant information, bogens titel, redaktør, forlag, udgivelsesårs og ISBN-nummer. De informationer, som der ikke har været mulige at finde, udelades fra litteraturlisten.

\subsubsection*{Figurhenvisning}
Alle figurer i rapporten er tildelt et unikt nummer, som kan blive henvist til flere gange i løbet af rapporten. Det første tal i henvisningen refererer til det kapitel, som figuren befinder i, mens det andet tal angiver, hvilket nummer figuren har i det angivne kapitel. Under en figur er der anbragt en kort beskrivelse. Alle figurer uden kildeangivelse er fremstillet af gruppen. Følgende er et eksempel på en figur efterfulgt af et stykke tekst med en henvisning til samme figur.

\figur{Figurer/Billeder/Figureksempel.jpg}{En hvid svane.}{FigurEksempel}{0.3}

\noindent \textit{Farven på en svane er hvid, hvilket kan ses på figur \ref{fig:FigurEksempel}.}

\subsubsection*{Tabeller}
Alle tabeller overholder næsten samme regler som figurer, dog foregår nummereringen af tabeller og figurer separat. Følgende er et eksempel på en tabel efterfulgt af et stykke tekst med en henvisning til samme tabel.

\begin{table}[ht]
    \centering
    \begin{tabular}{ l | l }
        \textbf{Hvide svaner} & \textbf{Sorte svaner} \\
        \hline \hline
        99 & 1 \\
    \end{tabular}
    \caption{\textit{Eksempel på en tabel}}
    \label{tab:abc}
\end{table}

\noindent \textit{Der er flere hvide svaner end sorte svaner, hvilket kan ses på tabel \ref{tab:abc}}

%\subsubsection*{Kodeeksempler}
%Alle kodeeksempler overholder næsten samme regler som figurer og tabeller, dog foregår nummeringen separat. Følgende er et eksempel på et kodeeksempel efterfulgt af et stykke tekst med en henvisning til samme kodeeksempel. Der kan være fjernet kode der ikke er relevant for eksempel - dette er markeret med kommentarer i kodestykket.

%\CSharp{Kode/eksempel.cs}{Print "Hello World" på skærmen.}{Code_example}

%\noindent \textit{Funktionen i kodeeksempel \ref{code:Code_example} er meget simpel.}

\subsubsection*{Fodnoter}
Fodnoter genkendes som et lille tal, der er opløftet i forhold til teksten. Fodnoten er et nummer, der henviser til et modstykke i bunden af den pågældende side. Fodnoten benyttes til at forklare fagtermer, som ikke er eksplicit forklaret i konteksten. Følgende er et eksempel på en fodnote.\\

\noindent \textit{Svaner er herbivores\fn{Herbivores}{Planteæder} ligesom mange andre fugle.}

\subsubsection*{Metatekst}
Metatekst genkendes som et lille afsnit, der står i kursiv i begyndelsen af nye kapitler. Teksten beskriver hvad kapitlet indeholder. Et eksempel på en metatekst kan se forneden.\\

\noindent \textit{Problembeskrivelsen har fremstillet en enkeltstående og konkret problemformulering, som dette afsnit har til formål at løse. Dette vil afslutningsvist resultere i et produkt, som løser projektet problemformulering.}

%%% FJERNET DA VI IKKE BRUGER DET %%
%\subsubsection*{Fagtermer}
%Fagtermonologi vises ved at ordet er markeret med kursiv. Fagtermer anvendes ved ord, hvor det ikke er en selvfølge, at læseren har et kendskab til ordets betydning. Fagtermet vil blive forklaret i den kontekstuelle sammenhæng, eller ved benyttelse af fodnoten.\\
%\noindent Svanen er en \textit{Planteæder}, ligesom andre fugle.
