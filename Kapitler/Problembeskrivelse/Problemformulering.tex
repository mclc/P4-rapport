\section{Problemformulering}
Det funktionelle programmeringsparadigme egner sig især til mindre anlæg, og dermed mindre programmeringsopgaver. I takt med at anlægget bliver mere omstændeligt, ville det sandsynligvis falde den erfarende programmør naturligt at skifte til \gls{oop}-programmeringsparadigmet, hvorved forøget abstraktionsniveau ville ledsage til forsøget udviklingsomkostnigner, begrænset test-tid, programmets robusthed samt et forøget overblik. 

Det kan dog anses for værende en tidskrævende proces - især for industriteknikere, som ikke besidder yderligere programmeringserfaring, foruden det obligatoriske \gls{plc}-kursus - at tilpasse sig \gls{oop}-programmeringsparadigmet\cite{dislikes_oop}. 

Kan der i stedet konstruereres en programmeringsløsning, som tager det bedste fra begge programmeringsparadigmer, uden at gå på kompromis med bestandige funktionaliteter, ville der muligvis være flere industriteknikere, som ville migrere over til et mere moderne programmeringssprog. Med denne afslutning, er følgende problemformulering blevet opstillet: \\

\noindent\textit{Hvordan kan der konstruereres et hensigtsmæssigt programmeringssprog, som gør det nemmere at bevare overblikket ved større \gls{plc}-programmeringsopgaver.