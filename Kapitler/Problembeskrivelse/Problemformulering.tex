\section{Problemformulering}
Problemanalysen belyste indledningsvist (se afsnit \ref{subsec:pa-hardware}) en række hardwaremæssige fordele og ulemper. En af fordelene var især \gls{plc}'ens mekaniske konstruktion, som viste sig at være designet til hårde arbejdsmiljøer. Konstruktionen gjorde \gls{plc}’en modstandsdygtig overfor rystelser, støj og temperatursvingninger. Imidlertid blev det afdækket, at \gls{plc}’er ikke har kompatibilitet på tværs af producenterne. Udvidelser på \gls{plc}’en, i form af extension-moduler, skulle være af tilsvarende producent. De opsatte hardwaremæssige problemstillinger, er dog ikke noget som kan løses med en hensigtsmæssig softwareløsning, og bliver dermed afgrænset. \\ 

\noindent Kompatibilitetsproblemerne var ikke kun noget som bandt sig til det hardwaremæssige. Også softwaren var påvirket af den manglende konsensus på området. \gls{iec}-standardens intention var netop, at instrukserne skulle generaliseres, og ikke være bundet til den enkelte producent. Omron og Siemens viste dog deres modvilje, og ændrede flere af instrukserne til egne notationer. Dette giver en række problemstillinger med kompatibiliteten mellem producenterne, såfremt at PLC++ oversættes til \gls{il}-instrukser. \gls{il} havde nemlig den fundamentale egenskab, at alle funktionelle ønsker til \gls{plc}’en kan udtrykkes i sproget. Det vil gøre \gls{il} til det ideelle sprog at oversætte til. \\

\noindent \gls{plc}'er er opbygget i det logiske programmeringsparadigme, hvor de grafiske sprog såsom \gls{lad} og \gls{fbd}, gør det nemt for programmøren at opsætte et fungerende program. Det kræver ikke den store programmeringserfaring, udover grundlæggende kendskab til variabler og håndtering af hukommelse. Det viste sig dog også (se afsnit refxx\sfix{Reference}) at mellemstore programmer, bestående af ét transportbånd og én drejearm, kan have en negativ indflydelse på læsbarheden af koden. Læsbarheden blev både forringet i det grafiske (\gls{lad}) såvel som det tekstrepræsenteret sprog (\gls{il}). \\

\noindent Ved store programmer, vil et skift til CODESYS’s \gls{oop}-paradigme give en række fordele, som også blev afdækket i problemanalysen (se afsnit refxx). \gls{oop} kan dog være en stor mundfuld for programmørne, som ikke har fået et omfattende kursus. Langt de fleste anlæg bliver sjældent i et størrelseomfang, hvor \gls{oop}’s kvaliteter ville gavne mere end de ville være en tidsmæssig belastning. Muligvis ville det imperative paradigme være en løsning som ville kunne gavne både små og mellemstore projekter. Med denne afslutning, er følgende problemformulering blevet opstillet: \\

\noindent\textit{Hvordan kan der konstruereres et hensigtsmæssigt programmeringssprog, som gør det nemmere at bevare overblikket ved mellemstore \gls{plc}-programmeringsopgaver, men stadig ikke bliver så omstændigt, at industriteknikere ikke vil skifte?}


%Problemanalysen belyste indledningsvist (se afsnit \ref{subsec:pa-hardware}) en række hardwaremæssige fordele og ulemper. En af fordelene var især \gls{plc}'ens mekaniske konstruktion, som viste sig at være designet til hårde arbejdsmiljøer, hvor rystelser, støj og temperatursvigninger var . Imidlertid blev det belyst, at \gls{plc}'en har kompatibilitetsproblemer, både på det hardware- såvel som det softwaremæssige plan. Programmørene har dermed ikke mulighed for at udvide produktionsanlæggets med udvidelsesmoduler, som ikke er produceret af tilsvarende firma.

%De opsatte hardwaremæssige problemstillinger, er dog ikke noget som kan løses med en hensigtsmæssig softwareløsning, og bliver dermed afgrænset. 

%IEC-1131-3 standarden, som blev udarbejdet for at skabe en generel kode-konsensus for \gls{plc}'er, viste problemanalysen at producenterne, Omron og Siemens har modificeret \gls{il}-instrukserne, hvilket naturligvis ledsager til softwaremæssige kompatibilitetsproblemer. Dette problem vil komme til udtryk, såfremt at PLC++ oversættes til \gls{il}-instrukser. Imidlertid ville problemet kunne løses ved at tilpasse \gls{il}-instrukserne baseret på hvilken model, der skal køre det pågældende program.

%Gennemgangen af \gls{plc}'ens software viste sig ligeledes at have en række fordele og ulemper. \gls{plc}'er er opbygget i det logiske programmeringsparadigme, hvor de grafiske sprog såsom \gls{lad} og \gls{fbd}, gør det nemt for programmøren at opsætte et fungerende program. Det kræver ikke den store programmeringserfaring, udover kendskab til variabler og håndtering af hukommelse, for at få et simpelt program op at køre. Det viste sig dog også (se afsnit refxx) at mellemstore programmer, bestående af ét transportbånd og én drejearm, kan have en negativ indflydelse på læsbarheden af koden. Læsbarheden blev både forringet i det grafiske (\gls{lad}) såvel som det tekstrepræsenteret sprog (\gls{il}). 

%\noindent Det logiske programmeringsparadigme egner sig især til mindre anlæg, og dermed mindre programmeringsopgaver.




 %I takt med at anlægget bliver mere omstændeligt, ville det sandsynligvis falde den erfarende programmør naturligt at skifte til \gls{oop}-programmeringsparadigmet, hvorved forøget abstraktionsniveau ville ledsage til forsøget udviklingsomkostnigner, begrænset test-tid, programmets robusthed samt et forøget overblik. 

%Det kan dog anses for værende en tidskrævende proces - især for industriteknikere, som ikke besidder yderligere programmeringserfaring, foruden det obligatoriske \gls{plc}-kursus - at tilpasse sig \gls{oop}-programmeringsparadigmet\cite{dislikes_oop}. 
%Kan der i stedet konstruereres en programmeringsløsning, som tager det bedste fra begge programmeringsparadigmer, uden at gå på kompromis med bestandige funktionaliteter, ville der muligvis være flere industriteknikere, som ville migrere over til et mere moderne programmeringssprog. 


%\gls{lad} viste sig at være det mest udbredte programmeringssprog, hvilket sandsynligvis skyldes producenternes implementation af Drag \& Drop i deres tilhørende \gls{ide}. Det giver programmøren - uden større programmeringsmæssige erfaring - et let tilgængeligt og indlæringsnemt programmeringsværktøj. Imidlertid viste det sig at der var en flere ulemper forbundet med \gls{lad}-programmeringssproget. 

%Race-conditions er et problem, som ofte forekommer i \gls{plc}-verdenen, og ligeledes kan aritmetiske operationer være besværlig at udtrykke - uden at det går ud over læsbarheden. Ydermere understøttes iterative kontrolstrukturer ikke, hvor \gls{lad} i stedet anvender counter-funktionsblokke. Det ledsager til situationer, hvor programmøren er nødsaget til implementere funktioner, som består af identiske kontrolstrukturer, med den eneste forskel, at counterens preset-værdi varierer. \\
