\subsubsection{Udtryk}
Et udtryk er en sekvens af operatorer og operander. I PLC++ findes der 2 typer udtryk. Aritmetiske og boolske udtryk. Aritmetiske udtryk returnerer et hel- eller kommatal. Boolske udtryk returnerer en \textit{Bool}. Herunder ses et udsnit af semantikken, mens den fulde semantik findes i bilag \ref{bil:semantik}

\begin{Grammar}
 \begin{grammar}

 <e> ::= $n$ | $fp$ | $x$  | $e_1$ "+" $e_2$ | $e_1$ "-" $e_2$ | $e_1$ "*" $e_2$ | $e_1$ "/" $e_2$ | $e_1$ "\%" $e_2$ | "("$e_1$")" | $x$ "=" $e_1$ | $e$ "?" $e_1$ ":" $e_2$ | $e$"++" | $e$"--" | "-"$e$ | $A[e]$ | "true" | "false" | $e_1$ "==" $e_2$ | $e_1$ != $e_2$ | $e_1$ "<" $e_2$ | $e_1$ "<=" $e_2$ | $e_1$ ">" $e_2$ | $e_1$ ">=" $e_2$ | $e_1$ "\&\&" $e_2$ | $e_1$ "||" $e_2$ | "!"$e_1$ | $\epsilon$

 \end{grammar}
 \caption{Abstract syntax for udtryk}\label{gra:expressions}
\end{Grammar}

\noindent Før der gås i detaljen med de forskellige udtryk, er det nødvendigt at omtale brugen af operatorer og operatorprioritering.

\subsubsection{Operatorer}
I tabel \ref{tab:operatorprioritering} ses operatorerne der kan benyttes i udtryk - ved sammensatte udtryk med forskellige operatorer, er operatorprioritering nødvendig. Hvis ikke disse regler fastsættes, kan der fås forskellige resultater, alt efter rækkefølgen, som udtrykket udregnes i. Tabellen er sorteret efter prioritering, hvor de øverste udregnes før de nederste.

\begin{table}[H]
    \centering
    \begin{tabular}{|l|l|}
        \hline
        \centering

        Primær             & x++ \quad x- - \quad a{[}x{]} \quad x.y \quad (x)                 \\ \hline
        Unær               & ++x \quad - -x \quad ! \quad -x \quad +x  \\ \hline
        Multiplikationer   & * \quad / \quad \%                                               \\ \hline
        Additioner         & + \quad -                                                        \\ \hline
        Sammenligning      & \textless \quad \textgreater \quad \textless= \quad\textgreater= \\ \hline
        Ligheder           & == \quad !=                                                      \\ \hline
        Logisk og          & \&\&                                                              \\ \hline
        Logisk eller       & ||                                                               \\ \hline
        Betinget udtryk    & ?:                                                               \\ \hline
        Sammensatte udtryk & *= \quad /= \quad \%= \quad += \quad -=                          \\ \hline
        Assignment         & =                                                                \\ \hline

    \end{tabular}
    \caption{Operatorprioritering (højest til lavest)}
    \label{tab:operatorprioritering}
\end{table}
%\sfix{Venstre/højre assosivitet, samt affix}


\subsubsection{Boolske udtryk}
Boolske udtryk kendetegnes ved at have logiske sammenligningsoperatorer - disse findes i tabel \ref{tab:operatorerbool}. Fælles for dem alle er at de returnerer en \textit{Bool}.

\begin{table}[H]
    \centering
    \begin{tabular}{|l|l|}
        \hline
        \centering

        Sammenligning      & \textless \quad \textgreater \quad \textless= \quad\textgreater= \\ \hline
        Ligheder           & == \quad !=                                                      \\ \hline
        Logisk og          & \&\&                                                              \\ \hline
        Logisk eller       & ||                                                               \\ \hline


    \end{tabular}
    \caption{Boolske operatorer}
    \label{tab:operatorerbool}
\end{table}

\noindent Eksempler på disse boolske udtryk, vil nu blive gennemgået.

\subsubsubsection{AND expression}
\noindent Et "AND expression"\mbox{} er et udtryk, som tager 2 operander og returnerer \textit{true} hvis begge operander er sande - ellers returneres \textit{false}. Herunder ses semantikken for \textit{And expression}\\

\noindent For $[AND_\top]$ skal semantikken læses således: Hvis $e_1$ evaluerer til \textit{true} i $SEnv$ og $e_2$ evaluerer til \textit{true} i $SEnv$, så evaluerer udtrykket $e_1 \&\& e_2$ til true i $SEnv$. 

For $[AND_\bot]$ skal semantikken læses således: Hvis $e_i$ evaluerer til \textit{false} i $SEnv$  for $i$ tilhørende 1 eller 2, så evaluerer udtrykket $e_1  \&\& e_2$ til \textit{false} i $SEnv$. Det vil altså sige at hvis bare én operand er \textit{false}, vil hele udtrykket returnere \textit{false}.

\begin{align*}
&[AND_\top] & &\frac{SEnv \vdash e_1 \rightarrow_e \top \quad SEnv \vdash e_2 \rightarrow_e \top}{SEnv \vdash e_1\; \&\&\; e_2 \rightarrow_e \top}\\\\
&[AND_\bot] & &\frac{SEnv \vdash e_i \rightarrow_e \bot}{SEnv \vdash e_1 \&\& e_2 \rightarrow_e \bot} & &i \in \{1, 2\}\\\\
\end{align*}        


\subsubsubsection{OR expression}
Et \textit{OR expression} tager ligesom \textit{AND expression} 2 operander, men returnerer \textit{true}, hvis bare én af operanderne evaluerer til \textit{true}. 

\noindent Herunder ses semantikken for \textit{OR expression}. Semantikken læses som på samme måde som ved \textit{AND expression}.

\begin{align*}
&[OR_{\top\top}] & &\frac{SEnv \vdash e_1 \rightarrow_e \top \quad SEnv \vdash e_2 \rightarrow_e \top}{SEnv \vdash e_1\; ||\; e_2 \rightarrow_e \top}\\\\
&[OR_{\top\bot}] & &\frac{SEnv \vdash e_1 \rightarrow_e \top \quad SEnv \vdash e_2 \rightarrow_e \bot}{SEnv \vdash e_1\; ||\; e_2 \rightarrow_e \top}\\\\
\end{align*}

    %%%%%%%%%%%%%%%%%%%%%%%%%%%%%%%%%%%%%%%%%%%%%%%%%%%%%%%%%%%%%%%%%%%%%%%%%%%%%%%%%%%%%%%%%%%%%%%%%%%%%%%%%%%%%%%%%%%%%%%%%%%%%%%%%%%%%%%%%%%%%
\subsubsubsection{Equality expression}
Sammenlignings udtryk kommer i seks forskellige varianter. Mindre end, mindre end eller lig med, større end, større end eller lig med, lig med og forskellig fra hinanden. Her vises dog kun "lig med"\mbox{}.

\noindent Herunder ses semantikken for \textit{Equality expression}. 

\noindent For $[EQ_\top]$ læses grammatikken således: hvis $e_1$ evaluerer til $v_1$ i $SEnv$ og $e_2$ evaluerer til $v_2$ i $SEnv$, så evaluerer udtrykket $e_1 == e_2$ til \textit{true} i $SEnv$ hvis $v_1$ er lige med $v_2$.

\begin{align*}
&[EQ_\top] & &\frac{SEnv \vdash e_1 \rightarrow_e v_1 \quad SEnv \vdash e_2 \rightarrow_e v_2}{SEnv \vdash e_1 == e_2 \rightarrow_e \top} & &\text{if } v_1 = v_2\\\\
\end{align*}

%%%%%%%%%%%%%%%%%%%%%%%%%%%%%%%%%%%%%%%%%%%%%%%%%%%%%%%%%%%%%%%%%%%%%%%%%%%%%%%%%%%%%%%%%%%%%%%%%%%%%%%%%%%%%%%%%%%%%%%%%%%%%%%%%%%%%%%%%%%%%
\subsubsection{Aritmetiske udtryk}
Aritmetiske udtryk returnerer et hel- eller kommatal. Disse udtryk kendetegnes ved at benytte operatorerne, som ses i tabel \ref{tab:aritmetiskeOperatorer}. Kun de mest basale aritmetiske udtryk vil blive gennemgået her.

\begin{table}[H]
    \centering
    \begin{tabular}{|l|l|}
        \hline
        \centering

        Primær             & x++ \quad x- -                                    \\ \hline
        Unær               & ++x \quad - -x \quad -x \quad +x                   \\ \hline
        Multiplikationer   & * \quad / \quad \%                                \\ \hline
        Additioner         & + \quad -                                         \\ \hline

    \end{tabular}
    \caption{Aritmetiske Operatorer}
    \label{tab:aritmetiskeOperatorer}
\end{table}

\subsubsubsection{Add and sub expressions}
\textit{Add and sub expressions} returnerer et hel-eller kommatal alt efter operandernes type. Hvis en af operanderne er af typen \textit{float}, returnerer udtrykket en \textit{float}, ellers returnerer udtrykket en \textit{int}. Herunder ses semantikken for "Add and sub expressions"\mbox{}.

\noindent For $[ADD]$ læses semantikken således: hvis $e_1$ evaluerer til $v_1$ i $SEnv$ og $e_2$ evaluerer til $v_2$ i $SEnv$, så evaluerer udtrykket $e_1 + e_2$ til $v$ hvor $v$ er lig med $v_1 + v_2$.
\noindent For $[SUB]$ læses semantikken på samme måde som ved $[PLUS]$.

\begin{align*}
&[ADD] & &\frac{SEnv \vdash e_1 \rightarrow_e\; v_1\quad SEnv \vdash e_2 \rightarrow_e\; v_2}{SEnv \vdash e_1 + e_2 \rightarrow_e\; v} & &\text{where}\; v = v_1 + v_2\\\\
&[SUB] & &\frac{SEnv \vdash e_1 \rightarrow_e\; v_1\quad SEnv \vdash e_2 \rightarrow_e\; v_2}{SEnv \vdash e_1 - e_2 \rightarrow_e\; v} & &\text{where}\; v = v_1 - v_2\\\\
\end{align*}

%%%%%%%%%%%%%%%%%%%%%%%%%%%%%%%%%%%%%%%%%%%%%%%%%%%%%%%%%%%%%%%%%%%%%%%%%%%%%%%%
\subsubsubsection{Mult, div and mod expression}
\textit{Mult, div and mod expression} kommer i tre forskellige varianter. \textit{Mult} returnerer produktet af de to operander, \textit{div} returnerer heltalsdivisionen og \textit{mod} returnerer resten af heltalsdivisionen. Nedenfor ses semantikken for \textit{Mult}, \textit{div} og \textit{mod} expressions.
For $[MULT]$ læses semantikken således: hvis $e_1$ evaluerer til $v_1$ i $SEnv$ og $e_2$ evaluerer til $v_2$ i $SEnv$ så evaluerer udtrykket $e_1 * e_2$ til $v$, hvor $v$ er lig med $v_1 \cdot v_2$.
For $[DIV]$ og $[MOD]$ læses semantikken på samme måde.

\begin{align*}
&[MULT] & &\frac{SEnv \vdash e_1 \rightarrow_e\; v_1\quad SEnv \vdash e_2 \rightarrow_e\; v_2}{SEnv \vdash e_1 * e_2 \rightarrow_e\; v} & &\text{where}\; v = v_1 \cdot v_2\\\\
&[DIV] & &\frac{SEnv \vdash e_1 \rightarrow_e\; v_1\quad SEnv \vdash e_2 \rightarrow_e\; v_2}{SEnv \vdash e_1\; /\; e_2 \rightarrow_e\; v} & &\text{where}\; v = \frac{v_1}{v_2}\\\\
&[MOD] & &\frac{SEnv \vdash e_1 \rightarrow_e\; v_1\quad SEnv \vdash e_2 \rightarrow_e\; v_2}{SEnv \vdash e_1\; \%\; e_2 \rightarrow_e\; v} & &\text{where}\; v = v_1 - (v_2 \cdot \floor*{\frac{v_1}{v_2}})\\\\
\end{align*}


%%%%%%%%%%%%%%%%%%%%%%%%%%%%%%%%%%%%%%%%%%%%%%%%%%%%%%%%%%%%%%%%%%%%%%%%%%%%%%%%
\subsubsubsection{Increment and decrement expressions}
\textit{Increment and decrement expressions} findes i to forskellige varianter. Som en prefix-operator og en postfix-operator. Prefix versionen tæller operanden 1 op eller ned og returnerer den inc- eller decrementerede værdi. Postfix-versionen returnerer operandens værdi, hvorefter den inc- eller decrementerer operanden. Nedenfor ses grammatikken for begge typer increment og decrement.

For $[INC_{POST}]$ læses semantikken således: Hvis udtrykket $e$ evaluerer til $v$ i $SEnv$, så evaluerer udtrykket $e++$ i $sto$ til $v$ og $sto$, hvor $e$ mapper til $v^\prime$ - hvor $v^\prime$ er lig med $v+1$ 

\begin{align*}
&[INC_{POST}] & &\frac{SEnv \vdash e \rightarrow v}{Env \vdash \langle e\text{++}, sto \rangle \rightarrow  v, sto[e \mapsto v^\prime] } & &\text{where }v^\prime = v + 1\\\\
\end{align*}