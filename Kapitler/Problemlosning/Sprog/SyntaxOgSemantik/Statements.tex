\noindent \subsubsection{Statements}
Sproget PLC++ består af en række forskellige statements. Statements er eksempelvis funktionskald, iterative og selektive kontrolstrukturer. Under dette afsnit vil et udvalg af statements blive gennemgået og forklaret med tilhørende semantik. Nedenfor ses den abstrakte syntax for statements i PLC++, som anvendes i semantikken på grund af dens simplificering.

\begin{Grammar}
 \begin{grammar}
 <St> ::= "{" $Sts$ "}" | "if" "("$e_1$")" $St$ | "if" "("$e_1$")" $St_1$ "else" $St_2$ | "while" $e$ $St$ | $F$"("$e$")" | "for" "("$e$";" $e$";" $e$")" $St$ | $x$ "+=" $e$ | $x$ "-=" $e$ | $X$ "*=" $e$ | $x$ "/=" $e$ | $x$ "\%=" $e$ | $\epsilon$
 \end{grammar}
 \caption{Abstrakt syntaks for statement}\label{gra:Statements}
\end{Grammar}


\subsubsubsection{While-løkke}
For at kunne iterere i sproget, er der blevet valgt både while- og for-løkke som en del af sproget. Nedenfor ses semantikken for while-løkker. \\

\noindent For $[WHILE_\top]$ læses semantikken således: 
Hvis man i $Env$ eksekverer $St$ i $sto$ og får state $sto''$, hvorefter man i $Env$ ekseverer $while(e) St$ i $sto''$ og får state $sto'$. Så får vi ved eksekvering af $while(e) St$ i $sto$ state $sto'$ hvis $e$ evaluerer til \textit{true} i $Env$. Hvis $e$ derimod evaluerer til \textit{false}, sker der intet med $sto$.

\begin{align*}
&[WHILE_\top] & &Env \vdash \langle St, sto \rangle \rightarrow sto^{\prime\prime}\\
& & &\frac{Env \vdash \langle \text{while } (e)\; St,\; sto^{\prime\prime} \rangle \rightarrow sto^\prime}{Env \vdash \langle \text{while } (e)\; St,\; sto \rangle \rightarrow sto^\prime} & &\text{if } Env \vdash e \rightarrow_e \top\\\\
%
&[WHILE_\bot] & &Env \vdash \langle \text{while } (e)\; St,\; sto \rangle \rightarrow sto & &\text{if } Env \vdash e \rightarrow_e \bot\\\\
\end{align*}

\noindent I sproget findes også for-løkker - disse er ikke beskrevet i semantikken, da de på simpel vis bliver omskrevet til while-løkker inden code generation. I kodeeksempel \ref{code:fortowhile} ses hvordan en while-løkke, som er omskrevet fra en for-løkke, ser ud i konkret kode. Det tilsvarende \gls{ast} ses i figur \ref{fig:fortowhile}

\PPP{Kode/ForToWhile.ppp}{Omskrivning af for-løkker til while-løkker}{fortowhile}

\tikzfigure{Figurer/TikZ/ForToWhile.tex}{AST for For til While}{fortowhile}{1.0}

\noindent \textit{ASTSimplify}-klassen, som sørger for alle omskrivninger, indeholder desuden et tjek, hvor det tjekkes om der er en \textit{condition} i den pågældende for-løkke. Hvis ikke der er det, er det kun nødvendigt med en \textit{TrueExpr} - et eksempel på dette ses i kodeeksempel \ref{code:fortowhilesimple} og figur \ref{fig:fortowhilesimple}

\PPP{Kode/ForToWhileSimple.ppp}{Omskrivning af simple for-løkker til while-løkker}{fortowhilesimple}

\tikzfigure{Figurer/TikZ/ForToWhileSimple.tex}{Det resulterende AST for en omskrivning af simpel for-løkke}{fortowhilesimple}{1.0}

\subsubsubsection{If-statements}
\textit{If}-statements er en af de selektive kontrolstrukturer. \textit{If}-statement'et tager et boolsk udtryk, som afgører om dens krop skal eksekveres. Hvis det boolske udtryk evaluerer til \textit{false}, springes kroppen over. Dette \textit{if}-statement fungerer, som man kender det i C. Nedenfor ses semantikken for \textit{if}-statements.

\noindent For $[IF_\top]$ læses semantikken således: 

Hvis man i $Env$ eksekverer $St$ i $sto$ og får state $sto'$, så fås $sto'$ ved eksekvering af udtrykkel $if(e)\; St$ i $sto$, hvis $e$ evaluerer til \textit{true} i $Env$.

\begin{align*}
&[IF_\top] & &\frac{Env \vdash \langle St, sto \rangle \rightarrow sto^\prime}{Env \vdash \langle \text{if } (e)\; St,\; sto \rangle \rightarrow sto^\prime} & &\text{if } Env \vdash e \rightarrow_e \top\\\\
\end{align*}

%%%%%%%%%%%%%%%%%%%%%%%%%%%%%%%%%%%%%%%%%%%%%%%%%%%%%
\subsubsubsection{Structs}
For at kunne definere mere avancerede typer, med tilhørende adfærd, er \textit{structs} en del af PLC++. \textit{Structs} kan indholde felter af andre typer, som man kender dem fra C. Som en udbyggelse af C's \textit{structs}, er funktioner i \textit{struct} blevet tilføjet, da det ønskes at kunne definere adfærd.

Felter i \textit{structs} kan tilgås på to forskellige måder - alt efter om det er en variabel eller et funktionskald. Nedenfor ses semantikken for at tilgå både et felt, som er en variabel, og et funktionskald.

For $[SVAR]$ læses semantikken således:
Hvis $X$ evaluerer til $l$ i $env'_s$ og $env'_v$ så evaluerer $s.X$ til $l$ i $env_s$, $env_v$ og $sto$, hvor $env_s s$ er lig med $env'_S$, $env'_F$ og $env'_S$. 


For $[SFUNC]$ læses semantikken således: \textit{s.SF}
Hvis $SF$ evaluerer til $St$, $env''_v$, $env''_F$ og $env''_S$ i $env'_s$ og $env'_F$, så evaluerer $s.SF$ til $St$, $env''_v$, $env''_F$ og $env''_S$ i $env_s$ og $env_F$, hvor $env_s s$ er $env'_S$, $env'_F$ og $env'_S$. 


\begin{align*}
&[SVAR] & &\frac{env_S^\prime, env_V^\prime \vdash X \rightarrow l}{env_S, env_V, sto \vdash s.X \rightarrow l} & &\text{where } env_S\; s = (env_V^\prime, env_F^\prime, env_S^\prime)\\\\
&[SFUNC] & &\frac{env_S^\prime, env_F^\prime \vdash SF \rightarrow (St, env_V^{\prime\prime}, env_F^{\prime\prime}, env_S^{\prime\prime})}{env_S, env_F \vdash s.SF \rightarrow (St, env_V^{\prime\prime}, env_F^{\prime\prime}, env_S^{\prime\prime})} & &\text{where } env_S\; s = (env_V^\prime, env_F^\prime, env_S^\prime)
\end{align*}