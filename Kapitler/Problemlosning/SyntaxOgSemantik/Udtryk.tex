\subsection{Udtryk}
Et udtryk er en sekvens af operatorer og operander. I PLC++ findes der 2 typer udtryk. Aritmetiske og boolske udtryk. Aritmetiske udtryk returnerer et hel- eller kommatal. Boolske udtryk returnerer en bool. Før der gåes i detaljen med de forskellige udtryk bliver vi nødt til at snakke om hvilke operatorer der kan benyttes.

\subsubsection{Operatorer}
I tabellen herunder ses operatorerne der kan benyttes i udtryk. Ved sammensatte udtryk med forskellige operatorer er operator prioritering nødvendig. Hvis ikke disse regler fastsættes kan der fåes forskellige resultater, alt efter hvilken rækkefølge udtrykket udregnes. Tabellen er ordnet efter prioritering hvor de øverste udregnes før de nederste.
\begin{table}[H]
    \centering
    \begin{tabular}{|l|l|}
        \hline
        \centering

        Primær             & x++ \quad x- - \quad a{[}x{]} \quad x.y \quad (x)                 \\ \hline
        Unær               & ++x \quad - -x \quad ! \quad -x \quad +x \quad (cast)x       \\ \hline
        Multiplikationer   & * \quad / \quad \%                                               \\ \hline
        Additioner         & + \quad -                                                        \\ \hline
        Sammenligning      & \textless \quad \textgreater \quad \textless= \quad\textgreater= \\ \hline
        Ligheder           & == \quad !=                                                      \\ \hline
        Logisk og          & \&\&                                                              \\ \hline
        Logisk eller       & ||                                                               \\ \hline
        Betinget udtryk    & ?:                                                               \\ \hline
        Sammensatte udtryk & *= \quad /= \quad \%= \quad += \quad -=                          \\ \hline
        Assignment         & =                                                                \\ \hline

    \end{tabular}
    \caption{\textit{Operatorprioritering (højest til lavest)}}
    \label{tab:operatorprioritering}
\end{table}
\subsubsection{Boolske udtryk}
Boolske udtryk kendetegnes ved at have logisk og, logisk eller, Sammenlignings eller ligheds operatorerne. operatorerne ses herunder.

\begin{table}[H]
    \centering
    \begin{tabular}{|l|l|}
        \hline
        \centering

        Sammenligning      & \textless \quad \textgreater \quad \textless= \quad\textgreater= \\ \hline
        Ligheder           & == \quad !=                                                      \\ \hline
        Logisk og          & \&\&                                                              \\ \hline
        Logisk eller       & ||                                                               \\ \hline


    \end{tabular}
    \caption{\textit{Operatorer i boolske udtryk}}
    \label{tab:operatorprioritering}
\end{table}

\noindent\textbf{And expression}

\noindent Et and expression er et udtryk der tager 2 operander og returnerer true hvis begge operander er sande. hvis ikke returnerer udtrykket false.
herunder kan gramatikken ses i BNF.
\mfix{Centrer dette BNF exsempel}
\input{Kode/CFG/BNF_and_expr.scc}


    \bgroup
    \def\arraystretch{3}
    \begin{table}[H]
    \centering
    \begin{tabular}{l c l}
        
        $[AND-1_{BS}]$ &$\frac{b_1 \rightarrow \; tt \quad b_2 \rightarrow \; tt}{b_1 \bigwedge b_2 \rightarrow \; tt}$ & \\
    
        $[AND-2_{BS}]$ &$\frac{b_i \rightarrow \; ff}{b_1 \bigwedge b_2 \rightarrow \; ff}$ & $i \in \{1, 2\}$\\
        
    \end{tabular}
    \caption{AND expression}
    \label{tab:andexpr}
    \end{table}
    \egroup

\noindent\textbf{Or expression}

\noindent Et or expression tager ligesom and expression 2 operander men returnerer sand hvis bare et af operanderne er sande. Gramatikken for and expression kan ses herunder.

\input{Kode/CFG/BNF_or_expr.scc}

    \bgroup
    \def\arraystretch{3}
    \begin{table}[H]
    \centering
    \begin{tabular}{l c l}
        
        $[OR-1_{BS}]$ &$\frac{b_1 \rightarrow \; tt \quad b_2 \rightarrow \; ff}{b_1 \bigvee b_2 \rightarrow \; tt}$ & \\
    
        $[OR-2_{BS}]$ &$\frac{b_i \rightarrow \; ff}{b_1 \bigvee b_2 \rightarrow \; ff}$ & $i \in \{1, 2\}$\\
        
    \end{tabular}
    \caption{OR expression}
    \label{tab:orexpr}
    \end{table}
    \egroup
    
\textbf{Equality expressions}

\noindent\noindent Whatever
    
    \input{Kode/CFG/BNF_equality_expr.scc}


    \bgroup
    \def\arraystretch{3}
    \begin{table}[H]
    \centering
    \begin{tabular}{l c l}
        
        $[EQUALS-T_{BS}]$ &$\frac{expr_1 \rightarrow v_1 \quad expr_2 \rightarrow v_2}{expr_1\;=\;expr_2 \rightarrow \; tt}$ & if $v_1 = v_2$ \\
        
        $[EQUALS-\bot_{BS}]$ &$\frac{expr_1 \rightarrow v_1 \quad expr_2 \rightarrow v_2}{expr_1\;=\;expr_2 \rightarrow \; ff}$ & if $v_1 \ne v_2$ \\
        
    \end{tabular}
    \caption{Equality expressions}
    \label{tab:equaexpr}
    \end{table}
    \egroup

\noindent \subsubsection{Aritmetiske udtryk}

\noindent \textbf{Add and sub expressions}

    \bgroup
    \def\arraystretch{3}
    \begin{table}[H]
    \centering
    \begin{tabular}{l c l}
        
        $[PLUS_{BS}]$ &$\frac{expr_1 \rightarrow v_1 \quad expr_2 \rightarrow v_2}{expr_1\;+\;expr_2 \rightarrow v}$ & where $v = v_1 + v_2$ \\
        
        $[MINUS_{BS}]$ &$\frac{expr_1 \rightarrow v_1 \quad expr_2 \rightarrow v_2}{expr_1\;-\;expr_2 \rightarrow v}$ & where $v = v_1 - v_2$ \\
        
    \end{tabular}
    \caption{Add and sub expression}
    \label{tab:addandsub}
    \end{table}
    \egroup

    \begin{Grammar}
 \begin{grammar}

    <add\_sub\_expr> ::= <add\_sub\_expr> "+" <mult\_div\_mod\_expr>
    \alt <add\_sub\_expr> "-" <mult\_div\_mod\_expr>
  
 \end{grammar}
 \caption{Add and sub expressions}\label{gra:addsub}
\end{Grammar}
    
  