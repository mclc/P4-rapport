\subsection{Udtryk}
Et udtryk er en sekvens af operatorer og operander. I PLC++ findes der 2 typer udtryk. Aritmetiske og boolske udtryk. Aritmetiske udtryk returnerer et hel- eller kommatal. Boolske udtryk returnerer en \textit{Bool}. Herunder ses gramatikken for udtryk.

\begin{Grammar}
 \begin{grammar}

 <e> ::= $n$ | $fp$ | $x$  | $e_1$ "+" $e_2$ | $e_1$ "-" $e_2$ | $e_1$ "*" $e_2$ | $e_1$ "/" $e_2$ | $e_1$ "\%" $e_2$ | "("$e_1$")" | $x$ "=" $e_1$ | $e$ "?" $e_1$ ":" $e_2$ | $e$"++" | $e$"--" | "-"$e$ | $A[e]$ | "true" | "false" | $e_1$ "==" $e_2$ | $e_1$ != $e_2$ | $e_1$ "<" $e_2$ | $e_1$ "<=" $e_2$ | $e_1$ ">" $e_2$ | $e_1$ ">=" $e_2$ | $e_1$ "\&\&" $e_2$ | $e_1$ "||" $e_2$ | "!"$e_1$ | $\epsilon$

 \end{grammar}
 \caption{Udtryk}\label{gra:expressions}
\end{Grammar}

\noindent Før der gås i detaljen med de forskellige udtryk er det nødvendigt at omtale brugen af operatorer og operatorprioritering.

\subsubsection{Operatorer}
I tabel \ref{tab:operatorprioritering} ses operatorerne der kan benyttes i udtryk. Ved sammensatte udtryk med forskellige operatorer er operator prioritering nødvendig. Hvis ikke disse regler fastsættes kan der fåes forskellige resultater, alt efter rækkefølgen udtrykket udregnes. Tabellen er ordnet efter prioritering hvor de øverste udregnes før de nederste.

\begin{table}[H]
    \centering
    \begin{tabular}{|l|l|}
        \hline
        \centering

        Primær             & x++ \quad x- - \quad a{[}x{]} \quad x.y \quad (x)                 \\ \hline
        Unær               & ++x \quad - -x \quad ! \quad -x \quad +x \quad (cast)x       \\ \hline
        Multiplikationer   & * \quad / \quad \%                                               \\ \hline
        Additioner         & + \quad -                                                        \\ \hline
        Sammenligning      & \textless \quad \textgreater \quad \textless= \quad\textgreater= \\ \hline
        Ligheder           & == \quad !=                                                      \\ \hline
        Logisk og          & \&\&                                                              \\ \hline
        Logisk eller       & ||                                                               \\ \hline
        Betinget udtryk    & ?:                                                               \\ \hline
        Sammensatte udtryk & *= \quad /= \quad \%= \quad += \quad -=                          \\ \hline
        Assignment         & =                                                                \\ \hline

    \end{tabular}
    \caption{Operatorprioritering (højest til lavest)}
    \label{tab:operatorprioritering}
\end{table}
\mfix{Er primær det rigtige ord?}\sfix{Venstre/højre assosivitet, samt affix}


\subsubsection{Boolske udtryk}
Boolske udtryk kendetegnes ved at have logisk og, logisk eller, Sammenlignings eller ligheds operatorerne. operatorerne ses i tabel \ref{tab:operatorerbool}. Fælles for dem alle er at de returnerer en \textit{Bool}.

\begin{table}[H]
    \centering
    \begin{tabular}{|l|l|}
        \hline
        \centering

        Sammenligning      & \textless \quad \textgreater \quad \textless= \quad\textgreater= \\ \hline
        Ligheder           & == \quad !=                                                      \\ \hline
        Logisk og          & \&\&                                                              \\ \hline
        Logisk eller       & ||                                                               \\ \hline


    \end{tabular}
    \caption{Boolske operatorer}
    \label{tab:operatorerbool}
\end{table}
%%%%%%%%%%%%%%%%%%%%%%%%%%%%%%%%%%%%%%%%%%%%%%%%%%%%%%%%%%%%%%%%%%%%%%%%%%%%%%%%%%%%%%%%%%%%%%%%%%%%%%%%%%%%%%%%%%%%%%%%%%%%%%%%%%%%%%%%%%%%%
\subsubsubsection{AND expression}

\noindent Et "AND expression" er et udtryk der tager 2 operander og returnerer true hvis begge operander er sande. hvis ikke returnerer udtrykket false.

\noindent Herunder ses semantikken for \textit{And expression}. 

\noindent For $[AND_\top]$ skal semantikken læses således: Hvis $e_1$ evaluerer til \textit{true} i $SEnv$ og $e_2$ evaluerer til \textit{true} i $SEnv$ så evaluerer udtrykket $e_1 || e_2$ til true i $SEnv$. 

\noindent For $[AND_\bot]$ skal semantikken læses således: Hvis $e_i$ evaluerer til $false$ i $SEnv$  for $i$ tilhørende 1 eller 2, så evaluerer udtrykket $e_1  \&\&  e_2$ til false i $SEnv$.

\begin{align*}
&[AND_\top] & &\frac{SEnv \vdash e_1 \rightarrow_e \top \quad SEnv \vdash e_2 \rightarrow_e \top}{SEnv \vdash e_1\; \&\&\; e_2 \rightarrow_e \top}\\\\
&[AND_\bot] & &\frac{SEnv \vdash e_i \rightarrow_e \bot}{SEnv \vdash e_1 \&\& e_2 \rightarrow_e \bot} & &i \in \{1, 2\}\\\\
\end{align*}        


%%%%%%%%%%%%%%%%%%%%%%%%%%%%%%%%%%%%%%%%%%%%%%%%%%%%%%%%%%%%%%%%%%%%%%%%%%%%%%%%%%%%%%%%%%%%%%%%%%%%%%%%%%%%%%%%%%%%%%%%%%%%%%%%%%%%%%%%%%%%%
\subsubsubsection{OR expression}
Et \textit{OR expression} tager ligesom \textit{AND expression} 2 operander men returnerer \textit{true} hvis bare en af operanderne evaluerer \textit{true}. 

\noindent Herunder ses semantikken for \textit{OR expression}. Semantikken læses som på samme måde som ved \textit{AND expression}.

\begin{align*}
&[OR_{\top\top}] & &\frac{SEnv \vdash e_1 \rightarrow_e \top \quad SEnv \vdash e_2 \rightarrow_e \top}{SEnv \vdash e_1\; ||\; e_2 \rightarrow_e \top}\\\\
&[OR_{\top\bot}] & &\frac{SEnv \vdash e_1 \rightarrow_e \top \quad SEnv \vdash e_2 \rightarrow_e \bot}{SEnv \vdash e_1\; ||\; e_2 \rightarrow_e \top}\\\\
&[OR_\bot] & &\frac{SEnv \vdash e_i \rightarrow_e \bot}{SEnv \vdash e_1 \;||\; e_2 \rightarrow_e \bot} & &i \in \{1, 2\}\\\\
\end{align*}

    %%%%%%%%%%%%%%%%%%%%%%%%%%%%%%%%%%%%%%%%%%%%%%%%%%%%%%%%%%%%%%%%%%%%%%%%%%%%%%%%%%%%%%%%%%%%%%%%%%%%%%%%%%%%%%%%%%%%%%%%%%%%%%%%%%%%%%%%%%%%%
\subsubsubsection{Equality expression}
Sammenlignings udtryk kommer i seks forskellige varianter. Mindre end, mindre end eller lig med, større end, større end eller lig med, lig med og forskellig fra hinanden. Her vises dog kun "lig med". De resterende kan ses i bilag.

\noindent Herunder ses semantikken for \textit{Equality expression}. 

\noindent For $[EQ_\top]$ læses grammatikken således: hvis $e_1$ evaluerer til $v_1$ i $SEnv$ og $e_2$ evaluerer til $v_2$ i $SEnv$, så evaluerer udtrykket $e_1 == e_2$ til $true$ i $SEnv$ hvis $v_1$ er lige med $v_2$.

\begin{align*}
&[EQ_\top] & &\frac{SEnv \vdash e_1 \rightarrow_e v_1 \quad SEnv \vdash e_2 \rightarrow_e v_2}{SEnv \vdash e_1 == e_2 \rightarrow_e \top} & &\text{if } v_1 = v_2\\\\
\end{align*}

%%%%%%%%%%%%%%%%%%%%%%%%%%%%%%%%%%%%%%%%%%%%%%%%%%%%%%%%%%%%%%%%%%%%%%%%%%%%%%%%%%%%%%%%%%%%%%%%%%%%%%%%%%%%%%%%%%%%%%%%%%%%%%%%%%%%%%%%%%%%%
\subsubsection{Aritmetiske udtryk}
Aritmetiske udtryk returnerer et hel eller kommatal. Disse udtryk kendetegnes ved at benytte operatorerne som ses i tabel \ref{tab:aritmetiskeOperatorer}. Kun de mest basale aritmetiske udtryk vil blive gennemgået her. Resten kan findes i afsnit \ref{ssec:aexpr}


\begin{table}[H]
    \centering
    \begin{tabular}{|l|l|}
        \hline
        \centering

        Primær             & x++ \quad x- -                                    \\ \hline
        Unær               & ++x \quad - -x \quad -x \quad +x                   \\ \hline
        Multiplikationer   & * \quad / \quad \%                                \\ \hline
        Additioner         & + \quad -                                         \\ \hline

    \end{tabular}
    \caption{Aritmetiske Operatorer}
    \label{tab:aritmetiskeOperatorer}
\end{table}

\subsubsubsection{Add and sub expressions}
\textit{Add and sub expressions} returnerer additionen og substraktionen af de to operander og returnerer et heltal eller kommatal alt efter operandernes type. Hvis en af operanderne er af typen $float$ returnerer udtrykket en $float$, ellers returnerer udtrykket en $Int$.

\noindent Herunder ses semantikken for "Add and sub expressions"\mbox{}.

\noindent For $[ADD]$ læses semantikken således: hvis $e_1$ evaluerer til $v_1$ i $SEnv$ og $e_2$ evaluerer til $v_2$ i $SEnv$ så evaluerer udtrykket $e_1 + e_2$ til $v$ hvor $v$ er lig med $v_1 + v_2$.
\noindent For $[SUB]$ læses semantikken på samme måde som ved $[PLUS]$.

\begin{align*}
&[ADD] & &\frac{SEnv \vdash e_1 \rightarrow_e\; v_1\quad SEnv \vdash e_2 \rightarrow_e\; v_2}{SEnv \vdash e_1 + e_2 \rightarrow_e\; v} & &\text{where}\; v = v_1 + v_2\\\\
&[SUB] & &\frac{SEnv \vdash e_1 \rightarrow_e\; v_1\quad SEnv \vdash e_2 \rightarrow_e\; v_2}{SEnv \vdash e_1 - e_2 \rightarrow_e\; v} & &\text{where}\; v = v_1 - v_2\\\\
\end{align*}

%%%%%%%%%%%%%%%%%%%%%%%%%%%%%%%%%%%%%%%%%%%%%%%%%%%%%%%%%%%%%%%%%%%%%%%%%%%%%%%%
\subsubsubsection{Mult, div and mod expression}
\textit{Mult, div and mod expression} kommer i tre forskellige varianter. Mult returnerer produktet af de to operander. Div returnerer heltalsdivisionen og mod returnerer resten af heltalsdivisionen.

\noindent Nedenfor ses semantikken for Mod, div og mod expressions.
For $[MULT]$ læses semantikken således: hvis $e_1$ evaluerer til $v_1$ i $SEnv$ og $e_2$ evaluerer til $v_2$ i $SEnv$ så evaluerer udtrykket $e_1 * e_2$ til $v$ hvor $v$ er lig med $v_1 * v_2$.
For $[DIV]$ og $[MOD]$ læses semantikken på samme måde.

\begin{align*}
&[MULT] & &\frac{SEnv \vdash e_1 \rightarrow_e\; v_1\quad SEnv \vdash e_2 \rightarrow_e\; v_2}{SEnv \vdash e_1 * e_2 \rightarrow_e\; v} & &\text{where}\; v = v_1 \cdot v_2\\\\
&[DIV] & &\frac{SEnv \vdash e_1 \rightarrow_e\; v_1\quad SEnv \vdash e_2 \rightarrow_e\; v_2}{SEnv \vdash e_1\; /\; e_2 \rightarrow_e\; v} & &\text{where}\; v = \frac{v_1}{v_2}\\\\
&[MOD] & &\frac{SEnv \vdash e_1 \rightarrow_e\; v_1\quad SEnv \vdash e_2 \rightarrow_e\; v_2}{SEnv \vdash e_1\; \%\; e_2 \rightarrow_e\; v} & &\text{where}\; v = v_1 - (v_2 \cdot \floor*{\frac{v_1}{v_2}})\\\\
\end{align*}


%%%%%%%%%%%%%%%%%%%%%%%%%%%%%%%%%%%%%%%%%%%%%%%%%%%%%%%%%%%%%%%%%%%%%%%%%%%%%%%%
\subsubsubsection{Increment and decrement expressions}
\textit{Increment and decrement expressions} findes i to forskellige varianter. Som en prefix operator og en sufix operator. Prefix versionen tæller operanden en op eller ned og returnerer den inc- eller decrementerede værdi. Postfix versionen returnerer operandens værdi hvorefter den inc- eller decrementerer operanden. Nedenfor ses grammatikken for begge typer increment og decrement.

\noindent For $[INC_{POST}]$ læses semantikken således: Hvis udtrykket $e$ evaluerer til $v$ i $SEnv$, så evaluerer udtrykket $e++$ i $sto$ til $v$ og $sto$ hvor $e$ evaluerer til $v'$ i $Env$, Hvor $v'$ er lig med $v+1$ 

\begin{align*}
&[INC_{POST}] & &\frac{SEnv \vdash e \rightarrow v}{Env \vdash \langle e\text{++}, sto \rangle \rightarrow  v, sto[e \mapsto v^\prime] } & &\text{where }v^\prime = v + 1\\\\
\end{align*}