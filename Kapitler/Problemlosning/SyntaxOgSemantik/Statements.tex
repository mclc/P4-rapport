\noindent \subsection{Statements}
Sproget PLC++ består af en række forskellige statements. Statements som funktionskald, iterative og selektive kontrolstrukturer. Under dette afsnit vil de mest betydende statements blive gennemgået og forklaret med tilhørende semantik. Nedenfor ses grammatikken for Statements i PLC++.

\begin{align*}
&[BRACES] & & \frac{E\vdash Sts : \text{ok}}{E\vdash\{\;Sts\;\} : \text{ok}}\\\\
&[COMP] & &\frac{E \vdash St : \text{ok}\quad E \vdash Sts : \text{ok}}{E \vdash St;\;Sts : \text{ok}}\\\\
&[IF] & &\frac{E \vdash e : \text{Bool}\quad E \vdash St : \text{ok}}{E \vdash \text{if}\; (e)\;  St : \text{ok}}\\\\
&[IF-ELSE] & &\frac{E \vdash e : \text{Bool}\quad E \vdash St_1 : \text{ok}\quad E \vdash St_2 : \text{ok}}{E \vdash \text{if}\; (e)\; St_1\; \text{else} \;St_2 : \text{ok}}\\\\
&[WHILE] & &\frac{E \vdash e : \text{Bool}\quad E \vdash S : \text{ok}}{E \vdash while\; (e)\; St : \text{ok}}\\\\
&[FOR] & &\frac{E \vdash e_1 : \text{ok}\enskip E \vdash e_2 : \text{bool}\enskip E \vdash e_3 : \text{ok}\enskip E \vdash : \text{ok}}{E \vdash \text{for } (\; e_1;\; e_2;\; e_3\;) \; St : \text{ok}}\\\\
&[CALL] & &\frac{E \vdash F : (x : T \rightarrow \text{ok})\quad E \vdash e : \text{T}}{E \vdash F(e) : \text{ok}}\\\\
&[CPS] & &\frac{E \vdash x : T\quad E \vdash e : T}{E \vdash x\; op\; e : \text{ok}}\\
& & &\text{where}\; op \in \{+=, -=, *=, *=, \%=\}
\end{align*}


\subsubsection*{While-løkke}
For at kunne iterere i sproget er der blevet valgt både while- og for-løkke som en del af sproget. Da while-løkker og for-løkker er semantiske ækvivalente, og for-løkker omskrives til while-løkker inden code generation, er det ikke relevant at opskrive en semantik for for-løkker. 

Nedenfor ses semantikken for while-løkker.

\begin{align*}
&[WHILE_\top] & &Env \vdash \langle St, sto \rangle \rightarrow sto^{\prime\prime}\\
& & &\frac{Env \vdash \langle \text{while } (e)\; St,\; sto^{\prime\prime} \rangle \rightarrow sto^\prime}{Env \vdash \langle \text{while } (e)\; St,\; sto \rangle \rightarrow sto^\prime} & &\text{if } Env \vdash e \rightarrow_e \top\\\\
%
&[WHILE_\bot] & &Env \vdash \langle \text{while } (e)\; St,\; sto \rangle \rightarrow sto & &\text{if } Env \vdash e \rightarrow_e \bot\\\\
\end{align*}

\subsubsection*{If-else statements}
If-else er en af de selektive kontrolstrukturer der er valgt til at kunne selektere eksekvering af kode. If-else statementet tager et boolsk udtryk der afgør om dens krop skal eksekveres. Hvis det boolske udtryk evaluerer til false bliver else kroppen eksekveret. Dette if-else statement fungerer nøjagtig som man kender det i C.
Nedenfor ses semantikken for if-else statementet.

\begin{align*}
&[IF-E_\top] & &\frac{Env \vdash \langle St_1, sto \rangle \rightarrow sto^\prime}{Env \vdash \langle \text{if } (e)\;St_1 \text{ else } St_2,\; sto \rangle \rightarrow sto^\prime} & &\text{if } Env \vdash e \rightarrow_e \top\\\\
%
&[IF-E_\bot] & &\frac{Env \vdash \langle St_2, sto \rangle \rightarrow sto^\prime}{Env \vdash \langle \text{if } (e)\; St_1 \text{ else } St_2,\; sto \rangle \rightarrow sto^\prime} & &\text{if } Env \vdash e \rightarrow_e \bot\\\\
\end{align*}


\subsubsection*{Structs}

