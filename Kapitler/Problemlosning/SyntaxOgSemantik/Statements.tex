\noindent \subsubsection{Statements}
Sproget PLC++ består af en række forskellige statements. Statements som funktionskald, iterative og selektive kontrolstrukturer. Under dette afsnit vil de mest betydende statements blive gennemgået og forklaret med tilhørende semantik. Nedenfor ses den abstrakte syntax for statements i PLC++.

\begin{Grammar}
 \begin{grammar}
 <St> ::= "{" $Sts$ "}" | "if" "("$e_1$")" $St$ | "if" "("$e_1$")" $St_1$ "else" $St_2$ | "while" $e$ $St$ | $F$"("$e$")" | "for" "("$e$";" $e$";" $e$")" $St$ | $x$ "+=" $e$ | $x$ "-=" $e$ | $X$ "*=" $e$ | $x$ "/=" $e$ | $x$ "\%=" $e$ | $\epsilon$
 \end{grammar}
 \caption{Abstrakt syntaks for statement}\label{gra:Statements}
\end{Grammar}


\subsubsection*{While-løkke}
For at kunne iterere i sproget er der blevet valgt både while- og for-løkke som en del af sproget. Nedenfor ses semantikken for while-løkker. \\

\noindent For $[WHILE_\top]$ læses semantikken således: 
hvis man i $Env$ eksekverer $St$ i $sto$ og får final state $sto''$. Hvis derefter i $Env$ ekseverer $while(e) St$ i $sto''$ får final state $sto'$. Så får vi ved eksekvering af $while(e) St$ i $sto$ final state $sto'$ hvis $e$ evaluerer til $true$ i $Env$.

\begin{align*}
&[WHILE_\top] & &Env \vdash \langle St, sto \rangle \rightarrow sto^{\prime\prime}\\
& & &\frac{Env \vdash \langle \text{while } (e)\; St,\; sto^{\prime\prime} \rangle \rightarrow sto^\prime}{Env \vdash \langle \text{while } (e)\; St,\; sto \rangle \rightarrow sto^\prime} & &\text{if } Env \vdash e \rightarrow_e \top\\\\
%
&[WHILE_\bot] & &Env \vdash \langle \text{while } (e)\; St,\; sto \rangle \rightarrow sto & &\text{if } Env \vdash e \rightarrow_e \bot\\\\
\end{align*}

\noindent I sproget findes også for-løkker. Disse er ikke beskrevet i semantikken, da disse på simpel vis bliver omskrevet til while-løkker inden code generation. I kodeeksempel \ref{code:fortowhile} ses hvordan en while-løkke, som er omskrevet fra en for-løkke, ser ud i konkret kode. Det tilsvarende \gls{ast} ses i figur \ref{fig:fortowhile}

\PPP{Kode/ForToWhile.ppp}{Omskrivning af for-løkker til while-løkker}{fortowhile}

\tikzfigure{Figurer/TikZ/ForToWhile.tex}{AST for For til While}{fortowhile}{1.0}

\noindent \textit{ASTSimplify}-klassen, som sørger for alle omskrivninger, indeholder desuden et tjek, hvor det tjekkes om der overhovedet er en condition i den pågældende for-løkke. Hvis ikke der er det, er det kun nødvendigt med en \textit{TrueExpr} - et eksempel på dette ses i kodeeksempel \ref{code:fortowhilesimple} og figur \ref{fig:fortowhilesimple}

\PPP{Kode/ForToWhileSimple.ppp}{Omskrivning af simple for-løkker til while-løkker}{fortowhilesimple}

\tikzfigure{Figurer/TikZ/ForToWhileSimple.tex}{Det resulterende AST for en omskrivning af simpel for-løkke}{fortowhilesimple}{1.0}

\subsubsection*{If statements}
$If$ statement er en af de selektive kontrolstrukturer der er valgt til at kunne selektere eksekvering af kode. $If$ statementet tager et boolsk udtryk der afgør om dens krop skal eksekveres. Hvis det boolske udtryk evaluerer til $false$ springes kroppen over. Dette $if$ statement fungerer som man kender det i C. Nedenfor ses semantikken for $if$ statementet.

\noindent For $[IF_\top]$ læses semantikken således: 

Hvis man i $Env$ eksekverer $St$ i $sto$ og får final state $sto'$, så får vi $sto'$ ved eksekvering af udtrykkel $if(e)  St$ i $sto$, hvis $e$ evaluerer til $true$ i $Env$.

\begin{align*}
&[IF_\top] & &\frac{Env \vdash \langle St, sto \rangle \rightarrow sto^\prime}{Env \vdash \langle \text{if } (e)\; St,\; sto \rangle \rightarrow sto^\prime} & &\text{if } Env \vdash e \rightarrow_e \top\\\\
\end{align*}

%%%%%%%%%%%%%%%%%%%%%%%%%%%%%%%%%%%%%%%%%%%%%%%%%%%%%
\subsubsection*{Structs}
For at kunne definere mere avancerede typer med tilhørende adfærd er structs blevet implementeret i syntaksen. Structs kan indholde felter af andre typer som man kender i C. Som en udvikling af C's struct er funktioner i struct ligeledes blevet udviklet, da det ønskes at kunne definere adfærd.

Felter i structs kan tilgås i semantikken på to forskellige måder alt efter om det er en variabel eller et funktionskald. Nedenfor ses semantikken for at tilgå både et felt som er en variabel samt et funktionskald.

For $[SVAR]$ læses semantikken således:
Hvis $X$ evaluerer til $l$ i $env'_s$ og $env'_v$ så evaluerer $s.X$ til $l$ i $env_s$, $env_v$ og $sto$ hvor $env_s s$ er lige med $env'_S$, $env'_F$ og $env'_S$. 


For $[SFUNC]$ læses semantikken således: \textit{s.SF}
Hvis $SF$ evaluerer til $St$, $env''_v$, $env''_F$ og $env''_S$ i $env'_s$ og $env'_F$ så evaluerer $s.SF$ til $St$, $env''_v$, $env''_F$ og $env''_S$ i $env_s$ og $env_F$ hvor $env_s s$ er $env'_S$, $env'_F$ og $env'_S$. 


\begin{align*}
&[SVAR] & &\frac{env_S^\prime, env_V^\prime \vdash X \rightarrow l}{env_S, env_V, sto \vdash s.X \rightarrow l} & &\text{where } env_S\; s = (env_V^\prime, env_F^\prime, env_S^\prime)\\\\
&[SFUNC] & &\frac{env_S^\prime, env_F^\prime \vdash SF \rightarrow (St, env_V^{\prime\prime}, env_F^{\prime\prime}, env_S^{\prime\prime})}{env_S, env_F \vdash s.SF \rightarrow (St, env_V^{\prime\prime}, env_F^{\prime\prime}, env_S^{\prime\prime})} & &\text{where } env_S\; s = (env_V^\prime, env_F^\prime, env_S^\prime)
\end{align*}