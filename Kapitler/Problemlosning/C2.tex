% Gadeberg 16-03
\section{Løsningsforslag - C2}
Ud fra problemformuleringen er der blevet lagt vægt på en løsning der tilgodeser et højere abstraktionsniveau samt en højere effektivitet ved udvikling af programmer. 

Løsningen der er blevet valgt er et imperativt sprog der lægger sig tæt op ad C. Motivationen for at vælge C som grundpille er C's udbredelse samt at mange sprog ligner C i deres syntax. Derfor vil sproget have en højere read og write-ability for en stor del af målgruppen.

Løsningsforslaget vil i rapporten blive omtalt som C2.

\subsection{Kravspecifikation}
Kravene til C2 der er blevet valgt er C med en række ændringer for at højne abstraktionsniveauet samt øge effektiviteten.

C bidrager til kravet om abstraktionsniveauet med dets mulighed for at benytte funktioner. Ligeledes har C datatypen \textit{struct} som kan imitere objekter i den virkelige verden. 

\subsubsection{Pointeraritmetik}
En af de vigtigste ting der blev valgt som afgørende for at mere effektivt programmeringssprog var afskaffelse af pointeraritmetik. I sproget skal pointeraritmetik fjernes så programmøren ikke eksplicit skal erklære hvad han/hun vil. Pointeraritmetik er kilde til mange potentielle fejlkilder og kan være en hindring for selv øvede programmører.


\subsubsection{Boolsk datatype}
En anden ting der blev lagt vægt på var at tilføje datatypen \textit{boolean}. Den boolske datatype tilføjer en højere abstraktion af virkeligheden. Som ved pointeraritmetik giver brugen af integers til boolsk udtryk også anledning til potentielle fejlkilder, samt et mindre korrekt sprog.

