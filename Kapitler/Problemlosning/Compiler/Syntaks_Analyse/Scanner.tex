\subsubsection{Scanner}
Scanneren som er den første del af compileren der bliver kørt har til opgave at læse kildekoden og dele den op i tokens, som er de mindste dele af et sprog. Keywords i et sprog er et eksempel på tokens som skal læses. Til at illustrere dette ses her et kodestykke fra sproget Port der bygger på dele af PLC++.

% Q#0.0 = I#0.1  == I#0.2 && I#0.3 != true;
\PPP{Kode/Port/Port.tex}{Eksempel på kodestykke på Port}{port_sample}

\noindent Alle tokens er defineret på følgende måde:
\SCC{Kode/Port/Tokens.tex}{Tokens i sproget Port}{port_tokens}

\noindent Af ignorerede tokens er white\_space.

\noindent En scanner indlæser de forskellige "ord"\mbox{}, og laver dem om til tokens som parseren senere kan bruge. Et eksempel på en scanner til følgende tokens implementeret i Java uden brug af værktøjer kan ses i bilag \ref{bil:cd} eller i et udsnit i kodeeksempel \ref{code:scanner_java}.

\Java{Kode/Port/Scanner.java}{Udsnit af scanneren}{scanner_java}

\noindent Som det kan ses er scanneren meget ens kode. Dette gør det attraktivt at automatisere udviklingen af scanneren. Afsnit \ref{ssec:toolsforcc} vil komme nærmere ind på hvilke værktøjer der kan bruges til at lave en scanner.

For at illustrere et eksempel på en kørsel af scanneren på kodeeksempel \ref{code:port_simple} er denne her tabel, der viser typen af den token og det konkrete indhold.

% Q#0.0 = true;
\PPP{Kode/Port/SimplePort.tex}{Eksempel på kodestykke fra sproget Port}{port_simple}

\begin{table}[H]
\centering
    \begin{tabular}{|c|c|c|c|c|}
    \hline
    \textbf{port\_identifier} & \textbf{decimal\_literal} & \textbf{assignment\_operator} & \textbf{true\_keyword} & \textbf{semi} \\ \hline
    Q\#          & 100.0          & =                   & true              & ;             \\ \hline
    \end{tabular}
\caption{Opdeling af tokens}
\label{tab:tokensMT}
\end{table}

\noindent Da input'et nu er scannet og opdelt i tokens, er det muligt at bygge et abstrakt syntakstræ. \cite{Sebesta_2013}

%For at illustrere dette, ses et lille kodestykke i sproget Mini Triangle i kodeeksempel \ref{code:minitriangleforscanner}.

%\MT{Kode/MiniTriangleForScanner.mt}{Lille program i Mini Triangle}{minitriangleforscanner}

%\noindent Hvis der tager udgangspunkt i linje 1 fra kode \ref{code:minitriangleforscanner} ses det at der er nogle forskellige "ord"\mbox{}. Det er disse ord der skal forstås som tokens. Disse er inddelt i forskellige kategorier - typisk én kategori til hvert keyword\fn{Keyword}{Et bestemt ord i et programmeringssprog, som er reserveret til et bestemt formål.}, for eksempel \textit{let} og \textit{var}, samt identifiers. Opdelingen i tokens ses i tabel \ref{tab:tokensMT}

%\begin{table}[H]
%\centering
%    \begin{tabular}{|c|c|c|c|c|}
%    \hline
%    \textbf{let} & \textbf{var} & \textbf{identifier} & \textbf{colon} & \textbf{identifier} \\ \hline
%    let          & var          & y                   & :              & Integer             \\ \hline
%    \end{tabular}
%\caption{\textit{Opdeling af tokens}}
%\label{tab:tokensMT}
%\end{table}

%\noindent Da input'et nu er scannet og opdelt i tokens, er det muligt at bygge et abstrakt syntakstræ.