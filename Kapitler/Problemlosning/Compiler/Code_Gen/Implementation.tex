\subsubsection{Source til Target}
For at opnå kode, som kan kompileres til en \gls{plc}, har det været nødvendigt at lave interessante oversættelse, som dette afsnit vil belyse.

\begin{table}[H]
    \centering\ttfamily
    \begin{tabular}{l|l l}
        PLC++       & \gls{il} \\\hline
        i = 1 + 3;  & MOVL \&1 W402     &// Push(1)\\
                    & MOVL \&3 W404     &// Push(3)\\
                    & + W404 W402 W402  &// Push(Pop() + Pop())\\
                    & MOVL W402 i       &// i = Pop()
    \end{tabular}
    \caption{Eksempel på kode som compileren genererer}
    \label{tab:codegenExample}
\end{table}

\noindent Eksemplet i tabel {\ref{tab:codegenExample} viser hvordan koden deler udtrykket \textit{i = 1 + 3} op i flere dele. Alle disse dele stammer fra det \gls{ast}, der blev genereret til dette stykke kode, og kan ses på figur \ref{fig:codegenAST}. En ting der er vigtigt at bemærke er at visitor'en laver en \gls{dfs}, hvilken normalt går fra venstre mod højre, men for at kunne tilskrive værdien af venstre udtryk af AAssignExpr til højre udtryk, bliver den højre først evalueret. Dette er én af de modifikationer som \textit{CodeGenerator}-visitoren har.

\tikzfigure{Figurer/TikZ/CodeGenExampleAST.tex}{\gls{ast} for samme stykke kode som i tabel \ref{tab:codegenExample}}{codegenAST}

\begin{table}[H]
    \centering\ttfamily
    \begin{tabular}{l|l|l l}
         &PLC++                 & \gls{il} \\\hline
        1&if(true)\{            & SET W402.00       &// Push(true)\\
        2&  i = 1;              & CJP \#0           &// Jump if false to JME \#0\\
        3&\} else \{            & LD W402.00        &// Start a new ladder\\
        4&  i = 0;              & // Udeladt: i = 0\\
        5& \}                   & JMP \#1           &// Jump to JME \#1\\
        6&                      & JME \#0           &\\
        7&                      & LD P\_FirstCycle  &// Start a new ladder\\
        8&                      & // Udeladt: i = 1\\
        9&                      & JME \#1
    \end{tabular}
    \caption{Eksempel på hvordan kode generatoren håndterer et if-else statement.}
    \label{tab:codegenIf}
\end{table}

\noindent Et ligende eksempel er når man compilerer en if-else kæde, som set i tabel \ref{tab:codegenIf}. Her bliver der byttet rundt på rækkefølgen af \textit{Do something} og \textit{Do something else}, da man ellers skulle invertere resultatet fra betingelsen, i dette tilfælde \textit{true}, fordi instruktionen \textit{CJP} kun bliver udført, hvis den modtager et lavt signal.

\tikzfigure{Figurer/TikZ/CodeGenIfAST.tex}{\Gls{ast} for samme stykke kode som i tabel \ref{tab:codegenIf}. Detalier om hvad der sker med tokens inden i løkker er udeladt.}{codegenIfAST}

\noindent En sidste metode, som er interessant at tage fat i, er genereringen af arrays, fordi \gls{plc}'er ikke har pointere på samme måde, som man ville have dem i eksempelvis Assembly og C. Det betyder at PLC++ compileren bliver nødt til at finde på et alternativ.

\begin{table}[H]
    \centering\ttfamily
    \begin{tabular}{l|l|l l}
         &PLC++       & \gls{il} \\\hline
        1&i = A[2];   & MOVL A W402       &// Push(address of first element in A)\\
        2&            & MOVL \&2 W404     &// Push(2)\\
        3&            & * W404 \&2 W404   &// Scale to word size\\
        4&            & + W404 W402 W402  &// Add offset and first element\\
        5&            & MOVL W402 D24     &// Move to data memory\\
        6&            & MOVL @D24 W402    &// Push the value D24 points to\\
        7&            & MOVL W402 i       &// i = Pop()
    \end{tabular}
    \caption{Eksempel på hvordan kode generatoren ville generere en værdi fra et array.}
    \label{tab:codegenArray}
\end{table}

\noindent Arrays i programmet bliver håndteret ved at en mængde adresser bliver reserveret til dataen et symbol bliver indsat, som holder værdien af den første adresse af arrayet. Måden hvorpå værdierne repræsenteres i arrayet, kan ses i tabel \ref{tab:codegenArray}. Dette forgår ved at den henter addressen på første element og hvor langt den skal gå fra den (linje 2-3), lægger dem sammen (linje 4) og bruger resultatet som en pointer til at hente værdien (linje 6).\\

%\subsubsection{Implementation}
\noindent Måden code generatoren er implementeret på, er som før nævnt gennem et visitor-mønster. Klassen hedder \textit{CodeGenerator} og består at funktioner til visitor-mønstret samt en funktion til at skrive resultatet til en fil; \textit{Emit}.

\Java{Kode/CodeGeneratorEmit.java}{Emit-funktionen fra CodeGenerator klassen}{emit}

\noindent \textit{Emit}-funktionen, som ses i kodeeksempel \ref{code:emit}, er relativt simpel. Den åbner eller lukker ikke for nogle af sine text writers, men skriver blot til dem ud fra en boolsk værdi. Grunden til dette er blandt andet at CX-Programmer, som bliver brugt til videre kompilering af programmet, gemmer en symboltabel, som PLC++ udnytter til at forøge læsbarheden af den kompilerede kode ved at erstatte vilkårlige adresser med variable navne. 

For at kunne bruge mange af de teknikker, som allerede er udviklet til assembly-compilere, er det nødvendigt at have en stack. \gls{plc}'en har en indbygget stack, som viste sig at være utilstrækkelig, grundet manglende stackpointer, som betød at det ikke var muligt at tage det x. element. Derfor har PLC++ compileren en indebygget stack, der bruger working memory området W400 - W512, som har en pointer således at man frit kan vælge fra stacken.

\Java{Kode/Stack.java}{Push()-funktionen fra Stack klassen}{push}

\noindent Klassen \textit{Stack} sørger for dette med standardfunktionerne, som forventes af en stack-datastruktur; \textit{push()} og \textit{pop()}. \textit{Push()} ses i kodeeksempel \ref{code:push}. Her tager den hensyn til hvilken variabel, der bliver pushet til stacken, men i sidste ende kommer de alle til at være gemt som den største datatype, der er til gængelig på \gls{plc}'en. Det giver noget dataspild, men til gengæld er \textit{pop()}-funktionen kun en decrement af pointeren og returnering af værdien, hvorefter modtageren håndterer denne.