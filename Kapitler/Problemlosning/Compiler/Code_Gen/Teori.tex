%\subsubsection{Baggrund}
For generere \gls{il}-koden blev der udviklet endnu en \textit{visitor}. Denne vistor's primære opgaver er at skrive \gls{il}-kode til en destinationsfil, som udfører samme opgave, som beskrevet i kildekoden. Da \gls{il} minder om \textit{Assembly} i syntaksen, og tilgængelige instruktioner, kan der bruges mange teknikker og teorier fra andre compilere. Ud over de små syntaktiske forskelligheder, er \gls{il} fokuseret på at værdierne skal kunne repræsenteres i den virkelige verden i form af noget output. Derfor har \gls{il} flere muligheder for at bruge enkelte bits - eksempelvis har de fleste hukommelsesområder udvidelser til adresserne, som giver direkte adgang til bitsne. \\

%\paragraph{Hukommelses områder}
\noindent En anden grundlæggende forskel mellem Assembly og \gls{il} er abstraktionen af hukommelse. I Assembly er der kun en hukommelse, som bliver tilgået med adresser. Dette giver muligheder for pointere, som kan pege til både værdier i stack'en og heap'en.

\noindent I \gls{plc}'er er det opdelt i hukommelsesområder, som alle bliver refereret til med et prefix til adressen, for eksempel {\ttfamily W402}, hvor {\ttfamily W} er hukommelsesområdet og {\ttfamily 402} er adressen på værdien. Dette giver bedre overblik over hvilken type hukommelse man tilgår - til gengæld findes pointere kun i ét af hukommelsesområderne. Det betyder at \enquote*{call by reference} er upraktisk, hvis man vil have data i andre områder.\\

\noindent For at udnytte hukommelsen på \gls{plc}'en bedre, bruger PLC++ datahukommelsesområdet ({\ttfamily D}) til data, som variabler og array, og Working-hukommelsesområde ({\ttfamily W}) til en stack datastruktur.Den eneste ulempe ved dette, er at det er nødvendigt at flytte værdien over til {\ttfamily D}-området, hvis det ønskes at gøre brug at pointere. Dette ses i eksemplet på arrays i tabel \ref{tab:codegenArray}. Årsagen til at {\ttfamily W} bruges til stack, fremfor {\ttfamily D}, er at {\ttfamily W} er meget hurtigere at tilgå på \gls{plc}'en. Dette skyldes at {\ttfamily D} er et persistent hukommelsesområde, hvilket typisk er langsommere end et ram-lignende område.