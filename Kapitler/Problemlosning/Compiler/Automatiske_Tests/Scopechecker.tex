\subsubsection{Scopechecker}

\noindent De to første tests vist i kodeeksempel \ref{code:assignmenttest} viser to forskellige former for assignments, som derudover også bidrager til forståelsen af scope-regler i PLC++. Den tredje viser scope-regler for funktioner. I den første test, \textit{errorOnNonDeclaredVariable()}, prøves der er tildele værdien 1 til \textit{i}, hvilket bør give fejl da variabler ikke kan bruges før de er deklareret - dette giver altså en \textit{SymbolNotFoundException}.

\noindent Test 2, \textit{canAssignToDeclaredVariableInParentScope()}, viser at det med scope-reglerne skal være muligt at have globale variable. Dette ses ved at der tildeles en værdi til \textit{i}, som tidligere blev deklareret udenfor funktionen.\\

\noindent Den sidste test, \textit{canCallFunctionBelow()}, viser at en funktion, som er deklareret nedenunder kaldstedet, bliver kaldt. I et sprog som C, ville dette ikke kunne lade sig gøre uden prototypes. Dette bruges dog ikke i PLC++, men alligevel fejler testen ikke - dette kan der læses mere om i afsnit \ref{ssec:contextimple}

\Java{Kode/Tests/Scopechecker/Assignments.java}{Assignment Tests}{assignmenttest}