\subsubsection{Parser}
\label{subsec:parser}

Muligheden for at lave korte if-else sætninger, hvor man ikke eksplicit angiver start og slut (enten ved akkolader eller \textit{end}), åbner op for en interessant problemstilling: dangling else (se afsnit \ref{subsec:danglingelse} for mere info). 

\PPP{Kode/Tests/Parser/DanglingElse1.ppp}{Dangling Else, mulighed 1}{danglingelseppp1}

\PPP{Kode/Tests/Parser/DanglingElse2.ppp}{Dangling Else, mulighed 2}{danglingelseppp2}

\noindent Kodeeksempel \ref{code:danglingelseppp1} og \ref{code:danglingelseppp2} illustrerer dette problem. Det er præcis det samme stykke kode - det er bare umuligt at se hvad \textit{else} på linje 4 knytter sig til. Implementationen af løsning findes i afsnit \ref{subsec:danglingelse}, men det er også en god idé at lave automatiserede tests, som tester om løsningen opfører sig som forventet. Testen ses i kodeeksempel \ref{code:danglingelsetest}

\Java{Kode/Tests/Parser/DanglingElseTest.java}{Dangling Else Test}{danglingelsetest}

\noindent På linje 6 ses det at den første, og eneste, funktion hentes ind i variablen \textit{function}. Derefter findes \textit{ifStatement}, som er den første if-sætning (den uden else). Derefter findes \textit{ifElseStatement}, som skulle indeholde både en \textit{if} og en \textit{else}.

\noindent Linje 14 og 15 sørger for at tjekke om det pågældende else-statement rent faktisk binder sig til det der forventes (kodeeksempel \ref{code:danglingelseppp1}). På linje 14 tjekkes det om \textit{ifStatement} har en \textit{sibling} til højre, hvilken den ikke bør have, da \textit{else} ikke binder sig til den. Næste linje tjekker om \textit{ifElseStatement} har - altså at \textit{getRight()} ikke returnerer \textit{null}. 

Hvis begge disse antagelser er sande, er testen gået igennem, og dangling else problemet er løst som det forventes.