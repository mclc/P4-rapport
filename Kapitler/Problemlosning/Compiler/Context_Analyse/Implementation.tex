\subsubsection{Implementation}
\label{ssec:contextimple}
Context analysen implementeret i compileren, består af to dele; en scope checker og en type checker. Scope checkeren opretter forskellige miljøer for forskellige scopes og tildeler symbolerne til de miljøer der koresponderer til deres scope. På den måde kan symboler fra utiltænkte miljøer ikke tilgåes.

Type checkeren kontrollerer at de typer der benyttes i udtryk og statements er ækvivalente med de typer som udtrykene og statementsne accepterer. Men før det er muligt at kigge på scope- og type checker skal visitor mønstret, som det er implementeret i SableCC, gennemgås.

\subsubsubsection{Visitor mønstret i SableCC}\label{sct:visitorSableCC}
SableCC generer som standard et interface ved navn \textit{Analytics} som implementerer interfacet \textit{Switch}. \textit{Analytics} interfacet indeholder en metode for hver mulig node eller token i programmet, med præfixet case, som set i klassediagrammet på figur \ref{fig:sableCCAnalytics}.

 Alle noder i \gls{ast} implementerer interfacet \textit{Switchable}, som har en metode ved navn \textit{apply}, som tager klassen \textit{Switch} som input parameter. Noden skal dermed kalde sin egen metode i \textit{Analytics} interfacet. Et eksempel er at noden \textit{ASubExpr} kalder metoden \textit{caseASubExpr} i \textit{Analytics} interfacet som ses i kodeeksempel \ref{code:nodeVisitorJava}

\Java{Kode/NodeVisitor.java}{Implementering af interfacet \textit{Switchable} i noderne genereret af SableCC}{nodeVisitorJava}

\noindent Dette gør det muligt at lave en \textit{AnalyticsAdapter} som implementerer \textit{Analytics} interfacet. Alle case metoder i \textit{Analytics} interfacet kalder en metode ved navn \textit{defaultCase}, som tager noden som input parameter. Metoden \textit{defaultCase} gør som standard ikke noget, metoden er dog virtual og kan overrides. \textit{AnalyticsAdapter} implementerer derudover også metoder til at gemme data for hver node i en hashtable, som normalt ville implementeres direkte i \gls{ast}. Dette skyldes SableCC's fleksibilitet i forhold til oprettelse af \gls{ast} ikke er særlig omfattende.

%\bigskip

Til at gennemløbe træet genererer SableCC en klasse med navn \textit{DepthFirstAdapter}, som nedarver fra \textit{AnalyticsAdapter} klassen. \textit{DepthFirstAdapter} overskriver alle case funktioner på noder (ikke tokens) således, at alle case funktioner kalder \textit{apply} med \textit{DepthFirstAdapteren} som argument på alle børn, fra venstre mod højre, i den pågældende node. Det bruges dermed til en simpel \gls{dfs} (Se klassediagram på figur \ref{fig:sableCCAnalytics}). For at have øge fleksibiliteten oprettes en \textit{in} og \textit{out} metode for alle de overskrevne case metoder. Case metoden kalder nu først in metoden, dernæst \textit{apply(this)} på alle børn og til sidst out metoden. Man starter hele denne process ved at kalde \textit{apply} metoden på roden, som genereres under kørsel af parseren.\\
% Det er dermed muligt at bruge en stack til at håndtere matematiske operationer kun ved at overskrive out metoderne.


%\noindent Processen med at gennemløbe træet startes ved at kalde apply på start noden som genereres under kørslen af parseren, metoden apply tager som input parameter et objekt der implementer Analytics interfacet (Switchable for at være nøjagtig, dog castes der altid til Analytics og der vil dermed blive kastet en ClassCastException, hvis argumentet er et objekt af typen Switchable).

\tikzfigure{Figurer/TikZ/KlassediagramSabelCC.tex}{Klassediagram for SableCC's visitor mønster}{sableCCAnalytics}{1.0}

\noindent SableCC's visitor mønster er nu forklaret, som skal bruges til scope checkeren, type checkeren og code generatoren.

\subsubsubsection{Scope checker}
Scope checkerens opgave er at tjekke at de konkrete identifiers findes og kan tilgås i de konkrete scopes. Dog skal der her tages højde for om det er statisk- eller dynamisk scope. PLC++ specifikationen siger at det skal være dynamisk scope for funktioner, \textit{structs} og globale variabler, mens der skal være statisk scope inden for alle funktioner. Ved at have dynamisk scope for funktioner og structs er det muligt at slippe for header filer, prototyper og dermed nogle fejl, hvis headerfilerne bliver konfigurereret forkert.\\

\noindent For at håndtere dynamisk scope køres først en præprocessor, der indlæser funktioner, structs og globale variabler i symboltabellen. Dernæst kører scopecheckeren, som tjekker hver  identifier for om den findes i det pågældende scope. Hvis identifieren forsøges erklæret selv om den allerede eksisterer i symboltabellen, rapporterer compileren om fejl. Scopecheckeren håndterer også oprettelsen af lokale variabler, således det ikke er muligt at bruge en lokal variabel, der ikke er blevet deklareret endnu, herunder \textit{static scope}.\\

\noindent For at forhindre at alle funktioner, structs og globale variabler bliver registret to gange (både i præprocessoren og scopecheckeren) overrides case metoderne fra DepthFirstAdapteren for de tilsvarende funktioner, og de bliver sat til at udføre \textit{caseDefault}.

\subsubsubsection{Type checker}
Type checkerens opgave er at verificere, at typerne stemmer overens med det tilladte for en given operation. 
Expressions evalueres på en stack ved hjælp af \textit{DepthFirstAdapteren}'s out metoder. Eksempelvis vil udtrykket 3 + (5 * 2) blive evalueret som vist på figur \ref{fig:stackExprEval}:

\figur{Figurer/Billeder/stack.png}{Evaluering af udtrykket 3 + (5 * 2) ved hjælp af en stack}{stackExprEval}{0.7}

\noindent Dette gøres ved brug af out metoderne som beskrevet i afsnit \ref{sct:visitorSableCC}. Det parsede \gls{ast} kan ses på figur \ref{fig:evaluationAST}
\tikzfigure{Figurer/TikZ/EvaluationOfSimpleMath.tex}{\gls{ast} for udtrykket 3 + (5 * 2)}{evaluationAST}

Stacken vil indlæse de 3 første tal, multiplicere de 2 seneste tal og ligge på stacken. Til slut bliver det første tal adderet med resultatet fra multiplikationen af 5 og 2. Udtrykket kan dermed opstilles til 3 5 2 * +. Denne notation kaldes også omvendt polisk notation \cite{CraftingCompiler_2009} eller postfix notation.

I typecheckerens eksempel er det nødvendigt at registrere typen af tallene, hvorved man vil finde at true + 1 er et ugyldigt input, hvorimod 1 + 1 er gyldigt. Dette gøres på samme måde ved brug af stacken, bortset fra at der ligger et objekt på stacken, der indikerer typen for værdien. I kodeeksempel \ref{code:exprTypeEvaluator} ses et eksempel på hvordan konstanter bliver pushed til stacken, samt hvordan de 2 seneste værdier sammenlignes med større end operator.  

\Java{Kode/ExprTypeEvaluator.java}{Push af konstanters type på stacken som giver mulighed for typecheck}{exprTypeEvaluator}