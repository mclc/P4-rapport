\subsubsection{Type Checking}
    
    Type checking er øvelsen hvor man kontrollere at man ikke prøver at udføre operation der ikke er lovlige ifølge grammatikken. I mange programmerings sprog kan man fx ikke gange strings med integers i de fleste programmerings sprog. Dette er en ren kontekstuel del af grammatikken og er derfor ikke en del af \gls{cfg}.
    
    Måden at gøre dette på er ved at visitoren besøger noderne i \gls{ast}'et depth first, hvor den samler information om en nodes eventuelle børn som den så bruger til at verificere at den forventede type stemmer overens med den aktuelle type. Dette kan ses i figur \ref{fig:typecheck}.
    
    \tikzfigure{Figurer/TikZ/TypeChecking.tex}{Et eksempel på kontrol af typer for udtrykket: bool flag = 1 <= f}{typecheck}
    
\noindent Fra figuren kan vi se hvordan at et declaration med en initializations værdi bliver udført bliver type checket. Først findes den forventet type som kommer fra erklæringen af variablen \enquote*{flag}. Bagefter finder vi den aktuelle type assignmentet ved at finde resultatet af udtrykket, som i dette tilfælde er et binær operator, som sammenligner to tal; en integer og en float. Denne kompiler køre under generaliserings princippet, hvilket vil sige at hvis tilfælde som dette opstår, vil den type der er mindst blive implicit omformet til den anden type, som set i figuren under BinaryOperation's noden. I noden bliver resultatet så evalueret til at være \enquote*{bool} som returneres til assignment noden som den aktuelle type.
    
\subsubsection{Scope Checking}

At scope checke betyder at kontrollere for at et stykke kode ikke forsøger at få fat i variabler som ikke er tilgængelig for den, dette kan for eksempel være lokale variabler i en anden metode. De præcise regler for et sprog kan variere med designet af sproget, men der er hovedsageligt to grund principper til hvordan scopet fungere; dynamisk og statisk
    
\subsubsubsection{Dynamisk scope} 
Dynamisk scope er en gammel og ikke så velset metode for at dele scopes ind var at man dynamisk satte referencen til sidste gang variablen blev brugt, dette betød at noget kode kunne blive uoverskueligt eftersom at deklarationen af variabler kan bliver gemt inden i funktioner, og man ender meget nemt i situationer hvor en funktion kommer til at overskrive en variable som dermed giver uventede resultater. 
    
\subsubsubsection{Statisk scope} 
Statisk scope betyder derimod at man indeler sin kode i blokke hvor man altid vil tilgå den nærmeste variable med et givet navn. Normale regler for denne type scope giver den lov til at hente data fra det scope dette scope blev dannet i, men ikke den anden vej. Dette giver en træ ligende struktur hvor bladene kan søge mod roden for at finde metoder og variabler.


   % Det vil sige at en variable der er defineret i et scope bør være tilgængelig fra under scopes, men ikke de større scopes. Dette er selfølgeligt afhængigt af designet af sproget.
    
    %\tikzfigure{Figurer/TikZ/SableCC_NFA.tex}{}{}
    
\subsubsection{Symbol tabel}
    Som nævnt flere gange i de forrige vil mange af kontrollerne der bliver udført i løbet af context analysen ikke fungere uden en symbol tabel, dette kapilel vil prøve at redegøre for hvad der et. Når man udvikler en kompiler til et sprog med brugerdefineret variabler, er det en fordel at kompileren har et fast sted hvor den kan slå op for at finde ud af om variablen er defineret, hvad type den har, og eventuelt hvad scope den er defineret i, på tværs af scopes. Dette kan gøres med en datastruktur som en symbol tabel, som set i tabel \ref{tab:symboltabel}.
    
\PPP{Kode/SymTableExample.c}{Program brugt til at danne følgende symbol tabel}{tableprog}

    \begin{table}[H]
    \centering\footnotesize
    \begin{tabular}{l|l|l}
    \textbf{Id} & \textbf{Type} & \textbf{Scope} \\\bottomrule
    foo & function, double & global\\
    count & integer & function parameter\\
    sum & double & local block\\
    i & integer & for-loop statement
    \end{tabular}
    \caption{Eksempel på en symbol tabel for et simpelt program.}
    \label{tab:symboltabel}
    \end{table}
    
\noindent I denne tabel kan man se hvordan man kan finde viden om hvad type og scope et symbol. Dette er én måde at implementere symbol tabellen; en usorteret liste der indeholder nogle informationer om hvert enkelt element. Den er forholdsvis ineffektiv løsning eftersom at det vil tage $O(n)$ for at finde et symbol, hvor $n$ er antallet af elementer i listen. En bedre løsning vil være at implementere strukturen som et balanceret træ eller et bibliotek.\\
    
\noindent Man kunne også ofre lidt effektivitet og i stedet dele tabellerne op i flere efter hvad scope man befinder sig i. Dette betyder at vi naturligt får et statisk scope, hvor man kun kan få adgang til elementer i scopet, og det er let at implementere så man også kan søge i den ydre blok hvis man har en reference til den.
    
    %\tikzfigure{Figurer/TikZ/SymbolTable.tex}{Flere symbol tabeller som de fungerer i compileren til \sprognavn}{symtab}
    
    %\mfix{Hvad sker der for figuren?}
    
\noindent Ulempen ved at gøre det på denne måde er at fordi vi har flere tabeller bliver vi måske nød til at søge gennem flere for at finde et givet element hvis det ikke er erklæret i samme scope. 

\section{Syntaks og semantik}\label{sec:Syntax}
I dette afsnit vil de mest relevante dele af syntaksen med tilhørende semantik blive belyst. Den fulde syntaks og semantik specifikation kan findes i bilag \ref{bil:cfg} og \ref{bil:semantik}. Den grammatik der refereres til i dette afsnit er den abstrakte syntaks.

\subsection{Typer}
Inden gennemgang af syntaksen omtales de mest nødvendige datatyper.

\subsubsection*{Bool}
Den boolske datatype repræsenterer værdien \textit{sand} eller \textit{falsk}. Denne type kan ikke konverteres til andre typer ligesom i \textit{C} eller \textit{C++}. Alle boolske udtryk returnerer denne type.

\subsubsection*{Int}
\textit{Int} datatypen er datatypen beregnet til at holde heltalsværdier. \textit{Int} er en signed integer på 16bit. Det første bit er signbittet og bestemmer om værdien er positiv eller negativ. \textit{Int} kan maksimalt holde værdien 2\textsuperscript{15}-1 = 32767 og minimalt holde -2\textsuperscript{15} = -32768.

\subsubsection*{Array}
\textit{Array} datatypen er en datastruktur der kan holde n antal elementer i sekvens, hvor n angives på instansieringstidspunktet. Alle elementer i arrayet skal være i samme type og kan tilgås ved at skrive araryets navn efterfulgt af en integer \textit{i} indpakket i et sæt hårde parenteser, der returnerer det \textit{i.} element.

\subsubsection*{Port}
\textit{Port} datatypen er adskilder sig fra de andre datatyper ved at være være et alias for port nummeret på \gls{plc}'en. Den kan bruges hvis en yderligere eller mere hensigtsmæssig navngivning ønskes til ind og udgangsporte på \gls{plc}'en.
Af porte findes der digitale og analoge inputs samt digitale og analoge outputs. De digitale porte kan have den samme værdi som den boolske datatype, altså \textit{sand} eller \textit{falsk}. De Analoge datatyper holder en numerisk værdi fra 0 til 1000 som afspejler den analoge ports værdi i promille.

\subsection{Udtryk}
Et udtryk er en sekvens af operatorer og operander. I PLC++ findes der 2 typer udtryk. Aritmetiske og boolske udtryk. Aritmetiske udtryk returnerer et hel- eller kommatal. Boolske udtryk returnerer en bool. Før der gåes i detaljen med de forskellige udtryk bliver vi nødt til at snakke om hvilke operatorer der kan benyttes.

\subsubsection{Operatorer}
I tabellen herunder ses operatorerne der kan benyttes i udtryk. Ved sammensatte udtryk med forskellige operatorer er operator prioritering nødvendig. Hvis ikke disse regler fastsættes kan der fåes forskellige resultater, alt efter hvilken rækkefølge udtrykket udregnes. Tabellen er ordnet efter prioritering hvor de øverste udregnes før de nederste.
\begin{table}[H]
    \centering
    \begin{tabular}{|l|l|}
        \hline
        \centering

        Primær             & x++ \quad x- - \quad a{[}x{]} \quad x.y \quad (x)                 \\ \hline
        Unær               & ++x \quad - -x \quad ! \quad -x \quad +x \quad (cast)x       \\ \hline
        Multiplikationer   & * \quad / \quad \%                                               \\ \hline
        Additioner         & + \quad -                                                        \\ \hline
        Sammenligning      & \textless \quad \textgreater \quad \textless= \quad\textgreater= \\ \hline
        Ligheder           & == \quad !=                                                      \\ \hline
        Logisk og          & \&\&                                                              \\ \hline
        Logisk eller       & ||                                                               \\ \hline
        Betinget udtryk    & ?:                                                               \\ \hline
        Sammensatte udtryk & *= \quad /= \quad \%= \quad += \quad -=                          \\ \hline
        Assignment         & =                                                                \\ \hline

    \end{tabular}
    \caption{\textit{Operatorprioritering (højest til lavest)}}
    \label{tab:operatorprioritering}
\end{table}
\subsubsection{Boolske udtryk}
Boolske udtryk kendetegnes ved at have logisk og, logisk eller, Sammenlignings eller ligheds operatorerne. operatorerne ses herunder.

\begin{table}[H]
    \centering
    \begin{tabular}{|l|l|}
        \hline
        \centering

        Sammenligning      & \textless \quad \textgreater \quad \textless= \quad\textgreater= \\ \hline
        Ligheder           & == \quad !=                                                      \\ \hline
        Logisk og          & \&\&                                                              \\ \hline
        Logisk eller       & ||                                                               \\ \hline


    \end{tabular}
    \caption{\textit{Operatorer i boolske udtryk}}
    \label{tab:operatorprioritering}
\end{table}

\noindent\textbf{And expression}

\noindent Et "and expression" er et udtryk der tager 2 operander og returnerer true hvis begge operander er sande. hvis ikke returnerer udtrykket false.
herunder kan gramatikken ses i BNF.
\mfix{Centrer dette BNF exsempel}
\input{Kode/CFG/BNF_and_expr.scc}


    \bgroup
    \def\arraystretch{3}
    \begin{table}[H]
    \centering
    \begin{tabular}{l c l}
        
        $[AND-1_{BS}]$ &$\frac{and\_expr \rightarrow \; tt \quad equality\_expr \rightarrow \; tt}{and\_expr \bigwedge equality\_expr \rightarrow \; tt}$ & \\
    
        $[AND-2_{BS}]$ &$\frac{and\_expr \rightarrow \; ff}{and\_expr \bigwedge equality\_expr \rightarrow \; ff}$ & \\
        
        $[AND-2_{BS}]$ &$\frac{equality\_expr \rightarrow \; ff}{and\_expr \bigwedge equality\_expr \rightarrow \; ff}$ & \\
        
    \end{tabular}
    \caption{AND expression}
    \label{tab:andexpr}
    \end{table}
    \egroup

\noindent\textbf{Or expression}

\noindent Et or expression tager ligesom and expression 2 operander men returnerer sand hvis bare et af operanderne er sande. Gramatikken for and expression kan ses herunder.

\input{Kode/CFG/BNF_or_expr.scc}

    \bgroup
    \def\arraystretch{3}
    \begin{table}[H]
    \centering
    \begin{tabular}{l c l}
        
        $[OR-1_{BS}]$ &$\frac{b_1 \rightarrow \; tt \quad b_2 \rightarrow \; ff}{b_1 \bigvee b_2 \rightarrow \; tt}$ & \\
    
        $[OR-2_{BS}]$ &$\frac{b_i \rightarrow \; ff}{b_1 \bigvee b_2 \rightarrow \; ff}$ & $i \in \{1, 2\}$\\
        
    \end{tabular}
    \caption{OR expression}
    \label{tab:orexpr}
    \end{table}
    \egroup
    
\textbf{Equality expressions}

\noindent\noindent Whatever
    
    \input{Kode/CFG/BNF_equality_expr.scc}


    \bgroup
    \def\arraystretch{3}
    \begin{table}[H]
    \centering
    \begin{tabular}{l c l}
        
        $[EQUALS-T_{BS}]$ &$\frac{expr_1 \rightarrow v_1 \quad expr_2 \rightarrow v_2}{expr_1\;=\;expr_2 \rightarrow \; tt}$ & if $v_1 = v_2$ \\
        
        $[EQUALS-\bot_{BS}]$ &$\frac{expr_1 \rightarrow v_1 \quad expr_2 \rightarrow v_2}{expr_1\;=\;expr_2 \rightarrow \; ff}$ & if $v_1 \ne v_2$ \\
        
    \end{tabular}
    \caption{Equality expressions}
    \label{tab:equaexpr}
    \end{table}
    \egroup

\noindent \subsubsection{Aritmetiske udtryk}

\noindent \textbf{Add and sub expressions}

    \bgroup
    \def\arraystretch{3}
    \begin{table}[H]
    \centering
    \begin{tabular}{l c l}
        
        $[PLUS_{BS}]$ &$\frac{expr_1 \rightarrow v_1 \quad expr_2 \rightarrow v_2}{expr_1\;+\;expr_2 \rightarrow v}$ & where $v = v_1 + v_2$ \\
        
        $[MINUS_{BS}]$ &$\frac{expr_1 \rightarrow v_1 \quad expr_2 \rightarrow v_2}{expr_1\;-\;expr_2 \rightarrow v}$ & where $v = v_1 - v_2$ \\
        
    \end{tabular}
    \caption{Add and sub expression}
    \label{tab:addandsub}
    \end{table}
    \egroup

    \begin{Grammar}
 \begin{grammar}

    <add\_sub\_expr> ::= <add\_sub\_expr> "+" <mult\_div\_mod\_expr>
    \alt <add\_sub\_expr> "-" <mult\_div\_mod\_expr>
  
 \end{grammar}
 \caption{Add and sub expressions}\label{gra:addsub}
\end{Grammar}
    
  
\noindent \subsection{Declarations}
Declarationer kommer i 2 forskellige varianter. En med initialisering og en uden.
\noindent \subsection{Statements}
Sproget PLC++ består af en række forskellige statements. Statements som funktionskald, iterative og selektive kontrolstrukturer. Under dette afsnit vil de mest betydende statements blive gennemgået og forklaret med tilhørende semantik. Nedenfor ses den abstrakte syntax for statements i PLC++.

\begin{Grammar}
 \begin{grammar}
 <St> ::= "{" $Sts$ "}" | "if" "("$e_1$")" $St$ | "if" "("$e_1$")" $St_1$ "else" $St_2$ | "while" $e$ $St$ | $F$"("$e$")" | "for" "("$e$";" $e$";" $e$")" $St$ | $x$ "+=" $e$ | $x$ "-=" $e$ | $X$ "*=" $e$ | $x$ "/=" $e$ | $x$ "\%=" $e$ | $\epsilon$
 \end{grammar}
 \caption{Abstrakt syntaks for statement}\label{gra:Statements}
\end{Grammar}


\subsubsection*{While-løkke}
For at kunne iterere i sproget er der blevet valgt både while- og for-løkke som en del af sproget. Nedenfor ses semantikken for while-løkker. \\

\noindent For $[WHILE_\top]$ læses semantikken således: 
hvis man i $Env$ eksekverer $St$ i $sto$ og får final state $sto''$. Hvis derefter i $Env$ ekseverer $while(e) St$ i $sto''$ får final state $sto'$. Så får vi ved eksekvering af $while(e) St$ i $sto$ final state $sto'$ hvis $e$ evaluerer til $true$ i $Env$.

\begin{align*}
&[WHILE_\top] & &Env \vdash \langle St, sto \rangle \rightarrow sto^{\prime\prime}\\
& & &\frac{Env \vdash \langle \text{while } (e)\; St,\; sto^{\prime\prime} \rangle \rightarrow sto^\prime}{Env \vdash \langle \text{while } (e)\; St,\; sto \rangle \rightarrow sto^\prime} & &\text{if } Env \vdash e \rightarrow_e \top\\\\
%
&[WHILE_\bot] & &Env \vdash \langle \text{while } (e)\; St,\; sto \rangle \rightarrow sto & &\text{if } Env \vdash e \rightarrow_e \bot\\\\
\end{align*}

\noindent I sproget findes også for-løkker. Disse er ikke beskrevet i semantikken, da disse på simpel vis bliver omskrevet til while-løkker inden code generation. I kodeeksempel \ref{code:fortowhile} ses hvordan en while-løkke, som er omskrevet fra en for-løkke, ser ud i konkret kode. Det tilsvarende \gls{ast} ses i figur \ref{fig:fortowhile}

\PPP{Kode/ForToWhile.ppp}{Omskrivning af for-løkker til while-løkker}{fortowhile}

\tikzfigure{Figurer/TikZ/ForToWhile.tex}{AST for For til While}{fortowhile}{1.0}

\noindent \textit{ASTSimplify}-klassen, som sørger for alle omskrivninger, indeholder desuden et tjek, hvor det tjekkes om der overhovedet er en condition i den pågældende for-løkke. Hvis ikke der er det, er det kun nødvendigt med en \textit{TrueExpr} - et eksempel på dette ses i kodeeksempel \ref{code:fortowhilesimple} og figur \ref{fig:fortowhilesimple}

\PPP{Kode/ForToWhileSimple.ppp}{Omskrivning af simple for-løkker til while-løkker}{fortowhilesimple}

\tikzfigure{Figurer/TikZ/ForToWhileSimple.tex}{Det resulterende AST for en omskrivning af simpel for-løkke}{fortowhilesimple}{1.0}

\subsubsection*{If statements}
$If$ statement er en af de selektive kontrolstrukturer der er valgt til at kunne selektere eksekvering af kode. $If$ statementet tager et boolsk udtryk der afgør om dens krop skal eksekveres. Hvis det boolske udtryk evaluerer til $false$ springes kroppen over. Dette $if$ statement fungerer som man kender det i C. Nedenfor ses semantikken for $if$ statementet.

\noindent For $[IF_\top]$ læses semantikken således: 

Hvis man i $Env$ eksekverer $St$ i $sto$ og får final state $sto'$, så får vi $sto'$ ved eksekvering af udtrykkel $if(e)  St$ i $sto$, hvis $e$ evaluerer til $true$ i $Env$.

\begin{align*}
&[IF_\top] & &\frac{Env \vdash \langle St, sto \rangle \rightarrow sto^\prime}{Env \vdash \langle \text{if } (e)\; St,\; sto \rangle \rightarrow sto^\prime} & &\text{if } Env \vdash e \rightarrow_e \top\\\\
\end{align*}

%%%%%%%%%%%%%%%%%%%%%%%%%%%%%%%%%%%%%%%%%%%%%%%%%%%%%
\subsubsection*{Structs}
For at kunne definere mere avancerede typer med tilhørende adfærd er structs en del af syntaksen. Structs kan indholde felter af andre typer som man kender i C. Som en udvikling af C's struct er funktioner i struct ligeledes blevet udviklet da det ønskes at kunne definere adfærd.

Nedenfor ses semantikken for 


    
    %Tabellen bliver hovedsageligt brugt i forbindelse med type checking af et \gls{ast}, men bliver også brugt i andre faser af kompileringsprocessen til at skrive, læse og dele information til/fra tabellen omkring variabler, funktioner og andet. For eksempel kan man vælge at udvide et \gls{ast} med information der bliver brugt til optimering af objekt koden.