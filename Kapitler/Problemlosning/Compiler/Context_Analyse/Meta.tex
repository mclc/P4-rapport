\subsection{Contextual Analysis}\wip
    Den kontekstuelle analyse af et givent program består af to dele; scope checking og type checking, og resultere i fejlrapporter der fortæller om type inkonsistensitet, og et \gls{ast} som er blevet \enquote*{dekoreret} med værdier der fortæller forskellige ting om noden, fx hvad type noden er, hvad variable der er referere til fra symbol tabellen og ligende. % Til dette bruges symbol tabellen i høj grad da den kan fortælle om hvad type bruger defineret variabler og funktioner er.
    
    Metoden til at gøre det er ved at iterere gennem \gls{ast}'et som blev gjort i syntax analyse afsnittet \ref{ssec:syntaxanalysis}. Der blev brugt et visitor mønster til at iterere gennem et \gls{ast}, det er så her double dispatch idéen som mønsteret løser kommer tilgode, for nu kan man meget let tilføje en ny handling der skal ske under interationen ved at tilføje en ny implementation af visitor interfacet.