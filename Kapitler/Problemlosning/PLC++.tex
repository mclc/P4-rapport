% Gadeberg 16-03
\section{Løsningsforslag - PLC++}
Ud fra problemformuleringen er der blevet lagt vægt på en løsning der tilgodeser et højere abstraktionsniveau samt en højere effektivitet ved udvikling af programmer.

Løsningen der er blevet valgt er et imperativt sprog der er inspireret af C. Motivationen for at vælge C som grundpille er C's udbredelse samt at mange sprog ligner C i deres syntax. Derfor vil sproget have en højere read og write-ability for en stor del af målgruppen.

Løsningsforslaget vil i rapporten blive omtalt som PLC++.

\subsection{Kravspecifikation}
Kravene til PLC++ tager udgangspunkt i C med en del ændringer. C bidrager til kravet om abstraktionsniveauet med dets mulighed for at benytte funktioner. Ligeledes har C datatypen \textit{struct} som kan imitere objekter i den virkelige verden.

De vigitgste krav til PLC++ vil blive gennemgået og forklaret. Den fulde syntax specifikation kan findes i bilag \mgfix{Bilag til syntax specifikation}

\subsubsection{Pointeraritmetik}
En af de vigtigste ting der blev valgt som afgørende for at mere effektivt programmeringssprog var afskaffelse af pointeraritmetik. I sproget skal pointeraritmetik fjernes så programmøren ikke eksplicit skal erklære hvad han/hun vil. Pointeraritmetik er kilde til potentielle fejlkilder og kan være en hindring for selv øvede programmører.

\subsubsection{Boolsk datatype}
Et andet krav er datatypen boolean \textit{boolean}. Den boolske datatype tilføjer en højere abstraktion af virkeligheden og som ved pointeraritmetik giver brugen af integers til boolsk udtryk også anledning til potentielle fejlkilder, samt et mindre korrekt sprog.

\subsubsection{Port}
Port er en datatype der bruges til digital og analogt in og output. Ligesom i ladderprogrammering bruges der bestemte tokens til in og output og dette er en ækvivalent til det.

\subsubsection{Timer}
En anden ting der er vurderet som et krav er lettere benyttelse af timere. Derfor er en timer en del af syntaksspecifikationen. Dette forhøjer read og writability.