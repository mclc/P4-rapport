\section{Compilere}
Compilere er grundstenen i alt programmering, og store dele af datalogien, som vi kender den i dag. De tidlige udgaver af programmer og styresystemer blev skrevet i ren maskine - en disciplin som de færreste mennesker er i stand til at udføre. Derfor blev konceptet med høj-niveau sprog opfundet. Det vil sige at man kan skrive et stykke software, med bogstaver og ord som er væsentligt mere læsbar for mennesket. Herefter transformerer et andet stykke software dette tekst om til maskinkode - dette er en compiler.


%\figur{Figurer/Compiler-opbygning.png}{Opbygning af en compiler}{Compiler-opbygning}{1}

\tikzfigure{Figurer/TikZ/CompilerOpbygning.tex}{Opbygningen af en compiler}{Compiler-opbygning}


\noindent Som nævnt har en compiler den funktion at omdanne et højniveau-sprog til maskinkode. Kigger man på figur \ref{fig:Compiler-opbygning} ses det dog at der flere andre ting undervejs.

\subsection{Syntax Analysis}
Den første del af en compiler, syntaksanalysen, har til ansvar at læse kildekoden som rent tekst og omdanne det til et abstrakt syntakstræ, som efterfølgende kan bruges i det videre arbejde mod maskinkode. Denne analyse er typisk opdelt i to underdele: Scanneren og parseren. 

\subsubsection{Scanner}
Scanneren sørger for at opdele kildekoden i såkaldte "tokens"\mbox{}, som er de mindste dele af et sprog. For at illustrere dette, ses et lille kodestykke i sproget Mini Triangle i kodeeksempel \ref{code:minitriangleforscanner}.

\MT{Kode/MiniTriangleForScanner.mt}{Lille program i Mini Triangle}{minitriangleforscanner}

\begin{table}[h]
\centering
    \begin{tabular}{|c|c|c|c|c|}
    \hline
    \textbf{let} & \textbf{var} & \textbf{identifier} & \textbf{colon} &            \textbf{identifier} \\ \hline
    let          & var          & y                   & :              & Integer             \\ \hline
    \end{tabular}
\caption{\textit{Opdeling af tokens}}
\label{tab:tokensMT}
\end{table}

\subsubsection{Parser}



\subsection{Contextual Analysis}

\subsection{Code Generation}