\section{Compilere}
Compilere er grundstenen i alt programmering, og store dele af datalogien, som vi kender den i dag. De tidlige udgaver af programmer og styresystemer blev skrevet i ren maskine - en disciplin som de færreste mennesker er i stand til at udføre. Derfor blev konceptet med høj-niveau sprog opfundet. Det vil sige at man kan skrive et stykke software, med bogstaver og ord som er væsentligt mere læsbar for mennesket. Herefter transformerer et andet stykke software dette tekst om til maskinkode - dette er en compiler.


%\figur{Figurer/Compiler-opbygning.png}{Opbygning af en compiler}{Compiler-opbygning}{1}

\tikzfigure{Figurer/TikZ/CompilerOpbygning.tex}{Opbygningen af en compiler}{Compiler-opbygning}


\noindent Som nævnt har en compiler den funktion at omdanne et højniveau-sprog til maskinkode. Kigger man på figur \ref{fig:Compiler-opbygning} ses det dog at der flere andre ting undervejs.
I syntaxanalysen modtages kildekoden som tekst...