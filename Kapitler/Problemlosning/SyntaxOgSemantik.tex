\section{Syntax og semantik}\label{sec:Syntax}

\mfix{Pt. indsat som bilag - bare for at kunne se kompileret udgave}

\subsection{Typer}

\subsubsubsection{Bool}
Den boolske datatype repræsenterer enten værdien sand eller falsk. Denne type kan ikke konverteres til andre typer ligesom i C eller C++. Alle boolske udtryk konverteres til denne type.

\subsubsection{Heltal}
Af heltalstyper er det valgt 2 forskellige: Int og Long. 

\subsubsection{Int}
Int datatypen er en signed integer på 16bit. Det første bit er signbittet og bestemmer om værdien er positiv eller negativ. Int kan maksimalt holde værdien 2\textsuperscript{15}-1 = 32767 og minimalt holde -2\textsuperscript{15} = -32768.

\subsubsection{Long}
Long er ligeledes en signed integer, men på 32 bits. Den maksimale værdi er 2\textsuperscript{31}-1 og den minimale værdi er -2\textsuperscript{31}.

\subsubsection{Kommatal}
Af kommatal er der valgt to forskellige med forskellige opløsninger. En Float og en Double. Float er 4 byte og Double er 8 Byte.

\subsubsection{Struct}
Struct er en indkapsulering af andre typer og/eller funktioner. På denne måde kan der laves mere komplekse typer end dem der i forvejen er til rådighed. En struct skal defineres i roden på samme måde som en funktion hvorefter den kan instantieres som en almindelig variabel.

\subsubsection{Array}
Et Array er en datastruktur som holder et antal af elementer af en bestemt type. Elementerne holdes i sekvens og man kan tilgå hvert enkelt efter alementets index.

\subsubsection{Port}
Port datatypen er adskilder sig fra de andre datatyper til at være mapped til en port på \gls{plc}'en.

Af porte findes der Digitale og Analoge inputs samt Digitale og Analoge outputs. De digitale porte kan have værdien sand eller falsk som en bool. De Analoge datatyper holder en numerisk værdi fra 0 til 1000 som afspejler den analoge ports værdi i promille. \mgfix{så opløsningen på Analoge ind og ud er 1/1000 det er et problem ved bla bla-- tag med i diskussion}

\subsubsection{Timer}
En timer er for så vidt muligt en hardcoded struct hvor keyworded struct ikke er nødvendigt. \mgfix{Søren mangler at få lavet denne type færdig - referer til Gadeberg}

\subsection{Udtryk}
Et udtryk er en sekvens af operatorer og operander. I PLC++ findes der 2 typer udtryk. Aritmetiske og boolske udtryk. Aritmetiske udtryk returnerer et hel- eller kommatal. Boolske udtryk returnerer en bool. Før der gåes i detaljen med de forskellige udtryk bliver vi nødt til at snakke om hvilke operatorer der kan benyttes.

\subsubsection{Operatorer}
I tabellen herunder ses operatorerne der kan benyttes i udtryk. Ved sammensatte udtryk med forskellige operatorer er operator prioritering nødvendig. Hvis ikke disse regler fastsættes kan man få forskellige resultater, alt efter hvilken rækkefølge udtrykket udregnes. Tabellen er ordnet efter prioritering hvor de øverste udregnes før de nederste. 
\begin{table}[H]
    \centering
    \begin{tabular}{|l|l|}
        \hline
        \centering

        Primær             & x++ \quad x- - \quad a{[}x{]} \quad x.y \quad (x)                 \\ \hline
        Unær               & ++x \quad - -x \quad ! \quad -x \quad +x \quad (cast)x       \\ \hline
        Multiplikationer   & * \quad / \quad \%                                               \\ \hline
        Additioner         & + \quad -                                                        \\ \hline
        Sammenligning      & \textless \quad \textgreater \quad \textless= \quad\textgreater= \\ \hline
        Ligheder           & == \quad !=                                                      \\ \hline
        Logisk og          & \&\&                                                              \\ \hline
        Logisk eller       & ||                                                               \\ \hline
        Betinget udtryk    & ?:                                                               \\ \hline
        Sammensatte udtryk & *= \quad /= \quad \%= \quad += \quad -=                          \\ \hline
        Assignment         & =                                                                \\ \hline

    \end{tabular}
    \caption{\textit{Operatorprioritering (højest til lavest)}}
    \label{tab:operatorprioritering}
\end{table}
\subsubsection{Boolske udtryk}

\textbf{And expression}
	% Mangler der ikke noget?
	%and_expr && equality_expr	
	
    \bgroup
    \def\arraystretch{3}
    \begin{table}[H]
    \centering
    \begin{tabular}{l c l}
        
        $[AND-1_{BS}]$ &$\frac{b_1 \rightarrow \; tt \quad b_2 \rightarrow \; tt}{b_1 \bigwedge b_2 \rightarrow \; tt}$ & \\
    
        $[AND-2_{BS}]$ &$\frac{b_i \rightarrow \; ff}{b_1 \bigwedge b_2 \rightarrow \; ff}$ & $i \in \{1, 2\}$\\
        
    \end{tabular}
    \caption{AND expression}
    \label{tab:andexpr}
    \end{table}
    \egroup

\noindent \textbf{Or expression}
% Mangler der ikke noget?
%or_expr || and_expr	

    \bgroup
    \def\arraystretch{3}
    \begin{table}[H]
    \centering
    \begin{tabular}{l c l}
        
        $[OR-1_{BS}]$ &$\frac{b_1 \rightarrow \; tt \quad b_2 \rightarrow \; ff}{b_1 \bigvee b_2 \rightarrow \; tt}$ & \\
    
        $[OR-2_{BS}]$ &$\frac{b_i \rightarrow \; ff}{b_1 \bigvee b_2 \rightarrow \; ff}$ & $i \in \{1, 2\}$\\
        
    \end{tabular}
    \caption{OR expression}
    \label{tab:orexpr}
    \end{table}
    \egroup
    
\noindent \textbf{Equality expressions}
    % Mangler der ikke noget?
	%equality_expr == greater_less_expr
  %|	equality_expr != greater_less_expr
  
    \bgroup
    \def\arraystretch{3}
    \begin{table}[H]
    \centering
    \begin{tabular}{l c l}
        
        $[EQUALS-T_{BS}]$ &$\frac{expr_1 \rightarrow v_1 \quad expr_2 \rightarrow v_2}{expr_1\;=\;expr_2 \rightarrow \; tt}$ & if $v_1 = v_2$ \\
        
        $[EQUALS-\bot_{BS}]$ &$\frac{expr_1 \rightarrow v_1 \quad expr_2 \rightarrow v_2}{expr_1\;=\;expr_2 \rightarrow \; ff}$ & if $v_1 \ne v_2$ \\
        
    \end{tabular}
    \caption{Equality expressions}
    \label{tab:equaexpr}
    \end{table}
    \egroup

\noindent \textbf{Greater and less than expressions}

    \begin{Grammar}
 \begin{grammar}

    <greater\_less\_expr> ::= <greater\_less\_expr> "<" <add\_sub\_expr>
    \alt <greater\_less\_expr> "<=" <add\_sub\_expr>								
    \alt <greater\_less\_expr> ">" <add\_sub\_expr>								
    \alt <greater\_less\_expr> ">=" <add\_sub\_expr>
  
 \end{grammar}
 \caption{Greater and less than expressions}\label{gra:greaterorless}
\end{Grammar}
    
    \bgroup
    \def\arraystretch{3}
    \begin{table}[H]
    \centering
    \begin{tabular}{l c l}
        
        $[GREATERTHAN-T_{BS}]$ &$\frac{expr_1 \rightarrow v_1 \quad expr_2 \rightarrow v_2}{expr_1\;< \;expr_2 \rightarrow \; tt}$ & if $v_1 <  v_2$ \\
        
        $[GREATERTHAN-\bot_{BS}]$ &$\frac{expr_1 \rightarrow v_1 \quad expr_2 \rightarrow v_2}{expr_1\;< \;expr_2 \rightarrow \; ff}$ & if $v_1 \geq v_2$ \\
        
    \end{tabular}
    \caption{Greater and less than expressions}
    \label{tab:greaterexpr}
    \end{table}
    \egroup

\noindent \subsubsection{Aritmetiske udtryk}

\noindent \textbf{Add and sub expressions}

    \bgroup
    \def\arraystretch{3}
    \begin{table}[H]
    \centering
    \begin{tabular}{l c l}
        
        $[PLUS_{BS}]$ &$\frac{expr_1 \rightarrow v_1 \quad expr_2 \rightarrow v_2}{expr_1\;+\;expr_2 \rightarrow v}$ & where $v = v_1 + v_2$ \\
        
        $[MINUS_{BS}]$ &$\frac{expr_1 \rightarrow v_1 \quad expr_2 \rightarrow v_2}{expr_1\;-\;expr_2 \rightarrow v}$ & where $v = v_1 - v_2$ \\
        
    \end{tabular}
    \caption{Add and sub expression}
    \label{tab:addandsub}
    \end{table}
    \egroup

    \begin{Grammar}
 \begin{grammar}

    <add\_sub\_expr> ::= <add\_sub\_expr> "+" <mult\_div\_mod\_expr>
    \alt <add\_sub\_expr> "-" <mult\_div\_mod\_expr>
  
 \end{grammar}
 \caption{Add and sub expressions}\label{gra:addsub}
\end{Grammar}
  
\noindent \textbf{MultDivMod}

    \begin{Grammar}
 \begin{grammar}

    <mult\_div\_mod\_expr> ::=	<mult\_div\_mod\_expr> "*" <unary\_prefix\_expr>
    \alt <mult\_div\_mod\_expr> "/" <unary\_prefix\_expr>	
    \alt <mult\_div\_mod\_expr> "\%" <unary\_prefix\_expr>
  
 \end{grammar}
 \caption{MultDivMod}\label{gra:MultDivMod}
\end{Grammar}
    
    \bgroup
    \def\arraystretch{3}
    \begin{table}[H]
    \centering
    \begin{tabular}{l c l}
        
        $[MULT_{BS}]$ &$\frac{expr_1 \rightarrow v_1 \quad expr_2 \rightarrow v_2}{expr_1\;*\;expr_2 \rightarrow v}$ & where $v = v_1 \cdot v_2$ \\
        
        $[DIV_{BS}]$ &$\frac{expr_1 \rightarrow v_1 \quad expr_2 \rightarrow v_2}{expr_1\;/\;expr_2 \rightarrow v}$ & where $v = \frac{v_1}{v_2}$ \\
        
        $[MOD_{BS}]$ &$\frac{expr_1 \rightarrow v_1 \quad expr_2 \rightarrow v_2}{expr_1\;\%\;expr_2 \rightarrow v}$ & where $v = v_1 mod v_2$ \\
        
        
    \end{tabular}
    \caption{MultDivMod}
    \label{tab:multdivmod}
    \end{table}
    \egroup

\noindent \textbf{Unary expressions}

    \begin{Grammar}
 \begin{grammar}

    <unary\_prefix\_expr> = "++"<identifier>
     \alt	"-" "-"<identifier>			
     \alt	"+"<unary\_prefix\_expr>						
     \alt	"-"<unary\_prefix\_expr>
     \alt "!"<unary\_prefix\_expr>
     \alt "(" <type\_specifier> ")"<unary\_prefix\_expr>
  
 \end{grammar}
 \caption{Unary Expressions}\label{gra:unary}
\end{Grammar}


 
    
    \bgroup
    \def\arraystretch{3}
    \begin{table}[H]
    \centering
    \begin{tabular}{l c l}
        
        $[NEG-T_{BS}]$ &$\frac{expr_1 \rightarrow_b\; tt}{expr_1 \rightarrow_b\; ff}$ &  \\
        
        $[NEG-\bot_{BS}]$ &$\frac{expr_1 \rightarrow_b\; ff}{expr_1 \rightarrow_b\; tt}$ &  \\
        
        $[UNARY-MINUS_{BS}]$ &$\frac{expr_1 \rightarrow\; v}{-expr_1 \rightarrow\; -v}$ &  \\
        
        $[PLUS-MINUS_{BS}]$ &$\frac{expr_1 \rightarrow\; -v}{+expr_1 \rightarrow\; v}$ &  \\
        
    \end{tabular}
    \caption{Unary expressions}
    \label{tab:unary}
    \end{table}
    \egroup
    
  
\noindent \textbf{Primary expressions}

    \begin{Grammar}
 \begin{grammar}

    <post\_and\_infix\_expr> ::= <identifier>"++"
     \alt <identifier>"-" "-"
     \alt <identifier> "(" <post\_and\_infix\_expr> ")"
     \alt <post\_and\_infix\_expr>"."<expr\_value>
     \alt "(" <expr> ")"
  
 \end{grammar}
 \caption{Primary Expressions}\label{gra:primary}
\end{Grammar}





    
    \bgroup
    \def\arraystretch{3}
    \begin{table}[H]
    \centering
    \begin{tabular}{l c l}
        
        $[INCREMENT_{BS}]$ & $\frac{expr_1 \rightarrow v_1}{expr_1++\; \rightarrow v}$ & where $v = v_1 + 1$ \\
        
        $[DECREMENT_{BS}]$ &$\frac{expr_1 \rightarrow v_1}{expr_1- -\; \rightarrow v}$ & where $v = v_1 - 1$ \\
        
        
    \end{tabular}
    \caption{Primary expressions}
    \label{tab:primary}
    \end{table}
    \egroup
    \mfix{På en måde skal der være forskel på pre og post fix increment.}


\noindent \subsection{Statements}

\noindent \subsubsection{While-løkke}

    \bgroup
    \def\arraystretch{3}
    \begin{table}[H]
    \centering
    \begin{tabular}{l c l}
        
        $[WHILE-TRUE_{BS}]$ & $
        \frac{\langle S, s\rangle  \rightarrow s^{\prime\prime} \quad \langle while\;b\;\{ S\},\;s^{\prime\prime}\rangle  \rightarrow s^{\prime}}
        {\langle while\;b\;\{ S\},\;s\rangle  \rightarrow s^{\prime}}
        $ & if $s \vdash b \rightarrow_b tt$ \\
        
        $[WHILE-FALSE_{BS}]$ & $
        \langle while\;b\;\{S\},\;s\rangle  \rightarrow s$ & 
        if $s \vdash b \rightarrow_b ff$ \\

    \end{tabular}
    \caption{While-løkke}
    \label{tab:while}
    \end{table}
    \egroup
    
Da while-løkker og for-løkker er semantiske ækvivalente, og for-løkker omskrives til while-løkker inden code generation, er det ikke relevant at opskrive en semantik for for-løkker.

\noindent \subsubsection{If-else statements}

    \bgroup
    \def\arraystretch{3}
    \begin{table}[H]
    \centering
    \begin{tabular}{l c l}
        
        $[IF-TRUE_{BS}]$ & $
        \frac{\langle S_1,\;s\rangle  \rightarrow s^\prime}
        {\langle if\;b\;\{S_1\}\;else\;\{S_2\},\;s\rangle  \rightarrow s^\prime}
        $ & if $s \vdash b \rightarrow_b tt$ \\
        
        $[IF-FALSE_{BS}]$ & $
        \frac{\langle S_2,\;s\rangle  \rightarrow s^\prime}
        {\langle if\;b\;\{S_1\}\;else\;\{S_2\},\;s\rangle  \rightarrow s^\prime}
        $ & if $s \vdash b \rightarrow_b ff$ \\
        
    \end{tabular}
    \caption{If-else statements}
    \label{tab:ifelse}
    \end{table}
    \egroup

\noindent \subsection{Declarations}
Declarationer kommer i 2 forskellige varianter. En med initialisering og en uden.

\begin{Grammar}
 \begin{grammar}

  <declaration> ::= <type\_qualifier>? <type\_specifier> <identifier> <array\_definition>? ";"
  \alt <type_qualifier>? <type_specifier> <identifier> <array_definition>? "=" <expr> ";"
  
  
 \end{grammar}
 \caption{Declaration med/uden initialisering}\label{gra:declaration}
\end{Grammar}