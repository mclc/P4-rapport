%Claus - 18/02/2015
%Gade - 24-03-15
\section{Paradigme}\label{sec:paradigmer}

Et paradigme er på Oxford Onlinedictionary beskrevet som et typisk eksempel eller et mønster \cite{Oxford_????}. I programmeringssammenhæng forstås det som at være de overordnede principper for et programmeringssprog. Der findes i dag over 30 forskellige paradigmer, men mange af dem er kun forskellige på meget få punkter \cite{Roy_2009}. Derfor vil kun tre af hovedparadigmerne bliver forklaret: Logisk, imperitivt og objektorienteret.\sfix{Hvorfor er funktionelt ikke inkluderet - De 4 hovedparadigmer anyone?}

\subsection{Hovedparadigmer}\label{ssec:hovedparadigmer}
\subsubsection*{Imperativt}
Da man begyndte at udvikle generelle programmeringssprog der skulle gøre det lettere at formulere programmer, havde effektiviteten af programmet højeste prioritet. Derfor var det naturligt at man byggede ovenpå den grundlæggende arkitektur der stadig findes i moderne computere; von Neumann arkitekturen. Dette gav grundlag for mange af de imperative sprog der i dag findes, disse sprog er en abstraktion over arkitekturen og giver programmøren kontrol over tilpasning og optimering af enkelte programdele \cite[38-39]{Sebesta_2013}. I de imperative sprog er det variabler, der holder styr på stadiet programmet er i, og løkker der er de centrale funktioner \cite{Sebesta_2013}.

\subsubsection*{Logisk}
Som software blev dyrere at producere, relativt til hardware, opstod der et behov for en logisk fremgang, for at nedsætte den tid programmørene skulle bruge på at fortælle computeren hvordan den skal løse en opgave, men i stedet hvad den skulle løse. Til dette blev logisk programmering, der går ud på at man opstiller nogle regler, som compileren finder ud af hvordan og hvornår skal benyttes. Styrkerne ved dette paradigme er at den færdige kildekode er en type logisk definition og kan derfor lettere læses og vedligeholdes \cite[kapitel 16]{Sebesta_2013}.

\subsubsection*{Objekt-orienteret}
Objekt-orienteret programmering blev udviklet da det imperative paradigme viste sig at blive for kompleks til større programmer med meget data. Derfor blev der udviklet sprog baseret på det objektorienterede paradigme. Sprog i dette paradigme abstraherer fra det imperative paradigme ved at indkapsle funktioner og variabler i klasser. På den måde kan man via klasser modellere objekter i virkeligheden med data og metoder. En klasse er på den måde et opskrift hvoraf objekter kan laves/instantieres.

\subsection{Valg af paradigme}\label{ssec:paradigmevalg}
Ved \gls{plc} programmering er størstedelen af formålet at styre mekaniske anlæg og kommunikere med andre enheder som \gls{hmi} paneler via digitalt \gls{io}. Disse programmeringsopgaver er fortrinsvis simple i kontekst, da I/O i forvejen er forholdsvis simple. Der er derfor ikke behov for at kunne modellere komplekse objekter i virkeligheden. Dog skal det nævnes, at selv om I/O er simple opgaver i kontekst, skal det være muligt at kunne opdele programmet i flere abstraktioner, da et anlæg kan være - og ofte er - opbygget i flere sektioner/dele.

Ud fra disse betragtninger er det blevet vurderet, at det imperative paradigme passer bedst. Det løser problemet med at skulle kunne opdele anlægsdele i forskellige abstraktionsniveauer. Ligeledes indeholder det ikke features som er mindre relevante og primært kun bidrager til øget kompleksitet. Features som at kunne modellere objekter i virkeligheden, som ved objekt orienteret programmering. Det er også værd at nævne, at det logiske paradigme ikke løser problemet med abstraktionsniveauer og derfor ikke er taget i betragtning som en løsning.