%Claus - 18/02/2015
\section{Program Paradigme}\label{sec:paradigmer}

Et paradigme er på Oxford Onlinedictionary beskrevet som et typisk eksempel eller et mønster \cite{Oxford_????}. I programmeringssammenhæng forstås det som at være de overordnede principper for et programmeringssprog. Der findes i dag over 30 forskellige paradigmer, men mange af dem er kun forskellige på meget få punker \cite{Roy_2009}. Derfor vil kun tre af hovedparadigmerne bliver forklaret: Logisk, imperitivt og objektorienteret \cite{Normark_2003}.\cfix{Kilden snakker om fire hoved paradigmer.}

\subsection{Tre hovedparadigmer}\label{ssec:hovedparadigmer}
\paragraph{Imperativt}
Da man begyndte at udvikle generelle programmeringssprog der skulle gøre det lettere at formulere programmer, havde effektiviteten af programmet højeste prioritet. Derfor var det naturligt at man byggede ovenpå den grundlæggende arkitektur der stadig findes i moderne computere; von Neumann arkitekturen. Dette gav grundlag for mange af de imperative sprog der i dag findes, disse sprog er en abstraktion over arkitekturen og giver programmøren kontrol over tilpasning og optimering af enkelte programdele \cite[38-39]{Sebesta_2013}. I de imperative sprog er det variabler, der holder styr på stadiet programmet er i, og løkker der er de centrale funktioner \cite{Sebesta_2013}.

\paragraph{Logisk}
Som software blev dyre at producere, relativt til hardware, opstod der et behov for en logisk fremgang, for at nedsætte den tid programmørene skulle bruge på at fortælle computeren hvordan den skal løse en opgave, men i stedet hvad den skulle løse. Til dette blev logisk programmering, der går ud på at man opstiller nogle regler, som kompileren finder ud af hvordan og hvornår skal benyttes. Styrkerne ved dette paradigme er at den færdige kildekode er en type logisk definition og kan derfor lettere læses og vedlige holdes \cite[kapitel 16]{Sebesta_2013}. Det har vist sig at dette paradigme er lettere at lære for programmøre, der ikke er så matematisk indstillet, og derfor bliver dette paradigme brugt i industrien. \cfix{Citation needed}

\paragraph{Objekt-orienteret}
Objekt-orienteret programmering blev udviklet da det imperative paradigme viste sig at blive for kompleks i større programmer med meget data. Derfor blev der udviklet sprog baseret på det objektorienterede paradigme. Sprog i dette paradigme abstraherer fra det imperative paradigme ved at indkapsle funktioner og variabler i klasser, som kan opstå flere gange som instanser. En instans er en kopi af klassen hvis stadie kan ændres uden at påvirke andre instanser.

\subsection{Valg af paradigme}\label{ssec:paradigmevalg}

