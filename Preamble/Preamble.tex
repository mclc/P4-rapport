% Sæt encoding så indstillingerne kan loades korrekt
\usepackage[utf8]{inputenc}

% Indlæs indstillinger
% Settings - udkommenter for at slå fra
\newcommand*{\draft}{} % Skal der compiles som draft
\newcommand*{\CMDCode}{} % Kode kommandoer
\newcommand*{\CMDFixMe}{} % Aktiver fixme
\newcommand*{\CMDTikZ}{} % Aktiver TikZ

% Sæt dokumentinformationerne
\newcommand{\rapportnavn}{PLC++}
\newcommand{\gruppenummer}{SW402f15}
\newcommand{\afldate}{Maj 2015}
\newcommand{\sprognavn}{PLC++}

% Ekstra kommandoer
\newcommand{\figuregroup}{}
\newcommand{\tabelgroup}{}
\newcommand{\codegroup}{}
\newcommand{\screenshotgroup}{}
\newcommand{\lastseen}[1]{Sidst set d. #1}

% Definer forkortelser
\let\newfootnote\footnote % Kopier footnote
\newcommand{\fn}[2]{\newfootnote{#1: #2}}
\renewcommand{\footnote}{\errmessage{Brug fn i stedet for footnote}}           

% Indlæs packages
\usepackage[backend=bibtex,             % Set standard format for database, bibtex or biber
            style=ieee,                 % Style of bibliograhy
            sorting=none                % What order to sort by; none, nty(name, title, year),
        ]{biblatex}                     % Bibliografi
\usepackage{tabularx}                   % Til tabeller
\usepackage{fourier}                    % Font
\usepackage[danish]{babel}              % Dansk sprog
\usepackage{a4}                         % A4 format
\usepackage[margin=3.0cm]{geometry}     % Til at sætte margin
\usepackage{fancyhdr}                   % Pæn header og footer
\usepackage{lastpage}                   % Til at finde sidste side
\usepackage{hyperref}                   % Referencer i PDF dokumentet og referencer til labels
\usepackage{float}                      % Til at sætte figurer korrekt
\usepackage{url}                        % Til at lave URL
\usepackage{titlesec}                   % Til at ændre på størrelsen på sections
\usepackage{xr}                         % Gør det muligt at cross referere i andre filer
\usepackage{amsmath}                    % Matematik-ting
\usepackage{mathtools}                  % Flere matematik-ting
\usepackage{multirow}                   % Multirows i tabel
\usepackage{threeparttable}             % Til fodnoter i tabeller
\usepackage{csquotes}                   % Udvidelser til quotations (brugt af biblatex)
%\usepackage{natbib}                     % Udvidelse af kilder
\usepackage[toc,page]{appendix}         % Til bilag
\usepackage{enumitem}                   % Til lister
\usepackage[boxruled,linesnumbered,noend]{algorithm2e} % Til algoritmer
\usepackage{graphicx}                   % Billeder
\usepackage{pdfpages}                   % PDF sider
\usepackage{color}                      % Farver
\usepackage{soulutf8}                   % Til highlight
\usepackage{multicol}                   % Til itemize med colums
\usepackage{tikz}                       % Til figurer og lign.
\ifdefined\draft\usepackage{Preamble/Packages/currfile}\fi% Til indsættelse af filnavn - kun hvis draft er til
\ifdefined\draft\usepackage[inline]{showlabels}\fi % Til at vise label name - kun hvis draft er til

% Sæt overskrifter på algoritmer
% http://tex.stackexchange.com/questions/41012/change-the-algorithm-enumeration-in-the-algorithm2e-package
\renewcommand{\listalgorithmcfname}{Algoritmer}%
\renewcommand{\algorithmcfname}{Algoritme}%
\renewcommand{\algorithmautorefname}{algoritme}%
\renewcommand{\algorithmcflinename}{linje}%
\renewcommand{\procedureautorefname}{procedure}%

% Dokument informationskommandoer
\newcommand{\currentpage}{\thepage}     % nuværende side
\newcommand{\numpages}{\pageref{LastPage}}% antal sider
\newcommand{\sidetal}{Side {\currentpage} af {\numpages}} % side informationer
\newcommand{\pagetitle}{\rightmark}     % nuværende sektion
\newcommand{\comment}[1]{}

% Definerer titel osv.
\title{\rapportnavn}
\author{\gruppenummer}
\date{\afldate}

% Setup af header og footer
\fancypagestyle{plain}{
    % Nulstil header og footer
    \fancyhead{}
    \fancyfoot{}

    % Setup header
    % Left / Right - Even / Odd
    \fancyhead[LE]{\rapportnavn}        % Lige sider
    \fancyhead[RO]{\pagetitle}          % Ulige sider

    % Sæt footer op
    \fancyfoot[LE]{\sidetal}            % Lige sider
    \fancyfoot[RO]{\sidetal}            % Ulige sider
    
    % Sæt kun filnavn på en draft
    \ifdefined\draft
        \fancyfoot[LO]{\currfilepath}   % Ulige sider
        \fancyfoot[RE]{\currfilepath}   % Lige sider
    \fi
}

% Sætter pagestyle til plain - header og footer
\pagestyle{plain}

% Fjerner ligegyldig whitespace - men rykker ting op så vi ikke har teksten på slut
\raggedbottom

% Fjerner punktummet efter sektion-nummer i header (2.2. XXX --> 2.2 XXX)
\renewcommand\sectionmark[1]{%
  \markright{\MakeUppercase{\textbf{\thesection}\ #1}}}

% Fjerner afstand mellem kapitel og sætter størrelsen på skriften - se evt. her: http://tex.stackexchange.com/questions/63390/how-to-decrease-spacing-before-chapter-title
\titleformat{\chapter}[display]{\normalfont\huge\bfseries}{\chaptertitlename\ \thechapter}{20pt}{\Huge}
\titlespacing*{\chapter}{0pt}{0pt}{20pt}

% Indent til eller fra
\newcommand{\enableIndent}{\setlength\parindent{12pt}}
\newcommand{\disableIndent}{\setlength\parindent{0pt}}
\enableIndent

% Opretter subsubsubsection og subsubsubsubsection
\newcommand{\subsubsubsection}[1]{\noindent\paragraph{#1}\mbox{}\\}
\newcommand{\subsubsubsubsection}[1]{\noindent\subparagraph{#1}\mbox{}\\}

% Skriftstørrelse på sections
\titleformat*{\section}{\LARGE\bfseries}
\titleformat*{\subsection}{\Large\bfseries}
\titleformat*{\subsubsection}{\large\bfseries}
\titleformat*{\paragraph}{\normalsize\bfseries}
\titleformat*{\subparagraph}{\normalsize\bfseries}

% Definerer antallet af overskrifter der har tal, samt antallet af overskrifter i indholdsfortegnelsen
\setcounter{secnumdepth}{3}             % Antallet af overskrifter der har et nr - dybden
\setcounter{tocdepth}{2}                % Antallet af overskrifter i indholdsfortegnelse - dybden - eks. 1.1.1

% Orddeling
\hyphenation{}

% Ændrer indstillingerne for caption for figurer, tabeller og kodestykker - tilføjer bla. kusiv og gør typenavnet bold (eks. Figur 1.1: bliver fed)
\usepackage[font=small,format=plain,labelfont=bf,up,textfont=it,up]{caption}

% FixMe
\ifdefined\CMDFixMe
    % FixMe
\ifdefined\draft
    \usepackage[draft,danish]{fixme}    % Fixme draft
\else
    \usepackage[final,danish]{fixme}    % Fixme final
\fi

% FixMe
% http://mirrors.dotsrc.org/ctan/macros/latex/contrib/fixme/fixme.pdf
\newcommand{\fix}[2]{\fxfatal{\textcolor{red}{#1} /#2}}
\newcommand{\note}[2]{\fxnote{#1 /#2}}
\newcommand{\anfix}[3]{
    \begin{anfxfatal}{\textcolor{red}{#2} /#3} 
        \textcolor{red}{#1} /#3
    \end{anfxfatal}
}

\renewcommand\fxdanishfatalname{Fejl}
\renewcommand\fxdanishnotename{Note}
\renewcommand{\fixme}[1]{\fix{Benyt ikke fixme-command}}

%Slet fixme
%\let\fixme\relax
%Fix
\newcommand{\sfix}[1]{\fix{#1}{Søren}}
\newcommand{\jfix}[1]{\fix{#1}{Jimmi}}
\newcommand{\mefix}[1]{\fix{#1}{Edelskjold}}
\newcommand{\mfix}[1]{\fix{#1}{Mikkel}}
\newcommand{\cfix}[1]{\fix{#1}{Claus}}
\newcommand{\mgfix}[1]{\fix{#1}{Gadeberg}}

\newcommand{\notdonefix}[1]{\fxnote{\textcolor{blue}{Work in progress (#1)}}}
\newcommand{\notdone}[1]{\notdonefix{#1}}
\newcommand{\wip}{\notdonefix{}}
\fi

% Source code
\ifdefined\CMDCode
    % Sourcecode highlight i utf8 format
\usepackage{listingsutf8}

% Danske bogstaver i kodefiler
\lstset{literate=%
    {æ}{{\ae}}1
    {å}{{\aa}}1
    {ø}{{\o}}1
    {Æ}{{\AE}}1
    {Å}{{\AA}}1
    {Ø}{{\O}}1
}

% Source code tekster
\renewcommand{\lstlistingname}{Kodeeksempel}
\renewcommand{\lstlistlistingname}{Kodeeksempler}

% Kode kommandoer
\newcommand{\kode}[3]{\fix{Benyt ikke kode kommandoen}}
%Definer farver
\definecolor{gray95}{gray}{.95}
\definecolor{bluekeywords}{rgb}{0.13,0.13,1}
\definecolor{greencomments}{rgb}{0,0.5,0}
\definecolor{redstrings}{rgb}{0.9,0,0}

% Definer style
\lstdefinestyle{Default}
{ 
    numbers=left,
	numbersep=5pt,
	stepnumber=1,
	captionpos=b,
	keywordstyle=\color{bluekeywords},
	commentstyle=\color{greencomments},
	stringstyle=\color{redstrings},
	backgroundcolor=\color{gray95},
	frame=lrtb,
	framerule=0.5pt,
	linewidth=1.00\textwidth,
	tabsize=4,
	numberbychapter=true,
	basicstyle=\ttfamily\footnotesize, %\ttfamily
	breaklines=true,
	breakatwhitespace=true,
	showstringspaces=false,
	showspaces=false,
	showtabs=false,
	breakindent=20pt
}

% Indlæs kode fra fil med style
\newcommand{\kodeprintstyle}[5]{
    \lstinputlisting[language=#4, style=Default, caption={#2} ({#5}), label={code:#3}]{#1}
}
% Indlæs alt C-kode i fil
\newcommand{\AnsiC}[3]{
    \kodeprintstyle{#1}{#2}{#3}{C}{C}
}
\lstdefinelanguage{CSharp} {
    language=[Sharp]C,
    morekeywords={var},
    morekeywords={from,where,orderby,descending,select}, % LINQ
    morekeywords={get,set}
}

\newcommand{\CSharp}[3]{
    \kodeprintstyle{#1}{#2}{#3}{CSharp}{C\#}
}
\lstdefinelanguage{xaml} {
    morekeywords={BooleanExpression},
    alsoletter={:,,/,?},
    morestring=[b]{"},
    morecomment=[s]{&lt;!--}{--&gt;},keywordstyle=\color{forestGreen},
    morekeywords={TypeArguments,Name,Default,DisplayName,OperationName,ServiceContractName,Key,AddressUri,
CanCreateInstance, LogName, Message, MessageNumber, Expression,CorrelationHandle,Request}
}

\newcommand{\XAML}[3]{
    \kodeprintstyle{#1}{#2}{#3}{xaml}
}
\newcommand{\XML}[3]{
    \kodeprintstyle{#1}{#2}{#3}{XML}
}
\newcommand{\sql}[3]{
    \kodeprintstyle{#1}{#2}{#3}{SQL}
}
\newcommand{\html}[3]{
    \kodeprintstyle{#1}{#2}{#3}{HTML}{HTML}
}
\colorlet{punct}{red!60!black}
\definecolor{background}{HTML}{EEEEEE}
\definecolor{delim}{RGB}{20,105,176}
\colorlet{numb}{magenta!60!black}

\lstdefinelanguage{json}{
    basicstyle=\normalfont\ttfamily,
    numbers=left,
    numberstyle=\scriptsize,
    stepnumber=1,
    numbersep=8pt,
    showstringspaces=false,
    breaklines=true,
    frame=lines,
    backgroundcolor=\color{background},
    literate=
     *{0}{{{\color{numb}0}}}{1}
      {1}{{{\color{numb}1}}}{1}
      {2}{{{\color{numb}2}}}{1}
      {3}{{{\color{numb}3}}}{1}
      {4}{{{\color{numb}4}}}{1}
      {5}{{{\color{numb}5}}}{1}
      {6}{{{\color{numb}6}}}{1}
      {7}{{{\color{numb}7}}}{1}
      {8}{{{\color{numb}8}}}{1}
      {9}{{{\color{numb}9}}}{1}
      {:}{{{\color{punct}{:}}}}{1}
      {,}{{{\color{punct}{,}}}}{1}
      {\{}{{{\color{delim}{\{}}}}{1}
      {\}}{{{\color{delim}{\}}}}}{1}
      {[}{{{\color{delim}{[}}}}{1}
      {]}{{{\color{delim}{]}}}}{1},
}

\newcommand{\json}[3]{
    \kodeprintstyle{#1}{#2}{#3}{json}{JSON}
}
\lstdefinelanguage{MT} {
    language=C,
    morekeywords={var,let,Integer,in}
}

\newcommand{\MT}[3]{
    \kodeprintstyle{#1}{#2}{#3}{MT}{Mini Triangle}
}
% Til SableCC "kode"

\lstdefinelanguage{SCC} {
    %language=C,
    %morekeywords={var,let,Integer,in}
}

\newcommand{\SCC}[3]{
    \kodeprintstyle{#1}{#2}{#3}{SCC}{SableCC}
}
\newcommand{\Java}[3]{
    \kodeprintstyle{#1}{#2}{#3}{Java}{Java}
}
\fi

% TikZ
\ifdefined\CMDTikZ
    %Libraries
\usetikzlibrary{arrows,shapes,positioning,calc,shadows, automata}
%Path styles
\tikzstyle{line}  = [thick, draw, -latex']
\tikzstyle{arrow} = [line, ->, >=stealth, shorten >=1pt]
\tikzstyle{generalization} = [arrow, >=open triangle 45]   %Requires arrows tikzpackage
\tikzstyle{aggregation} = [line, <-, >=diamond]           %Requires arrows tikzpackage

%Node styles
\tikzstyle{class} = [rectangle split, rectangle split parts=3, draw, minimum width=2cm, fill=pink] %Requires shapes tikzpackage
\tikzstyle{box} = [rectangle, draw, minimum width=3cm, fill=black!5, align=left, minimum height=1cm]
\tikzstyle{note} = [box, draw=black!50]
\newcommand*{\tikzclass}[5]{\node[class, #2](#1){\nodepart[align=center]{one} \bfseries #3 \nodepart[align=left]{two} #4 \nodepart[align=left]{three} #5};}
\fi

% Kildeliste
%\bibpunct[, ]{[}{]}{;}{n}{,}{,} 		% Definerer de 6 parametre ved Harvard henvisning (bl.a. parantestype og seperatortegn)

% Udseende af litteraturlisten
%\bibliographystyle{Bibtex/Vancouver} 
%\bibliographystyle{unsrtnat}           % Inkluderer bl.a. isbn numre
\bibliography{Bibtex/Kilder}            % Inkludere kilder

% Oprettelse af links inde i pdf dokumentet
\hypersetup{
    pdftitle={\rapportnavn},
    pdfauthor={\gruppenummer},
    pdfsubject={\rapportnavn},
    bookmarksnumbered=true,
    bookmarksopen=false,
    bookmarksopenlevel=1,
    colorlinks=false,
    pdfstartview=Fit,
    pdfpagemode=UseOutlines,
    pdfpagelayout=TwoPageRight,
    pdfborder = {0 0 0}
}

% Badness underfull og overfull
\hbadness=10001                         % Slår alle underfull badness warnings fra
\hfuzz=1000pt                           % Slår de ligegyldige overfull warnings fra

% Figurer
\newcommand{\figur}[5][0]{
		\begin{figure}[H] \centering \em %H / h!
			\includegraphics[width=#5\textwidth, angle=#1]{#2}
			\caption{#3}\label{fig:#4}
		\end{figure}
}

\newcommand{\tikzfigure}[4]{
\begin{figure}[H] \centering
    \input{#1}
    \caption{#2}\label{fig:#3}
\end{figure}
}

% Inkluder ikke i ToC, men behold section number
\newcommand{\nocontentsline}[3]{}
\newcommand{\tocless}[2]{\bgroup\let\addcontentsline=\nocontentsline#1{#2}\egroup}

% Environments
\newenvironment{itemize_small}{
\begin{itemize}
  \setlength{\itemsep}{1pt}
  \setlength{\parskip}{0pt}
  \setlength{\parsep}{0pt}
}{\end{itemize}}

\newenvironment{enumerate_small}{
\begin{enumerate}
  \setlength{\itemsep}{1pt}
  \setlength{\parskip}{0pt}
  \setlength{\parsep}{0pt}
}{\end{enumerate}}

%For at kunne skrive grammatik med caption og label
\DeclareFloatingEnvironment[
  % the file extension for the file used to create the list:
 fileext   = logr,% don't use log here!
  % the heading for the list:
 listname  = {Grammatik},
  % the name used in captions:
  name      = Grammatik,
  % the default floating parameters if the environment is used
  % without optional argument:
  placement = H
]{Grammar}

\grammarindent=80pt